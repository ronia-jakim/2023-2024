\documentclass{article}

\usepackage{../../template}

\geometry{
%  showframe,
  a4paper, 
  total={170mm, 257mm}, 
  top=20mm, 
  marginparwidth={0mm},
  left={20mm},
  %right={26mm},
  headsep=5mm
}

\usepackage{array}
\usepackage{multirow}

\renewcommand{\arraystretch}{1.2}

\usepackage{listings}
\lstset{%
    %backgroundcolor=\color{yellow!20},%
    basicstyle=\ttfamily,
    breaklines=true,
    frame=single,
    rulecolor=\color{red!60!black!90}
    }

% Add your keywords here, and have this in a separate file
% and include it in your preamble
\lstset{emph={%  
    pi, sigma, x, tau, gamma, count, join
    },
    emphstyle={\color{blue}\bfseries}%
}%

\title{Lista 1}
\author{Weronika Jakimowicz}
\date{29.02.2024}

\begin{document}
\maketitle 

\begin{problem}
  Relacja $P(A, B)$ zawiera $p$ krotek, a relacja $S(B, C)$ zawiera $s$ krotek. Nic nie wiadomo na temat kluczy relacji. Dla każdego z poniższych wyrażeń wylicz (w zależności od $p$ i $s$) jaka może być minimalna i maksymalna liczba zwracanych krotek.
  \begin{enumerate}[label=\alph*)]
    \item $P\cup \rho_{S(A, B)}S$
    \item $\pi_{A, C}(P\bowtie S)$
    \item $\pi_B(P)\setminus(\pi_B(P)\setminus \pi_B(S))$
    \item $(S\bowtie S)\bowtie S$
    \item $\sigma_{A<B}(P)\cup \sigma_{A>B}(P)$
  \end{enumerate}
\end{problem}

\begin{solution}
  \textbf{\color{green}a) $P\cup \rho_{S(A, B)}S$}

  Operacja $\rho_{S(A, B)}S$ bierze całą relację $S(B, C)$ i podmienia nazwy jej kolumn na $A, B$, resztę pozostawiając bez zmian.

  {\color{blue}$\min=\max(s, k)$}

  W takim razie, najmniej krotek w sumie mnogościowej $P$ i $S(A, B)$ będzie jeśli cała jedna baza ma te same informacje co druga, np. gdy mamy takie, troszkę bezsensowne, bazy:
  \begin{center}
    \begin{tabular}{| m{0.05\textwidth} | m{0.1\textwidth} | m{0.1\textwidth} | m{0.08\textwidth} | m{0.15\textwidth} |}
      \cline{1-2}\cline{4-5}
      \multicolumn{2}{| c |}{P} & & \multicolumn{2}{ c |}{S} \\ 
      \cline{1-2}\cline{4-5}
      \slshape A=nr & \slshape B=imię & & \slshape B=imię & \slshape C=właściciel\\ 
      \cline{1-2}\cline{4-5}
      1 & Weles & & 2 & Kycia \\ 
      \cline{1-2}\cline{4-5} 
      2 & Kycia \\
      \cline{1-2}
    \end{tabular}
  \end{center}
  W takim razie najmniejsza ilość krotek to $\max(s, k)$.

  {\color{blue}$\max=s+p$}

  Najwięcej krotek będzie, gdy bazy po przemianowaniu będą całkowicie rozłączne, np.:
  \begin{center}
    \begin{tabular}{| m{0.1\textwidth} | m{0.1\textwidth} | m{0.1\textwidth} | m{0.08\textwidth} | m{0.1\textwidth} |}
      \cline{1-2}\cline{4-5}
      \multicolumn{2}{| c |}{P} & & \multicolumn{2}{ c |}{S} \\ 
      \cline{1-2}\cline{4-5}
      \slshape A=owoc &\slshape B=ilość & & \slshape B=ilość & \slshape C=zakup\\ 
      \cline{1-2}\cline{4-5}
      banan & 1 & & 5 & kartofli \\ 
      \cline{1-2}\cline{4-5} 
      winogran & 7 \\ 
      \cline{1-2}
      jabłka & 3\\
      \cline{1-2}
    \end{tabular}
  \end{center}
  \bigskip

  \textbf{\color{green}b) $\pi_{A, C}(P\bowtie S)$}

  {\color{blue}$\min=0$}

  Jeśli w $B$ nie ma wspólnych wpisów, to $P\bowtie S$ będzie pusty, czyli najmniejsza ilość krotek w wyniku to 0:
  \begin{center}
    \begin{tabular}{| m{0.05\textwidth} | m{0.1\textwidth} | m{0.1\textwidth} | m{0.08\textwidth} | m{0.15\textwidth} |}
      \cline{1-2}\cline{4-5}
      \multicolumn{2}{| c |}{P} & & \multicolumn{2}{ c |}{S} \\ 
      \cline{1-2}\cline{4-5}
      \slshape A=nr & \slshape B=imię & & \slshape B=imię & \slshape C=właściciel\\ 
      \cline{1-2}\cline{4-5}
      1 & Weles & & Stefan & Ania \\ 
      \cline{1-2}\cline{4-5} 
      2 & Kycia \\
      \cline{1-2}
    \end{tabular}

    \begin{tabular}{|m{0.15\textwidth} | m{0.15\textwidth} | m{0.15\textwidth} |}
      \hline 
      \multicolumn{3}{|c|}{$P\bowtie S$}\\ 
      \hline 
      \slshape A=nr & \slshape B=imię & \slshape C=właściciel \\ 
      \hline 
      - & - & -\\ 
      \hline
    \end{tabular}
  \end{center}

  {\color{blue}$\max=p\cdot s$}

  Jeśli $B$ ma tylko jeden element i w $P$ i w $S$, to wtedy $\bowtie$ zachowuje się jak produkt kartezjański
  \begin{center}
    \begin{tabular}{| m{0.05\textwidth} | m{0.1\textwidth} | m{0.1\textwidth} | m{0.08\textwidth} | m{0.15\textwidth} |}
      \cline{1-2}\cline{4-5}
      \multicolumn{2}{| c |}{P} & & \multicolumn{2}{ c |}{S} \\ 
      \cline{1-2}\cline{4-5}
      \slshape A=nr & \slshape B=imię & & \slshape B=imię & \slshape C=właściciel\\ 
      \cline{1-2}\cline{4-5}
      1 & Weles & & Weles & Kycia \\ 
      \cline{1-2}\cline{4-5} 
      2 & Weles & & Weles & Mirek \\ 
      \cline{1-2} \cline{4-5} 
      \multicolumn{3}{c|}{} & Weles & Ronia\\ 
        \cline{4-5}
    \end{tabular}
    
    \begin{tabular}{|m{0.15\textwidth} | m{0.15\textwidth} | m{0.15\textwidth} |}
      \hline 
      \multicolumn{3}{|c|}{$P\bowtie S$}\\ 
      \hline 
      \slshape A=nr & \slshape B=imię & \slshape C=właściciel \\ 
      \hline 
      1 & Weles & Kycia\\
      \hline 
      1 & Weles & Ronia \\ 
      \hline 
      1 & Weles & Mirek \\ 
      \hline 
      2 & Weles & Kycia\\ 
      \hline 
      2 & Weles & Mirek \\ 
      \hline 
      2 & Weles & Ronia \\
      \hline
    \end{tabular}
  \end{center}
  i wtedy krotek jest tyle, ile elementów w większej relacji, czyli $s\cdot p$.
  \bigskip

  \textbf{\color{green}c) $\pi_B(P)\setminus(\pi_B(P)\setminus\pi_B(S))$}

  {\color{blue}$\min=0$}

  Tutaj minimum to $0$, jeśli zapisy w kolumnie $B$ relacji $S$ są całkiem rozłączne z zapisami w kolumnie $B$ relacji $P$:
  
  \begin{center}
    \begin{tabular}{| m{0.05\textwidth} | m{0.1\textwidth} | m{0.1\textwidth} | m{0.08\textwidth} | m{0.15\textwidth} |}
      \cline{1-2}\cline{4-5}
      \multicolumn{2}{| c |}{P} & & \multicolumn{2}{ c |}{S} \\ 
      \cline{1-2}\cline{4-5}
      \slshape A=nr & \slshape B=imię & & \slshape B=imię & \slshape C=właściciel\\ 
      \cline{1-2}\cline{4-5}
      1 & Weles & & Doruś & Halina \\ 
      \cline{1-2}\cline{4-5} 
      2 & Kycia & & Dziunia & Grażyna \\ 
      \cline{1-2} \cline{4-5} 
      \multicolumn{3}{c|}{} & Stefan & Agata\\ 
        \cline{4-5}
    \end{tabular}
  \end{center}
  wtedy $\pi_B(P)\setminus \pi_B(S)=\pi_B(P)$.

  {$\color{blue}\max=p$}

  Natomiast najwięcej krotek jakie możemy dostać to $p$, jeśli wyrazy w kolumnie $B$ tablicy $S$ są takie same jak wyrazy z kolumny $B$ tablicy $P$ i w dodatku każda krotka z $P$ jest unikalna:
  \begin{center}
    \begin{tabular}{| m{0.05\textwidth} | m{0.1\textwidth} | m{0.1\textwidth} | m{0.08\textwidth} | m{0.15\textwidth} |}
      \cline{1-2}\cline{4-5}
      \multicolumn{2}{| c |}{P} & & \multicolumn{2}{ c |}{S} \\ 
      \cline{1-2}\cline{4-5}
      \slshape A=nr & \slshape B=imię & & \slshape B=imię & \slshape C=właściciel\\ 
      \cline{1-2}\cline{4-5}
      1 & Weles & & Weles & Halina \\ 
      \cline{1-2}\cline{4-5} 
      2 & Kycia & & Kycia & Grażyna \\ 
      \cline{1-2} \cline{4-5} 
    \end{tabular}
  \end{center}
  \bigskip

  \textbf{\color{green} d) $(S\bowtie S)\bowtie S$}

  {\color{blue}$\min=\max=s$}

  Tutaj zawsze będzie $s$ krotek, bo $S\bowtie S=S$.
  \bigskip 

  \textbf{\color{green} e) $\sigma_{A<B}(P)\cup \sigma{A>B}(P)$}

  {\color{blue}$\min=0$}

  Tutaj najmniejsza ilość krotek w wyniku jest wtedy, gdy $A=B$ zawsze w tabeli $P$, np.
  
  \begin{center}
    \begin{tabular}{| m{0.19\textwidth} | m{0.13\textwidth} |}
      \hline
      \multicolumn{2}{| c |}{P} \\
      \hline 
      \slshape A=nr w dzienniku & \slshape B=ocena \\ 
      \hline 
      1 & 1 \\
      \hline 
      2 & 2 \\
      \hline
      3 & 3\\ 
      \hline
    \end{tabular}
  \end{center}
  ponieważ $\sigma_{A>B}(P)$ oraz $\sigma_{A<B}(P)$ są zbiorami pustymi, to ich suma też jest pusta.

  {\color{blue}$\max=p$}

  Natomiast, jeśli nie ma krotek, gdzie elementy są takie same, to dostaniemy $p$ sztuk krotek:
  \begin{center}
    \begin{tabular}{| m{0.19\textwidth} | m{0.13\textwidth} |}
      \hline
      \multicolumn{2}{| c |}{P} \\
      \hline 
      \slshape A=nr w dzienniku & \slshape B=ocena \\ 
      \hline 
      1 & 2 \\
      \hline 
      2 & 3 \\
      \hline
      3 & 6\\ 
      \hline
    \end{tabular}
  \end{center}
\end{solution}

\begin{problem}
  Czy operator różnicy $\setminus$ da się wyrazić za pomocę algebry relacji z operatorami $\pi$, $\sigma$, $\rho$, $\times$, $\cup$? Przyjmijmy, że warunki $F$ są formułami zbudowanymi przy użyciu koniunkcji, alternatywy oraz zawierają wyłącznie atomy postaci $Atr_1=const$ lub $Atr_1=Atr_2$, gdzie $Atr_1$, $Atr_2$ są atrybutami, a $const$ stałą odpowiedniego typu. Czy odpowiedź na pytanie zmieni się, jeśli w warunkach dopuścimy negację? \emph{Wskazówka: poszukaj pewnej charakterystycznej cechy, którą mają wszystkie zapytania wyrażane za pomocą $\pi$, $\sigma$, $\rho$, $\times$, $\cup$, a której nie musi mieć zapytanie wyrażone z użyciem $\setminus$}.
\end{problem}

\begin{solution}
%  Ponieważ różnica ma sens tylko gdy atrybuty obu tablic są takie same, to niech $P(A, B)$ i $S(A, B)$ będą dowolnymi tablicami (piszę $A, B$ zamiast $Atr_i$).
%
%  Jeśli negacja jest dozwolona, to możemy zacząć od zmiany nazwy $S(A, B)$ na $S(A', B')$ i przemnożenia wyniku przez $P(A, B)$. Potem wystarczy wybrać te wyrażenia, w których obie kolumny są różne i rzutować to na kolumny odpowiadające $P(A, B)$:
%  $$P(A, B)\setminus S(A, B)=\pi_{A, B}{\color{orange}(} \sigma_{A\neq A'\land B\neq B'}{\color{green}[}P(A, B)\times \pi_{S(A',B')}{\color{red}(}S(A, B){\color{red})}{\color{green}]} {\color{orange})}$$
%  Dla przykładu
%  \begin{center}
%    \begin{tabular}{ | >{\centering\arraybackslash}p{0.07\textwidth} | >{\centering\arraybackslash}m{0.07\textwidth} | m{0.05\textwidth} | >{\centering\arraybackslash}p{0.07\textwidth} | >{\centering\arraybackslash}p{0.07\textwidth} | }
%      \cline{1-2} \cline{4-5}
%      \multicolumn{2}{|c|}{P} & & \multicolumn{2}{c|}{S}\\
%      \cline{1-2} \cline{4-5}
%      A & B & & A & B\\ 
%      \cline{1-2} \cline{4-5} 
%      1 & 2 & & 3 & 4\\ 
%      \cline{1-2} \cline{4-5}
%      3 & 4 & & 8 & 9\\ 
%      \cline{1-2} \cline{4-5} 
%    \end{tabular}
%
%    \begin{tabular}{ | >{\centering\arraybackslash}p{0.07\textwidth} | >{\centering\arraybackslash}m{0.07\textwidth} | m{0.05\textwidth} | >{\centering\arraybackslash}p{0.07\textwidth} | >{\centering\arraybackslash}p{0.07\textwidth} | }
%      \cline{1-2} \cline{4-5}
%      \multicolumn{2}{|c|}{P} & & \multicolumn{2}{c|}{$\pi_{S(A', B')}S$}\\
%      \cline{1-2} \cline{4-5}
%      A & B & & A' & B'\\ 
%      \cline{1-2} \cline{4-5} 
%      1 & 2 & & 3 & 4\\ 
%      \cline{1-2} \cline{4-5}
%      3 & 4 & & 8 & 9\\ 
%      \cline{1-2} \cline{4-5} 
%    \end{tabular}
%
%    \begin{tabular}{ | >{\centering\arraybackslash}p{0.07\textwidth} | >{\centering\arraybackslash}m{0.07\textwidth} | >{\centering\arraybackslash}p{0.07\textwidth} | >{\centering\arraybackslash}p{0.07\textwidth} | }
%      \hline 
%      \multicolumn{4}{|c|}{$P\times S(A', B')$}\\ 
%      \hline 
%      A & B & A' & B'\\ 
%      \hline 
%      1 & 2 & 3 & 4 \\ 
%      \hline 
%      1 & 2 & 8 & 9 \\ 
%      \hline 
%      3 & 4 & 3 & 4 \\
%      \hline 
%      3 & 4 & 8 & 9 \\ 
%      \hline
%    \end{tabular}
%
%    \begin{tabular}{ | >{\centering\arraybackslash}p{0.07\textwidth} | >{\centering\arraybackslash}m{0.07\textwidth} | >{\centering\arraybackslash}p{0.07\textwidth} | >{\centering\arraybackslash}p{0.07\textwidth} | }
%      \hline 
%      \multicolumn{4}{|c|}{$\sigma_{A= A'\land B = B'}P\times S(A', B')$}\\ 
%      \hline 
%      A & B & A' & B'\\ 
%      \hline 
%      3 & 4 & 3 & 4 \\ 
%      \hline
%    \end{tabular}
%  \end{center}

  Dla operatorów, które nie są $\setminus$, jeśli $A\subseteq B$, to $op\; A\subseteq op\; B$. Stąd, nie da się zapisać $\setminus$ przy użyciu pozostałych, nie ważne czy mamy negację czy nie mamy.
\end{solution}

\begin{problem}
  $X$, $Y$ i $Z$ są relacjami zawierającymi pojedynczą kolumnę o nazwie $A$. STudent ma napisać wyrażenie algebry relacji wyliczające wartość $X\cap (Y\cup Z)$ nie używając operatorów sumy i przekroju relacji. W bazie danych rozwiązań zadań z poprzednich edycji kursu znalazł następujące wyrażenie:
  $$\pi_A(\sigma_{A=A_Y\lor A=A_Z}(X\times \rho_{Y(A_Y)}Y\times\rho_{Z(A_Z)}Z))$$
  Czy powinien użyć tego rozwiązania? Jeśli zapytanie jest poprawne, to uzasadnij to, jeśli nie, to zastanów się czy i jak można je poprawić.
\end{problem}

\begin{solution}
  Zacznijmy od napisania, kiedy $x\in X\cap(Y\cup Z)$:
  $$x\in X\cap (Y\cup Z)\iff x\in X\land (x\in Y\lor x\in Z).$$
  W rozwiązaniu w bazie zaczynamy od krotek $(x, y, z)$, gdzie $x\in X$, $y\in Y$ i $z\in Z$. Potem wybieramy z nich te elementy, dla których $y\in X$ lub $z\in X$. Po rzutowaniu na pierwszą współrzędną dostajemy więc elementy $x\in X$, dla których $x\in Y$ lub $x\in Z$, czyli jest to rozwiązanie poprawne. 
\end{solution}

\begin{problem}
\end{problem}

\begin{solution}$ $

\begin{lstlisting}[mathescape=true]
pi movies.name, directors.last_name, movies_genres.genre (
  sigma (movies.id = movies_genres.movie_id and movies.year < 1960) (
	movies x 
	sigma (movies_genres.movie_id = movies_directors.movie_id) (
		movies_genres x
		(
			sigma (
				movies_directors.director_id = directors.id) (
					movies_directors x directors)
			)
		)
	)
)
  \end{lstlisting}

\begin{lstlisting}
pi actors.first_name, actors.last_name (
  sigma (directors.last_name != 'Tarantino' and directors.first_name != 'Quentin') (
    (sigma movies_directors.movie_id = roles.movie_id (
	    (sigma directors.id = movies_directors.director_id (
          directors x movies_directors
          )
        ) x 
	      (sigma roles.actor_id = actors.id (
          roles x actors
          )
        )
      )   
    )
  )
)
\end{lstlisting}

\begin{lstlisting}
pi actors.first_name, actors.last_name (sigma actors.id = actors.ii (
		actors x (
			(pi ii (rho ii <- id actors)) - 
			pi roles.actor_id (
				sigma roles.role != roles.r
				(
				sigma roles.actor_id = roles.act_id (
					roles x
					(rho act_id <- actor_id, mv_id <- movie_id, r <- role roles)
				)
				)
			)
		)
	)
)
\end{lstlisting}

\begin{lstlisting}
pi movies.name
  (
    sigma movies.id = movies_genres.mv_id ( movies x 
    ( sigma movies_genres.genre = 'Drama' and movies_genres.g = 'Sci-Fi' ( sigma 
      movies_genres.mv_id = movies_genres.movie_id ( movies_genres x 
        (rho mv_id <- movie_id, g <- genre movies_genres)
      )
      )
    )
  )
  )
\end{lstlisting}

\begin{lstlisting}
pi actors.last_name ( sigma roles.ac_id = actors.id (actors x 
(
sigma roles.ac_id = roles.actor_id and roles.role = roles.r and roles.mv_id != roles.movie_id (
	roles x
	(rho r <- role, ac_id <- actor_id, mv_id <- movie_id roles)
)
)
)
)
\end{lstlisting}

\begin{lstlisting}
pi directors.last_name (
sigma directors.id = movies_directors.director_id (
directors x 
(
pi movies_directors.director_id movies_directors - 
pi movies_directors.director_id (
	sigma movies_genres.genre = 'Horror' 
	(
		sigma movies_directors.movie_id = movies_genres.movie_id (movies_directors x movies_genres)
	)
)
)
) 
)
\end{lstlisting}

Tutaj krzyczy błąd, ale nie ma reżyserów, którzy nakręcili film bez kobiet i to chyba krzyczy na iloczyn kartezjański z pustą listą?

\begin{lstlisting}
pi directors.last_name (sigma directors.id = movies_directors.director_id ((pi movies_directors.director_id movies_directors) - 
(pi movies_directors.director_id 
(sigma actors.gender = 'F' and actors.id = roles.actor_id (
actors x 
(sigma movies_directors.movie_id = roles.movie_id (movies_directors x roles))
))
) x directors)
)
\end{lstlisting}

\end{solution}

\begin{problem}
\end{problem}

\begin{solution}
  a) tutaj problemem jest, że czasem aktor gra w więcej niż jednym filmie, jak to się dzieje z Michaelem

  Naprawa:
  
  \begin{lstlisting}
tau genders desc
gamma actors.first_name;
count(acotrs.gender) -> genders (pi first_name, id, gender (actors join actors.id=roles.actor_id roles))
  \end{lstlisting}

  b)

\begin{lstlisting}
gamma directors.last_name; count(roles.actor_id) -> nr 
  (pi directors.last_name, directors.id, roles.actor_id 
  (roles join roles.movie_id=movies_directors.movie_id 
    (directors join directors.id=movies_directors.director_id movies_directors)
  )
  )
\end{lstlisting}
\end{solution}

\end{document}
