\section{Wstęp}

\subsection{Kompleksy łańcuchowe}

Niech $R$ będzie dowolnym pierścieniem, natomiast $A, B, C$ będą $R$-modułami. Mając ciąg

\begin{center}\begin{tikzcd}
  A\arrow[r, "f"] & B\arrow[r, "g"] & C
\end{tikzcd}\end{center}

mówimy, że jest on \acc{dokładny}, jeśli $\ker(g)=\img(f)$. W szczególności implikuje to, że $g\circ f=gf:A\to C$ jest homomorfizmem zerowym.

\begin{definition}[Kompleks łańcuchowy]
  Rozważmy rodzinę $C=\{C_n\}_{n\in\Z}$ $R$-modułów wraz z mapami $d=d_n:C_n\to C_{n-1}$ takimi, że każde złożenie
  $$[d_{n-1}\circ d_n=]d\circ d:C_n\to C_{n-2}$$
  jest zerowe. Wówczas każdą mapę $d_n$ nazywamy \buff{różniczkami $C$}, a rodzina $C$ jest \buff{kompleksem łańcuchowym}.
\end{definition}

Jądra każdego $d_n$ nazywamy \acc{$n$-cyklami} $C$ i oznaczamy $Z_n=Z_n(C)$, natomiast obraz każdego $d_{n+1}$ jest nazywany \acc{$n$-granicą} $C$ i oznacza się jako $B_n=B_n(C)$. Ponieważ $d_n\circ d_{n+1}=0$, to
$$0\subseteq B_n\subseteq Z_n\subseteq C_n.$$

\begin{definition}[Homologia]
  \buff{$n$-tym modułem homologii} kompleksu $C$ nazywamy iloraz $\color{green}H_n(C)=Z_n/B_n$.
\end{definition}

\begin{problem}
  Ustalmy $C_n=\Z/8$ dla $n\geq0$ i $C_n=0$ dla $n<0$. Dla $n>0$ niech $d_n$ posyła $x\mod 8$ do $4x\mod 8$. Pokaż, że tak zdefiniowane $C$ jest kompleksem łańcuchowym $\Z/8$-modułów i policz moduły homologii.
\end{problem}

\begin{solution}
  Zauważyć, że $d_{n-1}d_n=0$ jest nietrudno dla $n\leq1$ ($d_{n-1}d_n:C_n\to C_{n-2}=0$). Z kolei dla dowolnego $n>1$ i dowolnego $x\in C_n$ wiemy, że $d_n(x)=4x\mod 8$. Jeśli $x$ było oryginalnie liczbą parzystą, to od razu widać, że $d_n(x)=0$. Z kolei gdy $x$ jest nieparzyste, to wówczas 
  $$d_{n-1}d_n(x)=d_{n-1}(4x\mod 8)=16x\mod 8=8\cdot(2x)\mod 8=0,$$
  a więc $d_{n-1}d_n=0$.

  Homologie dla $n<2$ są trywialne, natomiast dla $n\geq 2$ wszystkie są takie same (gdyż funkcje $d_n$ jak i moduły $C_n$ nie ulegają zmianie wraz z $n$). Wystarczy więc przyjrzeć się $Z_1/B_1$
  
  \begin{center}\begin{tikzcd}
    C_0=\Z/8 & C_1=\Z/8\arrow[l, "d_1"] & C_2=\Z/8\arrow[l, "d_2"]
  \end{tikzcd}\end{center}

  $Z_1$ to liczby parzyste w $\Z/8$ (kernel $d_1$), natomiast $B_1$ to liczby podzielne przez $4$, ale nie przez $8$ w $C_1$. W takim razie, $Z_1/B_1=\{4\}$.
\end{solution}
