\section{30.10.2023 : Kompleksy łańcuchowe i (ko)homologie}

\subsection{Kompleks łańcuchowy i sympleksy}

\begin{definition}[kompleks łańcuchowy]
  \buff{Kompleks (ko)łańcuchowy} w kategorii abelowej $\mathbf{A}$ to ciąg obiektów i morfizmów
  \begin{center}\begin{tikzcd}
    ...\arrow[r] & A^{n-1}\arrow[r, "d^{n-1}"] & A^{n} \arrow[r, "d^n"] & A^{n+1} \arrow[r] & ...
  \end{tikzcd}\end{center}
  taki, że dla każdego $n$ $d^n\circ d^{n-1}=0$
\end{definition}

\begin{example}[kompleksów łańcuchowych]
  \item Niech $X$ będzie \acc{kompleksem symplicjalnym}. Z takim sympleksem można teraz stowarzyszyć kompleks symplicjalny z obiektami
    $$C_nX=\bigoplus_{\sigma-\text{n-sympleks}}\Z$$
    i wtedy $\partial:C_nX\to C_{n-1}X$ jest odwzorowaniem brzegu między tymi obiektami takim, że
    $$\partial[\sigma^n]=\sum_{\tau^{n-1}<\sigma^n}\pm[\tau^{n-1}]$$
    gdzie $\sigma^n$ to generator składniku $\Z$ odpowiadający sympleksowi $\sigma^n$. Jeśli mamy sympleks $\sigma^n=(v_0,...,v_n)$ to przez ścianę $\tau^{n-1}$ rozumiemy
    $$\tau^{n-1}=(v_0,...,\hat{v_i},...,v_n)$$
    gdzie przez $\hat{v_i}$ rozumiemy opuszczenie tej współrzędnej.

  \item Niech $X$ będzie przestrzenią topologiczną, wówczas 
    $$S_nX=\bigoplus_{\sigma:\Delta^n\to X}\Z$$
    gdzie $\sigma:\Delta^n\to X$ jest ciągłym odwzorowaniem z sympleksu w $X$. To się nazywa \acc{kompleks singularny}.

    Odwzorowanie brzegu $\partial: S_nX\to S_{n-1}X$ na $\sigma:\Delta^n\to X$ przyjmuje wartość
    $$\partial\sigma=\sum_{i=0}^n(-1)^i(\sigma\restriction_\text{i-ta ściana})$$
    
  \item \buff{Kompleks de Rhama}

    Niech $M$ będzie gładką rozmaitością, $A^n=\Omega^nM$ będzie zbiorem gładkich form na niej. Wówczas $d:\Omega^nM\to\Omega^{n+1}M$ jest pochodną zewnętrzną.

    W szczególności, jeśli $M=T^2$, to $H^1=\R^2$, $H^2=\R$ oraz $H^{>2}=0$.
\end{example}

\subsection{Homologie}

Skoro $\partial_n\cdot\partial_{n+1}=0$, to $\img\partial_{n+1}\subseteq\ker\partial_n$, więc możemy zastanowić się nad
  $$H_nX=\ker\partial_n/\img\partial_{n+1}.$$
Tak zdefiniowane $H_nX$ nazywamy \buff{homologiami}.

\begin{definition}[ogólna definicja (ko)homologii]
  Niech $A^\cdot$ będzie kompleksem (ko)łańcuchowym i patrzymy na jego wycinek
  \begin{center}\begin{tikzcd}
    ...\arrow[r] & A^{n-1}\arrow[d, "a", purple] \arrow[r, "d^{n-1}"] & A^{n}\arrow[r, "d^n"] & A^{n+1} \arrow[r] & ...\\
                 & K\arrow[ur, "\ker d^n" below right]
  \end{tikzcd}\end{center}
  Ponieważ $d^n\circ d^{n-1}=0$, to pojawia się nam od razu odwzorowanie do jądra $a:A^{n-1}\to K$. Chcemy więc nazwać
  $$H^n(A^\cdot)=\coker a$$
  \buff{homologią}.
\end{definition}

  Ale to samo można zrobić dualnie, tzn.

  \begin{center}\begin{tikzcd}
    & & & C\arrow[d, "b", purple]\\
    ...\arrow[r] & A^{n-1} \arrow[r, "d^{n-1}"] & A^{n}\arrow[r, "d^n"]\arrow[ur, "\coker d^{n-1}"] & A^{n+1} \arrow[r] & ...
  \end{tikzcd}\end{center}
  i zdefiniować $H^n(A^\cdot)=\ker b$.

  %$R$-moduł 
  %$$H^n(A^\cdot)=\ker d^n/\img d^{n-1}$$
  %nazywamy \buff{$n$-tą homologią} kompleksu $A^\cdot$.

\begin{lemma}
  W definicji jak wyżej $H^n(A^\cdot):\coker a\cong\ker b$.
\end{lemma}

\begin{proof}
  \begin{itemize}
    \item Przy dodatkowym założeniu, że $d^{n-1}$ jest monomorfizmem, a $d^n$ jest epimorfizmem, dostajemy
  $$d^n=\coker\ker d^n=\coker k$$
  $$d^{n-1}=\ker c$$
  Pokażemy, że $a=\ker ck$ oraz $b=\coker ck$, z czego od razu wynika teza:
  \begin{center}\begin{tikzcd}
    A^{n-1}\arrow[r, "a"] & K\arrow[r, "ck"]\arrow[d] & C \arrow[r, "b"] & A^n\\
                          & \coker a\arrow[r, blue, "\star"] & \ker b\arrow[u]
  \end{tikzcd}\end{center}
  i strzałka $\color{blue}\star$ jest izomorfizmem na mocy lematu \ref{lemat 5.2}.

  {\large\color{red}POBAWIĆ SIĘ WYKRESEM za zdjęcia}

  \item Bez dodatkowych założeń

    {\large\color{red}ZDJĘCIA}
\end{itemize}

\end{proof}

\subsection{Pull-back i push-out}

Po polsku czasem mówi się na to kwadrat kartezjański i kwadrat kokartezjański.

\begin{definition}$ $\newline
  \begin{itemize}
    \item Pull-back to granica diagramu
      \begin{center}\begin{tikzcd}
        {\color{blue}X}\arrow[d, blue]\arrow[r, blue]& B\arrow[d]\\
        A\arrow[r] & C
      \end{tikzcd}\end{center}
    \item Push-out to z kolei kogranica diagramu
      \begin{center}\begin{tikzcd}
        C\arrow[d]\arrow[r]& B\arrow[d, blue]\\
        A\arrow[r, blue] & {\color{blue}Y}
      \end{tikzcd}\end{center}
  \end{itemize}
\end{definition}

\begin{fact}
  W abelowej kategorii istnieją pull-backi i push-outy.
\end{fact}

\begin{proof}
  Kandydatem na pull-back będzie jądro odwzorowania.
\end{proof}

