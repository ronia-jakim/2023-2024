\section{08.01.24 : }

\begin{center}\begin{tikzcd}
  K^+(I_\mathbf{A})\arrow[r]\arrow[dr, yshift=1mm, "\iota"] & K^+(\mathbf{A})\arrow[r, "K(F)"]\arrow[d] & K^+(\mathbf{B})\arrow[d]\\ 
                                                            & D^+(\mathbf{A})\arrow[ul, "\iota^{-1}", yshift=-1mm] & D^+(\mathbf{B})
\end{tikzcd}\end{center}

\begin{definition}
  Dla dowolnego $A\in\ob\mathbf{A}$ definiujemy
  $$R^iF(A):=H^i(RF(A[0]))$$
\end{definition}

\begin{fact}
  Jeśli $F$ jest lewo dokładny, i $0\to A\to B\to C\to 0$ jest dokładny w $\mathbf{A}$, to również
  \begin{center}\begin{tikzcd}
    0\arrow[r] & F(A)\arrow[r] & F(B)\arrow[r] & F(C)\arrow[r] R^1F(A)\arrow[r] R^1F(B)\arrow[r] & R^1F(C)\arrow[r] & R^2F(A)\arrow[r] & ...
    \end{tikzcd}\end{center}
    jest ciągiem dokładnym.
\end{fact}

\begin{fact}
  $$Ext_\mathbf{A}^i(A,-)=R^iHom(A, -)$$
\end{fact}

\begin{proof}
  Policzmy $(R^iHom(A, -))(B)$ dla dowolnego $B\in \mathbf{A}$. Mamy rezolwentę injektywną
  $$B\to I^0\to I^1\to I^2\to ...$$
  dostajemy wówczas
  \begin{center}\begin{tikzcd}
    ...\arrow[r] & 0\arrow[r] & Hom(A, I^0)\arrow[r] & Hom(A, I^1)\arrow[r] & Hom(A, I^2)
  \end{tikzcd}\end{center}
  Szukana grupa (obiekt) to 
  $$\ker d:\frac{Hom(A, I^i)\to Hom(A, I^{i+1})}{\img d:Hom(A, I^{i+1})\to Hom(A, I^i)}/$$
  Dowolna strzałka $Hom(A, I^i)\ni f:A\to I^i$ jest w $\ker d$ $\iff$ odwzorowanie
  \begin{center}\begin{tikzcd}
    ...\arrow[r] & 0\arrow[r]\arrow[d, "0"] & A\arrow[r]\arrow[d, "f"] & 0\arrow[r]\arrow[d, "0"] & 0\arrow[r]\arrow[d, "0"] & ...\\
    ...\arrow[r] & I^{i-1} \arrow[r] & I^i\arrow[r] & I^{i+1}\arrow[r] & I^{i+2}\arrow[r] & ...
  \end{tikzcd}\end{center}
  jest morfizmem kompleksów
  $$\ker d:\underbrace{Hom(A, I^i)}_{=Hom_{Kom\mathbf{A}}(A[i], I^*)}\to Hom(A, I^{i+1})$$

  Taka sama strzałka $f:A\to I^i$ jest w obrazie $\img d$ $\iff$ odwzorowanie kompleksów
  \begin{center}\begin{tikzcd}
    0\arrow[r]\arrow[d] & A\arrow[r]\arrow[d, "f"]\arrow[dl, green, "h"] & 0\arrow[d]\\ 
    I^{i-1}\arrow[r] & I^i\arrow[r] & I^{i+1}
  \end{tikzcd}\end{center}
  jest homotopijne z zerem.

  Zatem dostajemy
  $$H^i(Hom(A, I^*))=Hom_{K^+(\mathbf{A})}(A[i], I^*)$$
  ale ponieważ $I^*$ jest injektywny, to
  $$Hom_{K^+(\mathbf{A})}(A[i], I^*)=Hom_{D^+(\mathbf{A})}(A[i], I^*)$$
  ale przecież w  $D^+\mathbf{A}$ mamy $I^*\cong B[0]$, więc
  $$Hom_{D^+\mathbf{A}}(A[i], I^*)=Hom_{D^+\mathbf{A}}(A[i], B[0])=Ext^i_{\mathbf{A}}(A, B).$$
\end{proof}

\begin{fact}
  \begin{enumerate}
    \item $RF$ jest dokładny.
    \item $\exists$ naturalne przekształcenie $\epsilon_F:Q_B\circ K(F)\implies RF\circ Q_A$
    \item Jeśli $G:D^+\mathbf{A}\to D^+\mathbf{B}$ jest dokładny i dopuszcza naturalne przekształcenie $\epsilon_G:Q_B\circ K(F)\implies G\circ Q_A$ analogiczne jak wyżej, to istnieje naturalne przekształcenie $\eta:RF\to G$, które zamyka diagram
      \begin{center}\begin{tikzcd}
        & RF\circ Q_A\arrow[dd, "\eta"]\\
        Q_B\circ K(F)\arrow[ur, "\epsilon_F"]\arrow[dr, "\epsilon_G"]\\ 
        & G\circ Q_A
      \end{tikzcd}\end{center}
  \end{enumerate}
\end{fact}

\begin{proof}
  \begin{enumerate}
    \item Mając dany trójkąt wyróżniony \begin{tikzcd} A^*\arrow[r, "f"] & B^*\arrow[r] & C(f)\arrow[r] & A^*[1] \end{tikzcd} produkujemy stożek w sposób funktorialny
      \begin{center}\begin{tikzcd}
        F(A^*)\arrow[r, "F(f)"] & F(B^*)\arrow[r] & F(C(f))\arrow[r, "="] & C(F(f))
      \end{tikzcd}\end{center}

      Stąd $K(F)$ i $Q_B$ są dokładne. Ponieważ $RF=Q_B\circ K(F)\circ \iota^{-1}$, to wystarczy pokazać, że $\iota^{-1}$ jest dokładna.

      Bierzemy więc trójkąt wyróżniony w $D^+\mathbf{A}$, zaznaczony na zielono. Naszym celem będzie pokazać, że druga od góry linijka jest trójkątem wyróżninym
      \begin{center}\begin{tikzcd}
        \iota^{-1}A\arrow[r, "g"]\arrow[d, "="] & \iota^{-1} B\arrow[r]\arrow[d, "="] & C(g) \arrow[r] & \iota^{i-1}A[1]\arrow[d, "="] & \Delta\text{ w }K^+(I_\mathbf{A})\\

        \iota^{-1}A\arrow[r, "g"]\arrow[d, "\cong"] &\iota^{-1} B\arrow[r]\arrow[d, "\cong"] & \iota^{-1}C\arrow[r]\arrow[d, "\cong"] & \iota^{-1}A[1]\arrow[d, "\cong"] & \text{trójkąt w }K^+(I_\mathbf{A})\\
        \color{green}A\arrow[r, green] \arrow[d, "\cong"] &\color{green} B\arrow[r, green]\arrow[d, "\cong"] & \color{green} C\arrow[r, green] \arrow[d, "\cong"] & \color{green} A[1]\arrow[d, "\cong"] & \text{wyjściowy }\Delta\\ 
        A'\arrow[r]\arrow[uuu, bend left=20, orange] & B'\arrow[r]\arrow[uuu, bend left=20, orange] & C(f)\arrow[r]\arrow[uuu, dashed, orange, bend left=20, "\exists\phi" left] & A'[1]\arrow[uuu, bend right=20, orange] & \Delta\text{ w }Kom^+(\mathbf{A})
      \end{tikzcd}\end{center}
      pomarańczowa strzałka istnieje, ponieważ pomarańczowe strzałki to izomorfizmy między wyrazami trójkątów wyróżnionych -> musi się więc domknąć do pełnego odwzorowania między tymi trójkątami. Pokażemy teraż, że $\phi:C(f)\to C(g)$ jest qis. Rozważamy ciągi dokładne trójkątów:
      \begin{center}\begin{tikzcd}
        H^i(\iota^{-1}A)\arrow[r] & Hom^i(\iota^{-1}B)\arrow[r] & H^i(C(g))\arrow[r] & H^{i+1}(\iota^{-1}A)\arrow[r] & Hom^{i+1}(\iota^{-1}B)\\ 
        H^i(A')\arrow[u, "\cong"]\arrow[r] & H^i(B')\arrow[u, "\cong"]\arrow[r] & H^i(C(f))\arrow[u,dashed]\arrow[r] & H^{i+1}(A')\arrow[u, "\cong"]\arrow[r] & H^{i+1}(B')\arrow[u, "\cong"]
      \end{tikzcd}\end{center}
      z lematu o $5$ izomorfizmach wiemy, że strzałka z $H^i(C(f))$ jest izomorfizmem, czyli w rzeczy samej, $\phi$ jest qis.

      Mamy więc z tego odwzorowanie trójkąta z drugiej linii do trójkąta z 1 linii i to odwzorowanie jest izomorfizmem w $D^+(\mathbf{A})$. Ponieważ 1 i 2 linia składają się z obiektów injektywnych, to izomorfizm ten realizuje się w $K^+(\mathbf{A})$.

    \item Mamy diagram
      \begin{center}\begin{tikzcd}[column sep=large, row sep=large]
        K^+\mathbf{A}\arrow[r, "K(F)"]\arrow[d, "Q_A" left] & K^+\mathbf{B}\arrow[d, "Q_B"]\arrow[dl, "\varepsilon_F"]\\ 
        D^+\mathbf{A}\arrow[r, "RF" below] & D^+\mathbf{B}
      \end{tikzcd}\end{center}
      i drugi
      \begin{center}\begin{tikzcd}
        & A^*\arrow[r] \arrow[d] & F(A^*)\arrow[d]\\ 
        & A^*\arrow[dl, "\mu(A^*)" above left]\arrow[dr] & \arrow[dd, bend left=40, "F(\mu(A^*))"] F(A^*)\\ 
        I^* & & RF(A^*)\arrow[d, "\cong"]\\ 
            & & F(I^*)
      \end{tikzcd}\end{center}
    \item 
      \begin{center}\begin{tikzcd}
        K^+\mathbf{A}\arrow[d] \arrow[r, "K(F)"] & K^+\mathbf{B} \arrow[d] \\ 
        D^+\mathbf{A} \arrow[r, bend left=20, "RF" above]\arrow[r, bend right=20, "G" below] & D^+\mathbf{B}
      \end{tikzcd}\end{center}
      {\large\color{red}ZDJĘCIE}
  \end{enumerate}
\end{proof}

Jeśli $A, B$ są $R$-modułami, to $Tor(A, B)$ liczymy biorąc rezolwentę projektywną
$$...\to P_1\to P_0\to B$$ 
tensorujemy ją z $A$ i dostajemy
$$...\to A\otimes P_2\to A\otimes P_1\to A\otimes P_0\to 0$$
Wtedy kohomologie tego nowego kompleksu to tory: $H^i=Tor_i(A, B)$
  
\begin{theorem}
  Niech $\mathbf{A},\mathbf{B}$ będą kategoriami abelowymi takimi, że w $\mathbf{A}$ jest dostatecznie dużo oniektów injektywnych $I_\mathbf{A}$. Niech 
  $$F:\mathbf{A}\to \mathbf{B}$$
  będzie addytywnym i lewo-dokładnym funktorem. Niech $I_F$ będzie klasą obiektów $F$-acyklicznych. To znaczy takich $A\in\ob\mathbf{A}$, dla których $R^iF(A)=0$ dla $i>0$. Wtenczas
  \begin{enumerate} 
    \item $I_A\subseteq I_F$
    \item jeśli $A^*\in Kom^+\mathbf{A}$ jest dokładny oraz $A^i\in I_F$, to wówczas $F(A^*)$ też jest dokładny
    \item jeśli $A\to A^*$ jest rezolwentą $F$-acykliczną, to 
      $$RF(A[0])=RF(A^*)=F(A^*)$$
      a co więcej
      $$R^iF(A)=H^i(F(A^*)).$$
  \end{enumerate}
\end{theorem}

\begin{proof}
  \begin{enumerate}
    \item Niech $I\in I_A$ i chcemy sprawdzić, że wyższe pochodne $R^iF(I)=0$. Rozważamy rezolwentę injektywną $I$:
      \begin{center}\begin{tikzcd}
        I\arrow[r] \arrow[d, "F"] & I\arrow[r] & 0\arrow[r] & 0\arrow[r] & ... \\ 
        0\arrow[r] & F(I)\arrow[r] & 0\arrow[r] & 0 \arrow[r] & ...
      \end{tikzcd}\end{center}
      czyli $H^*$ to $0\quad F(I)=R^0F(I) \quad 0\quad 0=R^2F(I)\quad ...$.
    \item  Bez straty ogólności rozważmy
      $$0\to A^0\to A^1\to A^2\to....$$
      Niech $B^i=\img(d:A^{i-1}\to A^i)$. Z dokładności $A^*$ wynika dokładność ciągu
      $$0\to A^0\to B^1\to 0.$$
      W następnym korku możemy popatrzeć na włożenie $B^1\to A^2$, którego iloraz przez jądro jest dokładnie równe $B^2$. Kontynuując ten tok myślenia dostajemy cały zestaw ciągów dokłądnych
      $$0\to B^1\to A^1\to B^2\to 0$$
      $$0\to B^2\to A^2\to B^3\to 0$$
      $$...$$
      Czyli $B^1\cong A^0$ jest $F$-acykliczny. Z zestawu ciągów dokładnych dostajemy ciąg indukowany przez $F$
      \begin{center}\begin{tikzcd}
        0\arrow[r] & FB^1\arrow[r] & FA^1\arrow[r] & FB^2\arrow[r] & R^1F(B^1)\arrow[r] & R^1F(A^1)\arrow[r] & R^1F(B^2)\arrow[r] & R^2F(B^1)\arrow[r] & ...\\ 
                   & & & & 0\arrow[u, phantom, sloped, "="] & 0\arrow[u,phantom,sloped,"="] & 0 & 0
      \end{tikzcd}\end{center}

      \begin{center}\begin{tikzcd}
        & & 0 & & 0\\ 
        & & & B^k \\
        & & A^{k-1} & & A^k
      \end{tikzcd}\end{center}
      {\color{red}ZDJĘCIE}
    \item Użyjemy do tego punktu rezolwenty C-E z poprzedniego wykładu.
      Odwzorowanie $A[0]\to A^*$ jest qis, czyli jest izomorfizmem w $D^+\mathbf{A}$. Czyli $RF(A[0])\cong RF(A^*)$, w szczególności $R^iF(A)\cong H^i(RF(A^*))$.
      {\large\color{red}ZDJĘCIE BO NIE PRZEJDZIE RYSOWANIE TEGO}
  \end{enumerate}
\end{proof}

\begin{theorem}[o składaniu funktorów]
  Załóżmy, że są trzy kategorie abelowe $\mathbf{A},\mathbf{B},\mathbf{C}$ mające dostatecznie dużo obiektów injektywnych i dwa funktory $F:\mathbf{A}\to \mathbf{B}$ i $G:\mathbf{B}\to \mathbf{C}$, które są addytywne i lewo-dokładne. Załóżmy ponadto, że $G(I_\mathbf{A})\subseteq I_F$. Wtedy
  $$R(F\circ G)\cong RF\circ RG$$
  funktor pochodny złożenia jest izomorficzny ze złożeniem funktorów pochodnych.
\end{theorem}

\begin{proof}
  Z uniwersalności dostajemy $\varepsilon:R(F\circ G)\to RF\circ RG$.
  \begin{center}\begin{tikzcd}
    K^+\mathbf{C}\arrow[d]\arrow[dr, "\varepsilon_F"] & K^+\mathbf{B}\arrow[l]\arrow[d]\arrow[dr, "\varepsilon_G"] & K^+\mathbf{A}\arrow[l]\arrow[d] \\ 
    D^+\mathbf{C} & \arrow[l, "RF"] D^+\mathbf{B} & \arrow[l, "RG"] D^+\mathbf{A}
  \end{tikzcd}\end{center}
  {\large\color{red}ZDJĘCIE, BO COŚ SIĘ POPSUŁO}
\end{proof}

