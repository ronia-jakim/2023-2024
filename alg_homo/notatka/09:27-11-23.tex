\section{27.11.23 : Kategoria lokalna}

\begin{definition}[klasa lokalizująca]
  Klasa $S\subseteq Mor(\mathbf{B})$ jest nazywana \buff{lokalizującą}, jeśli spełnia następujące warunki:
  \begin{enumerate}
    \item identyczność dla dowolnego obiektu należy do $S$ i dla dowolnych $s,t\in S$ które można składać $s\circ t\in S$
    \item poniższe diagramy dopełniają się
      \begin{center}\begin{tikzcd}
        \bullet \arrow[r, dashed, green, "\in S"] \arrow[d, dashed] & \bullet\arrow[d] & & \bullet\arrow[r, green, "\in S"] \arrow[d] & \bullet\arrow[d, dashed]\\ 
        \bullet \arrow[r, green, "\in S" below] & \bullet & & \bullet\arrow[r, green, dashed, "\in S" below] & \bullet
      \end{tikzcd}\end{center}
      to znaczy, mając strzałki nieprzerywane znajdziemy strzałki przerywane tak, że odpowiednia strzałka jest z $S$ i powstałe diagramy komutują.
    \item dla wszystkich \begin{tikzcd}X\arrow[r, yshift=1mm, "f"]\arrow[r, yshift=-1mm, "g" below] & Y\end{tikzcd} $(\exists\;s\in S)\;(sf=sg)\iff ((\exists\;t\in S)\;ft=gt)$.
  \end{enumerate}
\end{definition}

\begin{definition}[domki]
  Niech $S$ będzie klasą lokalizującą w kategorii $\mathbf{B}$. Definiujemy nową kategorię $\color{green}\mathbf{B}[S^{-1}]$:
  \begin{itemize}
    \item $\ob \mathbf{B}[S^{-1}]=\ob \mathbf{B}$
    \item morfizmy $A\to A'$ to klasy równoważności "domków", czyli diagramów
      \begin{center}\begin{tikzcd}
        & B\arrow[dl, "s\in S", orange] \arrow[dr, "f"]\\ 
        A & & A' 
      \end{tikzcd}\end{center}
      które można rozumieć jako "ułamek" $fs^{-1}$.

      Mówimy, że dwa domki $B$ i $B'$ przedstawiające morfizm $A\to A'$ w $\mathbf{B}[S^{-1}]$ są równoważne, jeśli istnieje $C$ wraz z morfizmami (w $\mathbf{B}$) $C\to B$ i $C\to B'$ takimi, że poniższy domek komutuje
      \begin{center}\begin{tikzcd}
        & & C\arrow[dl, "r"]\arrow[dr] \arrow[ddll, orange, "sr\in S" above left, bend right=30]\\ 
        & B\arrow[dl, "s\in S" above left, orange]\arrow[drrr] & & B'\arrow[dr]\arrow[dlll, orange, "s'\in S" above left]\\ 
        A & & & & A'
      \end{tikzcd}\end{center}
  \end{itemize}
\end{definition}

Jeśli mamy domki $A\leftarrow B\to A'$ i $A' \leftarrow B'\to A''$, to ich złożeniem jest nowy domek, którego istnienie wynika z 2 warunku klasy lokalizujące:
\begin{center}\begin{tikzcd}
        & & C\arrow[dl, "s''\in S" above left, orange]\arrow[dr] \\ 
        & B\arrow[dl, "s\in S" above left, orange]\arrow[dr] & & B'\arrow[dr]\arrow[dl, orange, "s'\in S" above left]\\ 
        A & & A' & & A''
\end{tikzcd}\end{center}

\begin{theorem}[kategoria domków jest ciekawa]
  \begin{enumerate}
    \item $\mathbf{B}[S^{-1}]$ jest dobrze określona:
      \begin{enumerate}
        \item relacja na domkach jest relacją równoważności 
        \item składanie domków nie zależy od wyboru reprezentantów
        \item składanie domków jest łączne
      \end{enumerate}
    \item Funktor $\mathbf{B}\to \mathbf{B}[S^{-1}]$ ma własność uniwersalną, tzn. jeśli mamy funktor $\mathbf{B}\to \mathbf{D}$ taki, że elementy klasy lokalizującej $S$ są przez niego posyłane na izomorfizmy w $\mathbf{D}$, to wówczas taki funktor faktoryzuje się przez $\mathbf{B}[S^{-1}]$:
      \begin{center}\begin{tikzcd}[column sep=large, row sep=small]
        \mathbf{B}\arrow[r]\arrow[ddr] & \mathbf{B}[S^{-1}]\arrow[dd, dashed]\\ 
        S\arrow[u, sloped, phantom, "\subseteq"]\arrow[ddr]\\
                                                                                               & \mathbf{D}\\ 
                                                                                               & \text{izo.}\arrow[u, sloped, phantom, "\subseteq"]
      \end{tikzcd}\end{center}
  \end{enumerate}
\end{theorem}

\begin{proof}
  \begin{enumerate}
    \item 
      \begin{enumerate}
        \item Jedyną ciekawą częścią jest sprawdzenie przechodności relacji równoważności domków. To znaczy, mając dane domki $A\leftarrow B\to A'\sim A\leftarrow B'\to A'$ oraz $A\leftarrow B'\to A'\sim A\leftarrow B''\to A'$ chcemy znaleźć dowód na równoważność domków $B$ i $B''$:
          \begin{center}\begin{tikzcd}
            & & D\arrow[dr, dashed]\arrow[dl, dashed]\\
            & C \arrow[dl]\arrow[dr] & & C''\arrow[dl]\arrow[dr]\\
            B\arrow[d, orange]\arrow[drrrr] & & B'\arrow[dll, orange]\arrow[drr] & & B''\arrow[d]\arrow[dllll, orange]\\
            A & & & & A'
          \end{tikzcd}\end{center}
          Obiekt $D$ wraz ze strzałkami $D\to C$ oraz $D\to C''$ dostajemy z warunku 2. klasy lokalizującej:
          \begin{center}\begin{tikzcd}[column sep=tiny, row sep=tiny]
              &   & D\arrow[ddll, orange]\arrow[ddrr] \\
              & {\color{white}d} \\ 
            C\arrow[dr]\arrow[ddrr, orange, bend right=30] &   &   &    & C'\arrow[dl]\arrow[ddll, orange, bend left=30] \\ 
                                                           & B \arrow[dr, orange] &   & B'\arrow[dl, orange]     \\ 
              &   & A
          \end{tikzcd}\end{center}
      \end{enumerate}
  \end{enumerate}
\end{proof}
