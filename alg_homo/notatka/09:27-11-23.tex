\section{27.11.23 : Klasa lokalizująca}

\begin{definition}[klasa lokalizująca]
  Klasa $S\subseteq Mor(\mathbf{B})$ jest nazywana \buff{lokalizującą}, jeśli spełnia następujące warunki:
  \begin{enumerate}
    \item identyczność dla dowolnego obiektu należy do $S$ i dla dowolnych $s,t\in S$ które można składać $s\circ t\in S$
    \item poniższe diagramy dopełniają się
      \begin{center}\begin{tikzcd}
        \bullet \arrow[r, dashed, green, "\in S"] \arrow[d, dashed] & \bullet\arrow[d] & & \bullet\arrow[r, green, "\in S"] \arrow[d] & \bullet\arrow[d, dashed]\\ 
        \bullet \arrow[r, green, "\in S" below] & \bullet & & \bullet\arrow[r, green, dashed, "\in S" below] & \bullet
      \end{tikzcd}\end{center}
      To znaczy, mając strzałki nieprzerywane znajdziemy strzałki przerywane tak, że odpowiednia strzałka jest z $S$ i powstałe diagramy komutują.
    \item dla wszystkich \begin{tikzcd}X\arrow[r, yshift=1mm, "f"]\arrow[r, yshift=-1mm, "g" below] & Y\end{tikzcd} $(\exists\;s\in S)\;(sf=sg)\iff ((\exists\;t\in S)\;ft=gt)$.
  \end{enumerate}
\end{definition}
