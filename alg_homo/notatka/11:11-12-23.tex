\section{11.12.23 : Funktor Ext}
 
\begin{definition}[$Ext^i_{\mathbf{A}}(X, Y)$]
  Niech $\mathbf{A}$ będzie kategorią abelową. Rozważamy kategorie $Kom(\mathbf{A}), K(\mathbf{A}), D(\mathbf{A})$, które wszystkie mają te same obiekty.

  Niech $X\in\mathbf{A}$ będzie obiektem. Wtedy $\color{blue}X[-i]$ jest kompleksem, w który na $i$-tym miejscu stoi $X$:
  \begin{center}\begin{tikzcd}[row sep=tiny]
    ...\arrow[r] & 0\arrow[r] & X\arrow[r] & 0\arrow[r] & ... \\ 
                 & & i
  \end{tikzcd}\end{center}

  Dla $X,Y\in\ob\mathbf{A}$ definiujemy
  $$\color{blue}Ext_{\mathbf{A}}^i(X, Y)=Hom_{D(\mathbf{A})}(X[0], Y[i])$$
\end{definition}

Powyższe kategorie niekoniecznie są abelowe, ale na pewno są addytywne, więc na $Hom$ jak wyżej mamy zdefiniowaną operację dodawania.

Zauważmy, że
$$Hom_{D(\mathbf{A})}(X[0], Y[i])\cong Hom_{D(\mathbf{A})}(X[l], Y[i+l]).$$
możemy więc na $Ext$ zdefiniować mnożenie:
\begin{definition}[mnożenie $Ext$]
  $$Ext_{\mathbf{A}}^i(X, Y)\times Ext_{\mathbf{A}}^j(Y, Z)\xrightarrow{} Ext_{\mathbf{A}}^{i+j}(X, A)$$
  takie, że
  $$(X[0]\to Y[i])(Y[i]\to Z[j+i])\mapsto(X[0], Z[i+j])$$
\end{definition}

\subsection{Konstrukcja Yonedy}

Niech będzie dany ciąg dokładny
\begin{center}\begin{tikzcd}
  ...\arrow[r] & 0\arrow[r] & Y^{-i}\arrow[r] & K^{-i+1}\arrow[r] & ...\arrow[r] & K^0\arrow[r] & X\arrow[r] & 0\arrow[r] & ...
\end{tikzcd}\end{center}

Możemy wziąć fragment tego ciągu aż do $X$ i dopełnić go $0$. Dostajemy ciąg który nie jest już dokładny, więc dokładamy mu $X$ na dole, czyli tak naprawdę tworzymy odwzorowanie w kompleks $X[0]$ i drugie w kompleks $Y[i]$:
\begin{center}\begin{tikzcd}
  0\arrow[r] & Y\arrow[r]\arrow[d, "id"] & K^{-i+1}\arrow[r] & ... \arrow[r] & K^{0}\arrow[r]\arrow[d, "qis"] & 0\\ 
             & Y[i] & & & X[0]
\end{tikzcd}\end{center}
Tworzy się więc domek:
\begin{center}\begin{tikzcd}
  & K^*\arrow[dl, "qis" above left] \arrow[dr] \\ 
  X[0] & & Y[i] 
\end{tikzcd}\end{center}
Który nazywamy $Y(K^*)$ i jest on wtedy elementem $Ext_{\mathbf{A}}^i(X, Y)$.

\begin{example}
  \item $Ext^1$
    \begin{center}\begin{tikzcd}
      0\arrow[r] & Y \arrow[r] & K^0\arrow[r] & X\arrow[r] & 0 
    \end{tikzcd}\end{center}
  \item W kategorii przestrzeni wektorowych $Vect_K$ wszystkie $Ext=0$.
\end{example}

\begin{theorem}
  \begin{enumerate}[label=(\alph*)]
    \item Dla $i>0$ każdy element grupy $Ext_{\mathbf{A}}^i(X, Y)$ jest postaci $Y(K^*)$. To znaczy, że domki konstruowane wyżej są przydatne.
    \item $Ext^0(X, Y)=Hom_{\mathbf{A}}(X Y)$ (było tydzień temu)
    \item Dla $i<0$ $Ext^i(X, Y)=0$.
  \end{enumerate}
\end{theorem}

\begin{proof}
  \begin{enumerate}[label=(\alph*)]
    \item Zaczynamy od kompleksu $K^*$ i cały manewr będzie polegał na usunięciu wszystkiego, co nie jest między $-i$ a $0$. Rysujemy diagram
      \begin{center}\begin{tikzcd}
        & K^*\arrow[dl]\arrow[dr] \\ 
        X[0] & & Y[i]
      \end{tikzcd}\end{center}
    \item b
    \item Zaczynamy od diagramu dla $A^*\in Hom_{D(\mathbf{A})}(X[0], Y[i])$
      \begin{center}\begin{tikzcd}
        & A^*\arrow[dl, "s" above left]\arrow[dr, "f" above right] \\ 
        X[0] & & Y[i] \\ 
             & B^*\arrow[ul, "s'" above left, orange]\arrow[ur, "0", orange] \arrow[uu, "t", orange]
      \end{tikzcd}\end{center}
      pokażemy, że te pomarańczowe strzałki da się dorysować, by cały diagram komutował.

      Niech $B^*$ będzie kompleksem
      \begin{center}\begin{tikzcd}
        B^*=...\arrow[r] & A^{-i-2}\arrow[r] & \ker d^{-i-1}\arrow[r] & 0\arrow[r] & ... 
      \end{tikzcd}\end{center}
      czyli nad $Y$ ma zero, a nad $X$ ma bardzo podobny wygląd - na pewno homologie w $X$ się nie różnią od homologii w $X$ w kompleksie $A^*$.

      Nowy, zerowy, domek dominuje stary domek, więc ten stary też musiał być zerowy.
  \end{enumerate}

  {\large\color{red}CAŁY TEN DOWÓD TO POWINNAM NA ZDJĘCIA POPATRZEĆ JESZCZE RAZ}
\end{proof}

\subsection{Trójkąty wyróżnione}

\begin{definition}[wymiar homologiczny kategorii]
  Dla kategorii abelowej $\mathbf{A}$ definiujemy jej \buff{wymiar homologiczny} jako maksymalne $d$ takie, że 
  $$Ext_{\mathbf{A}}^d(X, y)\neq0 $$
  dla pewnych $X,Y\in\ob\mathbf{A}$. Oznaczamy $\color{blue}dh(\mathbf{A})=d$.
\end{definition}

\begin{example}
\item $dh(Vet_K)=0$
\item $dh(\mathbf{Ab})=1$
\item Dla pierścienia $R$ wymiar homologiczny (globalny) to $dh(R-mod)$. Na przykład pierścień wielomianów nad ciałem $\mathfrak{K}$, czyli $\mathfrak{K}[x_1,...,x_n]$, to ilość jego zmiennych $dh(\mathfrak{K}[x_1,..., x_n])=n$.
\end{example}

\begin{definition}[trójkąt wyróżniony]
  \buff{Trójkąt wyróżniony} [czasem nazywany dokladnym] w $K(\mathbf{A})$ lub w $D(\mathbf{A})$ to trójkąt izomorficzny z trójkątem postaci
  \begin{center}\begin{tikzcd}
    A^*\arrow[r, "f"] & B^*\arrow[r] & C(f)\arrow[r] & A^*[1]
  \end{tikzcd}\end{center}
\end{definition}

\begin{fact}\label{fakt 11.2}$ $\newline
  \begin{enumerate}[label=(\alph*)]
    \item $ $\newline
      \begin{center}\begin{tikzcd}
        X\arrow[r, "id"]\arrow[d] & X\arrow[r]\arrow[d] & 0\arrow[r]\arrow[d] & X[1] \\ 
        X\arrow[r, "id"] & X\arrow[r] & C(id_X)\arrow[r] & X[1] 
      \end{tikzcd}\end{center}
      i tutaj stożek $C(id_X)\cong 0$, bo $h(a^{i+1}, a^i)=(a^i, 0)$.
    \item Jeśli mamy trójkąt wyróżniony
      \begin{center}\begin{tikzcd}
        X\arrow[r, "u"] & Y \arrow[r] & Z\arrow[r] & X[1]
      \end{tikzcd}\end{center}
      to również 
      \begin{center}\begin{tikzcd}
        Y\arrow[r] & Z\arrow[r] & X[1]\arrow[r, "-u( 1 )"] & Y[1]
      \end{tikzcd}\end{center}
      też jest trójkątem wyróżnionym.
    \item Dla trójkątów wyróżnionych można uzupełniać diagramy, tzn. jeśli wiersze są trójkątami wyróżnionymi, to istnieje strzałka domykająca go:
      \begin{center}\begin{tikzcd}
        X\arrow[r]\arrow[d, "f"] & Y\arrow[r]\arrow[d, "g"] & Z\arrow[r]\arrow[d, dashed, blue, "\exists"] & X[1]\arrow[d, "f(1)"]\\ 
        X'\arrow[r] & Y'\arrow[r] & Z'\arrow[r] & X'[1]
      \end{tikzcd}\end{center}
  \end{enumerate}
\end{fact}

Te wszystkie własności pojawiają się w definicji kategorii striangulowanej. Poza nimi jest jeszcze jeden aksjomat, który jest bardzo skomplikowany, ale nie korzysta się z niego prawie nigdy.

\begin{lemma}
  W trójkącie wyróżnionym złożenie dwóch kolejnych odwzorowań jest zerowe jako morfizm w kategoriach $K(\mathbf{A})$ lub $D(\mathbf{A})$.
\end{lemma}

To znaczy, że samo w sobie niekoniecznie jest zerowe, ale jest homotopijne z odwzorowaniem zerowym.

\begin{proof}
  \begin{center}\begin{tikzcd}
    X\arrow[r, "id"] \arrow[d, "id"] & X\arrow[r]\arrow[d, "f"] & 0\arrow[r]\arrow[d, dashed, blue, "\exists"] & X[1]\arrow[d, "id"]\\ 
    X\arrow[r, "f"]\arrow[rr, orange, "0?" below, yshift=-2mm, bend right=10] & Y\arrow[r] & Z\arrow[r] & X[1]
  \end{tikzcd}\end{center}

  {\large\color{red}DOKOŃCZYĆ DIAGRAM}
\end{proof}

\begin{lemma}
  Niech $X\to Y\to Z\to X[1]$ będzie trójkątem wyróżnionym. Wtedy następujące ciąg są dokładne:
  \begin{center}\begin{tikzcd} 
    ...\arrow[r] & Hom(U, X[i]) \arrow[r] & Hom(U, Y[i]) \arrow[d, phantom, ""{coordinate, name=Z}] \arrow[r] & Hom(U, Z[i])\arrow[dll, rounded corners,
    to path={ -- ([xshift=2ex]\tikztostart.east)
|- (Z) [near end]\tikztonodes
-| ([xshift=-2ex]\tikztotarget.west)
-- (\tikztotarget)}
    ] \\ 
                 & Hom(U, X[i+1]) \arrow[r] & ...
  \end{tikzcd}\end{center}
  
  oraz

  \begin{center}\begin{tikzcd}
    ...\arrow[r] & Hom(Z[i], U)\arrow[r] & Hom(Y[i], U)\arrow[r] & Hom(X[i], U)
    \arrow[dll, rounded corners, 
to path={ -- ([xshift=2ex]\tikztostart.east)
|- (Z) [near end]\tikztonodes
-| ([xshift=-2ex]\tikztotarget.west)
-- (\tikztotarget)}
    ]\\ 
                 & Hom(Z[i-1], U)\arrow[r] & ...
  \end{tikzcd}\end{center}
\end{lemma}

Co jest ciekawe, to takie ciągi są dokładne niezależnie od tego, czy patrzymy w kategorii $K(\mathbf{A})$ czy $D(\mathbf{A})$.

\begin{proof}
  Ze względu na niezmienniczość tych trójkątów na przesunięcia (\ref{fakt 11.2} (b)), to wystarczy patrzeć na dokładność w jednym miejscu $X\xrightarrow{\alpha} Y\xrightarrow{\beta} Z$.

  Patrzymy więc na ciąg
  \begin{center}\begin{tikzcd}
    Hom(Z, U)\arrow[r, "\alpha^*"] & {\color{orange}Hom(Y, U)}\arrow[r, "\beta^*"] & Hom(X, U) 
  \end{tikzcd}\end{center}
  Zaczniemy od pokazania, że $\beta^*\circ\alpha^*=0$, ale to jest wprost z faktu, że 
  $$\alpha^*\beta^*=(\alpha\beta)^*=(0)^*=0$$
  ktore można narysować
  \begin{center}\begin{tikzcd}[column sep=large, row sep=large]
    X\arrow[r, "\alpha"]\arrow[dr, "\alpha^*\beta^*f" below left] & Y\arrow[r, "\beta"]\arrow[d, "\beta^*f"] & Z\arrow[dl, "f" below right]\\ 
                                                       & W
  \end{tikzcd}\end{center}
  Dalej chcemy sprawdzić, czy $\ker\alpha^*=\img\beta^*$? Najpierw zauważmy, że jeśli $f\in\ker\alpha^*$, to wówczas
  \begin{center}\begin{tikzcd}
    X\arrow[r, "\alpha"]\arrow[dr, "0" below left] & Y\arrow[d, "f"] \\ 
                                                   & U
  \end{tikzcd}\end{center}
  komutuje. Czyli rysując już docelowy diagram, mamy:
  \begin{center}\begin{tikzcd}
    X\arrow[r, "\alpha"]\arrow[d, "0", orange] & Y\arrow[r, "\beta"]\arrow[d, "f", orange] & Z\arrow[r]\arrow[d, dashed, "\exists", blue] & X[1]\arrow[d, "0(1)", orange] \\ 
    0\arrow[r] & U\arrow[r, "id"] & U\arrow[r] & 0
  \end{tikzcd}\end{center}
  Niebieskie odwzorowanie zamyka diagram, stąd $id$ między $U$. Czyli 
  $$g\beta=f\implies \beta^* g=f$$
\end{proof}

\begin{conclusion}
  Niech 
  \begin{center}\begin{tikzcd}
    0\arrow[r] & A\arrow[r] & B\arrow[r] & C\arrow[r] & 0
  \end{tikzcd}\end{center}
  będzie krótkim ciągiem dokładnym w $\mathbf{A}$. Prowadzi on do trójkąta wyróżnionego $A\to B\to C\to A[1]$ w kategorii $D(\mathbf{A})$ (ćwiczenia).

  Dzięki temu można pisać długie ciągi dokładne dla $U\in\ob\mathbf{A}$:
  \begin{center}\begin{tikzcd}
    ...\arrow[r] & Ext^i(C, U)\arrow[r] & Ext^i(B, U)\arrow[r] & Ext^i(A, U)\arrow[r] & Ext^{i+1}(C, U)\arrow[r] &... 
  \end{tikzcd}\end{center}

  i tak samo dla drugiej funktorialności:
  \begin{center}\begin{tikzcd}
    ...\arrow[r] & Ext^i(U,A)\arrow[r] & Ext^i(U, B)\arrow[r] & Ext^i(U, C)\arrow[r] & Ext^{i+1}(U, A)\arrow[r] &... 
  \end{tikzcd}\end{center}
\end{conclusion}

\begin{fact}
  Jeśli $dh(\mathbf{A})=0$, to każdy krótki ciąg dokładny W $\mathbf{A}$
  \begin{center}\begin{tikzcd}
    0\arrow[r] & A\arrow[r] & B\arrow[r] & C\arrow[r] & 0
  \end{tikzcd}\end{center}
  rozszczepia się, tzn. istnieje
  \begin{center}\begin{tikzcd}
    0\arrow[r] & A\arrow[r, "\alpha"] & B\arrow[r, "\beta", yshift=1mm] & C\arrow[r]\arrow[l, yshift=-1mm, "\exists \gamma"] & 0
  \end{tikzcd}\end{center}
  takie, że $\beta\gamma=Id_C$. Stąd
  \begin{center}\begin{tikzcd}
    0\arrow[r] & A\arrow[r]\arrow[dr] & A\oplus C\arrow[r] & C\arrow[r] & 0\\ 
               &            & B\arrow[ur]\arrow[u, phantom, sloped, "\cong"]
  \end{tikzcd}\end{center}
\end{fact}

\begin{proof}
  Niech \begin{tikzcd}0\arrow[r] & A\arrow[r] & B\arrow[r] & C\arrow[r] & 0\end{tikzcd} będzie ciągiem dokładnym. Wtedy dla $U=A$ mamy
  \begin{center}\begin{tikzcd}
    0\arrow[r] & Ext^0(C, A)\arrow[r] & Ext^0(B, A)\arrow[r] & Ext^0(A, A) \arrow[r] & Ext^1(C, A) \arrow[r] &...
\\ 
               & Hom(C, A)\arrow[u, phantom, sloped, "="]\arrow[r] & Hom(B, A)\arrow[r] & Hom(A, A)\arrow[r] & 0\arrow[u, phantom, sloped, "="]\\ 
               & & \delta\arrow[u, phantom, sloped, "\in"]\arrow[r, mapsto] & id_A\arrow[u, phantom, sloped, "\in"]
  \end{tikzcd}\end{center}
  Stąd mamy $\ker \delta\xrightarrow{\beta} C$ i $\gamma:C\to \ker\delta$ i wtedy $\delta\alpha=id_A$ i całość rozszczepia się.
\end{proof}

\begin{conclusion}
  $dh(\mathbf{Ab})\geq 1$, bo $0\to \Z\to \Z\to \Z/2\to 0$ nie rozszczepia się.
\end{conclusion}

\subsection{Warianty kategorii pochodnej}

\begin{enumerate}
  \item $D^+(\mathbf{A})$ - podkategoria rozpinana przez kompleksy zerujące się na lewo od pewnego indeksu ($A^i=0$ dla $i<<0$)
  \item $D^-(\mathbf{A})$ - podkategoria rozpinana przez kompleksy zerujące się na prawo od pewnego indeksu ($A^i=0$ dla $i>>0$)
  \item $D^b(\mathbf{A})$ - podkategoria rozpinana przez kompleksy "ograniczone" ($A^i=0$ dla $|i|>>0$)
\end{enumerate}

\begin{definition}
  W kategorii $\mathbf{A}$ jest \acc{wystarczająco dużo obiektów injektywnych}, jeśli każdy obiekt wkłada się w pewien obiekt injektywny. Analogicznie mówimy, że w $\mathbf{A}$ jest \acc{dostatecznie dużo obiektów projektwynych}, jeśli każdy obiekt jest ilorazem (istnieje surjekcja) obiektu projektywnego.
\end{definition}

\begin{theorem}
  Jeśli w $\mathbf{A}$ jest dostatecznie dużo obiektów injektywnych, to naturalny funktor 
  $K^+(I)\to D^+(\mathbf{A})$ jest równoważnością kategorii.
\end{theorem}

$I$ jest klasą wszystkich injektywnych obiektów kategorii $\mathbf{A}$. Z kolei $K^+(I)$ to wszystkie kompleksy o obiektach injektywnych z dokładnością do homotopii. Analogicznie, $K^-(P)$ będzie zawierało wszystkie kompleksy na obiektach projektywnych z dokładnością do homotopii.

\begin{lemma}\label{lemat skladanie z qis}
  W $K^(\mathbf{A})$ niech $f:X\to Y$ będzie qis. Załóżmy, że mamy kompleks obiektów injektywnych $I^*$. Wtedy
  $$Hom_{K(\mathbf{A})}(Y^*, I^*)\xrightarrow{-\circ s}Hom_{K(\mathbf{A})}(X^*, I^*)$$
  jest izomorfizmem. 
\end{lemma}

\begin{proof}
  W przyszłym tygodniu
\end{proof}

\begin{conclusion}
  Jeśli $A^*$ oraz $I^*$ są kompleksami zaczynającymi się od pewnego miejsca ($\in Kom^+(\mathbf{A})$), a $I^*$ jest injektywny, to wtedy
  $$Hom_{K(\mathbf{A})}(A^*, I^*)\cong Hom_{D(\mathbf{A})}(A^*, I^*)$$
\end{conclusion}

\begin{proof}
  \begin{center}\begin{tikzcd}
    & B^*\arrow[dl, "qis=s" above left] \arrow[dr, "f"] \\ 
    A^*\arrow[rr, "f", dashed, blue] & & I^* 
  \end{tikzcd}\end{center}

  Zaczniemy od pokazania, że $Hom_{K(\mathbf{A})}(A^*, I^*)\to Hom_{D(\mathbf{A})}(A^*, I^*)$ jest epimorfizmem
  \begin{center}\begin{tikzcd}
    & B^*\arrow[dr, "s"] \arrow[dl, "f"] \arrow[dd, "s"] \\ 
    A^* & & I^* \\  
    & A^*\arrow[ul, "id"] \arrow[ur, "f'"]
  \end{tikzcd}\end{center}
  
  Natomiast monomorfizm tej strzałki wynika z tego, że 
  \begin{center}\begin{tikzcd}
    & & B^*\arrow[dl, "s"]\arrow[dr, "s"]\\ 
    & A^*\arrow[dl, "id"]\arrow[drrr, "f"] &   & A^*\arrow[dr, "g"]\arrow[dlll, "id"] \\ 
    A^* & & & & I^* 
  \end{tikzcd}\end{center}
  Chcemy z tego wywnioskować, że skoro $fs\cong gs$, to $f$ jest homotopijne z $g$ i używamy do tego lemat \ref{lemat skladanie z qis}.
\end{proof}
