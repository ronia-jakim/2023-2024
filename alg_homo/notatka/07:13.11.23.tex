\section{Wężowe homologie i kohomologie}

\subsection{Ciąg dokładny kompleksów}

\begin{lemma}
  W kategorii abelowej, krótki ciąg dokładny kompleksów 
  \begin{center}\begin{tikzcd}
    0\arrow[r] & A^* \arrow[r] & B^* \arrow[r] & C^* \arrow[r] & 0 
  \end{tikzcd}\end{center} 
  indukuje długi ciąg dokładny kohomologii:
  \begin{center}\begin{tikzcd}
    & & ... \\ 
    H^{n+1}(A) \arrow[r] & H^{n+1}(B) \arrow[r]\arrow[d, phantom, ""{coordinate, name=Z}] & H^{n+1}(C)  
    \arrow[u, rounded corners, to path={-- ([xshift=2ex]\tikztostart.east) -| ([xshift=4ex]\tikztostart.east) |- ([xshift=4ex]\tikztotarget.west)}] \\ 
    H^n(A) \arrow[r] & H^n(B) \arrow[r]\arrow[d, phantom, ""{coordinate, name=Y}] & H^n(C) 
    \arrow[ull, rounded corners, to path={ -- ([xshift=2ex]\tikztostart.east) |- (Z) [near end]\tikztonodes -| ([xshift=-2ex]\tikztotarget.west) -- (\tikztotarget) }] \\ 
                     & ... & H^{n-1}(C)
    \arrow[ull, rounded corners, to path={ -- ([xshift=2ex]\tikztostart.east) |- (Y) [near end]\tikztonodes -| ([xshift=-2ex]\tikztotarget.west) -- (\tikztotarget) }] \\ 
  \end{tikzcd}\end{center}
\end{lemma}

\begin{dygresja}
  Jeśli jesteśmy w kategorii abelowej, to kategoria jej kompleksów również jest abelowa.
\end{dygresja}

\begin{proof}
  Potrzebujemy dwóch diagramów i troszkę polowania:

  \begin{center}\begin{tikzcd}
    & \coker d_A^n\arrow[r] & \coker d_B^n\arrow[r] & \coker d_C^n \arrow[r] & 0\\ 
    & A^{n+1}\arrow[u]\arrow[r] & B^{n+1}\arrow[u]\arrow[r] & C^{n+1}\arrow[u]\arrow[r] & 0\\ 
    0\arrow[r] & A^n\arrow[r]\arrow[u] & B^n\arrow[r]\arrow[u] & C^n\arrow[r]\arrow[u] & 0\\ 
    0\arrow[r] & \ker d_A^n\arrow[u]\arrow[r] & \ker d_B^n\arrow[u]\arrow[r] & \ker d_C^n\arrow[u]
  \end{tikzcd}\end{center}

  \begin{center}\begin{tikzcd}
    & H^n(A)\arrow[r]\arrow[d] & H^n(B)\arrow[r]\arrow[d]\arrow[ddd, phantom, ""{coordinate, name=Z}] & H^n(C)\arrow[d] 
    \arrow[dddll, rounded corners, to path={ -- ([xshift=2ex]\tikztostart.east) |- (Z) [near end]\tikztonodes -| ([xshift=-2ex]\tikztotarget.west) -- (\tikztotarget) }, orange, thick] \\ 
    & \coker d_A^{n-1}\arrow[r]\arrow[d, "d_A^{n+1}"] & \coker d_B^{n-1} \arrow[r]\arrow[d, "d_B^{n+1}"] & \coker d_C^{n-1} \arrow[d, "d_C^{n+1}"]\arrow[r] & 0\\ 
    0\arrow[r] & \ker d_A^{n+1} \arrow[r]\arrow[d] & \ker d_B^{n+1}\arrow[d]\arrow[r] & \ker d_C^{n+1}\arrow[d] \\ 
      & H^{n+1}(A)\arrow[r] & H^{n+1}(B)\arrow[r] & H^{n+1}(C) 
  \end{tikzcd}\end{center}
  pomarańczowa strzałka wynika z lematu węża, natomiast różniczki dopisaliśmy, ponieważ "przenosi się" na dół.
  \begin{center}\begin{tikzcd}
    A^{n-1}\arrow[r, "d_A^{n-1}"] & A^n\arrow[r, "d_A^n"] \arrow[d, "epi" left] & A^{n+1}\arrow[r, "d_A^{n+1}"] & ... \\ 
                                  & \coker d_A^{n-1} \arrow[r, dashed]\arrow[ur, dashed] & \ker d_A^{n+1}\arrow[u, "mono"]
  \end{tikzcd}\end{center}
\end{proof}

\subsection{Quasi-izomorfizmy i kategoria pochodna}

\begin{definition}[quasi-izomorfizm]
  Morfizm $f:A^\cdot\to B^\cdot$ na kompleksach jest \buff{quasi-izomorfizmem} (qis), jeśli indukuje izomorfizm na kohomologiach, tzn dla każdego $i$ odwzorowanie
  $$H^i(f):H^i(A^\cdot)\to H^i(B^\cdot)$$
  jest izomorfizmem.
\end{definition}

\begin{example}
  \item Dla bardzo ubogiego kompleksu
    \begin{center}\begin{tikzcd}
      ...\arrow[r] & A\arrow[r, "id_A"] & A\arrow[r] & 0 \arrow[r] & ... 
    \end{tikzcd}\end{center}
    mamy $H^\cdot =0 $ i jeśli przyrównany do niego kompleks złożony tylko z $0$, to quasi-izomorfizmem między tymi dwoma kompleksami jest odwzorowanie zerowe.
  \item Mniej trywialny przykład to rozważenie ciągu dokładnego:
    \begin{center}\begin{tikzcd} 
      0 \arrow[r] & \Z\arrow[r, "2\times"] &\Z \arrow[r] &  \Z/2\Z \arrow[r] & 0 
    \end{tikzcd}\end{center}
    Możemy go podmienić na dwa kompleksy z odwzorowaniem między nimi jak niżej
    \begin{center}\begin{tikzcd}
      ...\arrow[r] & 0\arrow[r]\arrow[d] & \Z\arrow[r, "2\times"]\arrow[d, "0"] & \Z \arrow[r]\arrow[d, "epi"] & 0\arrow[d] \arrow[r] & ... \\ 
      ... \arrow[r] & 0\arrow[r] & 0\arrow[r] & \Z/2\Z \arrow[r] & 0\arrow[r] & ...
    \end{tikzcd}\end{center}
    Jedyna nietrywialna grupa kohomologii w obu kompleksach pojawia się w miejscu $\Z/2\Z$ i jest równa $\Z/2\Z$. Czy odwzorowanie, które prawie wszędzie jest zerowe, ale na wysokości nietrywialnej grupy kohomologii staje się epimorfizmem jest quasi-izomorfizmem kompleksów.
\end{example}

\begin{theorem}[kategoria pochodna]
  Niech $\mathbf{A}$ będzie kategorią abelową, natomiast $Kom(\mathbf{A})$ kategorią jej kompleksów. Istnieje \buff{kategoria pochodna} $D(\mathbf{A})$ i funktor $Q:Kom(\mathbf{A})\to D(\mathbf{A})$ taki, że 
  \begin{enumerate}
    \item jeśli $f$ jest qis w $kom(\mathbf{A})$, to $D(f)$ jest izomorfizmem w $D(\mathbf{A})$
    \item jeśli $F:Kom(\mathbf{A})\to D(\mathbf{A})$ spełnia warunek $1$, to faktoryzuje się przez $Q$:
      \begin{center}\begin{tikzcd}
        Kom(\mathbf{A})\arrow[rr, "Q"]\arrow[dr, "F"] & & D(\mathbf{A}) \arrow[dl, "\exists!G"] \\ 
                                                      & D(\mathbf{A})
      \end{tikzcd}\end{center}
  \end{enumerate}

  Kategorię $D(\mathbf{A})$ czasem oznaczamy $D'$.
\end{theorem}
