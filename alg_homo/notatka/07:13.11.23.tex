\section{Wężowe homologie i kohomologie}

\subsection{Ciąg dokładny kompleksów}

\begin{lemma}
  W kategorii abelowej, krótki ciąg dokładny kompleksów 
  \begin{center}\begin{tikzcd}
    0\arrow[r] & A^* \arrow[r] & B^* \arrow[r] & C^* \arrow[r] & 0 
  \end{tikzcd}\end{center} 
  indukuje długi ciąg dokładny kohomologii:
  \begin{center}\begin{tikzcd}
    & & ... \\ 
    H^{n+1}(A) \arrow[r] & H^{n+1}(B) \arrow[r]\arrow[d, phantom, ""{coordinate, name=Z}] & H^{n+1}(C)  
    \arrow[u, rounded corners, to path={-- ([xshift=2ex]\tikztostart.east) -| ([xshift=4ex]\tikztostart.east) |- ([xshift=4ex]\tikztotarget.west)}] \\ 
    H^n(A) \arrow[r] & H^n(B) \arrow[r]\arrow[d, phantom, ""{coordinate, name=Y}] & H^n(C) 
    \arrow[ull, rounded corners, to path={ -- ([xshift=2ex]\tikztostart.east) |- (Z) [near end]\tikztonodes -| ([xshift=-2ex]\tikztotarget.west) -- (\tikztotarget) }] \\ 
                     & ... & H^{n-1}(C)
    \arrow[ull, rounded corners, to path={ -- ([xshift=2ex]\tikztostart.east) |- (Y) [near end]\tikztonodes -| ([xshift=-2ex]\tikztotarget.west) -- (\tikztotarget) }] \\ 
  \end{tikzcd}\end{center}
\end{lemma}

\begin{dygresja}
  Jeśli jesteśmy w kategorii abelowej, to kategoria jej kompleksów również jest abelowa.
\end{dygresja}

\begin{proof}
  Potrzebujemy dwóch diagramów i troszkę polowania:

  \begin{center}\begin{tikzcd}
    & \coker d_A^n\arrow[r] & \coker d_B^n\arrow[r] & \coker d_C^n \arrow[r] & 0\\ 
    & A^{n+1}\arrow[u]\arrow[r] & B^{n+1}\arrow[u]\arrow[r] & C^{n+1}\arrow[u]\arrow[r] & 0\\ 
    0\arrow[r] & A^n\arrow[r]\arrow[u] & B^n\arrow[r]\arrow[u] & C^n\arrow[r]\arrow[u] & 0\\ 
    0\arrow[r] & \ker d_A^n\arrow[u]\arrow[r] & \ker d_B^n\arrow[u]\arrow[r] & \ker d_C^n\arrow[u]
  \end{tikzcd}\end{center}

  \begin{center}\begin{tikzcd}
    & H^n(A)\arrow[r]\arrow[d] & H^n(B)\arrow[r]\arrow[d]\arrow[ddd, phantom, ""{coordinate, name=Z}] & H^n(C)\arrow[d] 
    \arrow[dddll, rounded corners, to path={ -- ([xshift=2ex]\tikztostart.east) |- (Z) [near end]\tikztonodes -| ([xshift=-2ex]\tikztotarget.west) -- (\tikztotarget) }, orange, thick] \\ 
    & \coker d_A^{n-1}\arrow[r]\arrow[d] & \coker d_B^{n-1} \arrow[r]\arrow[d] & \coker d_C^{n-1} \arrow[d]\arrow[r] & 0\\ 
    0\arrow[r] & \ker d_A^{n+1} \arrow[r]\arrow[d] & \ker d_B^{n+1}\arrow[d]\arrow[r] & \ker d_C^{n+1}\arrow[d] \\ 
      & H^{n+1}(A)\arrow[r] & H^{n+1}(B)\arrow[r] & H^{n+1}(C) 
  \end{tikzcd}\end{center}
  pomarańczowa strzałka wynika z lematu węża.
\end{proof}
