\section{23.10.23 : Kategorie abelowe i addytywne, jądra}

\subsection{Funktory sprzężone [adjoint functors]}


\begin{definition}[funktory sprzężone]
  Para funktorów $L:\mathbf{A}\to\mathbf{B}$ i $R:\mathbf{B}\to\mathbf{A}$ nazywamy \buff{parą sprzężoną} ($L$ jest lewo sprzężony do $R$, a $R$ jest prawo sprzężony do $L$), jeśli istnieją naturalne bijekcje (zarówno względem $\mathbf{A}$ jak i $\mathbf{B}$)
  $$Hom_{\mathbf{B}}(L(A),B)\longleftrightarrow Hom_{\mathbf{A}}(A,R(B))$$

  Funktory sprzężone oznaczamy $L\dashv R$
\end{definition}

\begin{example}
  \item Jest sporo przykładów, gdy $R$ jest \acc{funktorem zapominającym}
    \begin{itemize}
      \item jeśli $R:\mathbf{Grp}\to\mathbf{Set}$, wtedy

        \begin{center}\begin{tikzcd}[row sep=tiny]
          Hom_{\mathbf{Grp}}(\star,B)\arrow[r, <->] & Hom_{\mathbf{Set}}(A,B)\\
          \text{grupa}\arrow[u, start anchor={[yshift=-2mm, xshift=3mm]}, end anchor={[xshift=5mm, yshift=2mm]}, bend right=30] & &\text{grupa jako zbiór}\arrow[ul, start anchor={[xshift=1mm]west}, end anchor={[xshift=4mm, yshift=1mm]}, bend left=15]
        \end{tikzcd}\end{center}

        $\star$ będzie grupą wolną o zbiorze generatorów $A$, co oznaczamy $F_A$.
      \item $R:\mathbf{Vect}_K\to\mathbf{Set}$ z bijekcjami zdefiniowanymi jako
        $$Hom_{\mathbf{Vect}_K}(LA,V)\longleftrightarrow Hom_{\mathbf{Set}}(A,V)$$
        gdzie $LA$ to przestrzeń liniowa o bazie równej zbiorowi $A$.
    \end{itemize}
  \item Dla $R$-modułów $A,B,X$ zachodzi
        $$Hom_R(A\otimes X, B)\cong Hom_R(A,Hom_R(X,B))$$
        dla $\phi\in Hom_R(A,Hom_R(X,B))$ mamy
        $$(a\otimes x\mapsto (\phi(a))(x))\mapsto \phi$$
        Dla ustalonego $X$ mamy funktory sprzężone z $R$-modułów w $R$-moduły: $L=-\otimes X$ oraz $R=Hom(X,-)$
  \item Bardzo często włożenie kategorii w inną kategorię jest funktorem mającym funktor sprzężonym.
    \begin{itemize}
      \item Włożenie kategorii $\mathbf{Ab}\hookrightarrow \mathbf{Grp}$ posiada funktor sprzężony:
      $$Hom_{\mathbf{Ab}}(\star, B)\longleftrightarrow Hom_{\mathbf{Grp}}(A, B)$$
      komutant dowolnej grupy $A$ przechodzi przez każdy homomorfizm $\phi:A\to B$ na element neutralny, więc od razu indukwoane mamy przekształcenie $A^{op}\to B$, stąd $\star=A^{op}$.
    \item Włożenie kategorii ciał w dziedziny wyrzuca część homomorfizmów. Mamy
      $$Hom_{\mathbf{Ciala}}(\star,K)\longleftrightarrow Hom_{\mathbf{Dziedziny}}(R,K)$$
      Jeśli mamy odwzorowanie z pierścienia $R$ w ciało $K$, to to odwzorowanie rozszerza się na odwzorowanie z ciała ułamków ciała $R$ w ciało $K$:
  
      \begin{center}\begin{tikzcd}
        R\arrow[rr, "\phi"]\arrow[dr, hookrightarrow] & & K\ni\frac{\phi(p)}{\phi(q)}\\
         & R_0\ni \frac{p}{q}\arrow[ur]
      \end{tikzcd}\end{center}
  
      stąd $\star=R_o$
    \item Włożenie zwartych przestrzeni Hausdorffa w przestrzenie topologiczne $\mathbf{Cpt}T_0\hookrightarrow \mathbf{Top}$ mamy
      $$Hom_{\mathbf{Cpf}T_0}(\star, Y)\longleftrightarrow Hom_{\mathbf{Top}}(X,Y)$$
      więc $\star=\beta X$ czyli uzwarceniem Cecha-Stone'a. To jest maksymalne możliwe uzwarcenie. 
  
      Bierzemy przestrzeń $X$ i patrzymy na wszystkie ciągłe odwzorowania z $X$ w $[0,1]$ i potem odwzorowujemy diagonalnie $X$ w ten produkt, a potem domykamy obraz tego diagonalnego odwzorowania i to jest maksymalne uzwarcenie.
  \end{itemize}
\end{example}

\begin{fact}[jedyność funktora sprzężonego]
  Funktor sprzężony, jeśli istnieje, to jest jedyny z dokładnością do izomorfizmu.
\end{fact}

\begin{proof}
  Bardzo poglądowy, bo trzeba się dokładnie wgryźć w spojrzenie jak to działa na morfizmach.

  $R(B)$ to jedyny element reprezentujący funktor
  $$A^{op}\ni A\mapsto Hom_{\mathbf{B}}(LA,B)\in\mathbf{Set}$$
    Z lematu Yonedy wiemy, że jeśli takie coś istnieje, to jest jedyne z dokładnością do izomorfizmu.
\end{proof}

\begin{fact}[funktory sprzężone zachowują granice (prostą/odwrotną)]\label{funktory sprzezone zachowuja granice}
  Jeśli $L\dashv R$, to $R$ zachowuje granicę, a $L$ kogranicę ($L:\mathbf{C}\to\mathbf{D}$).
\end{fact}

\begin{proof}
  Pokażemy tylko, że prawy sprzężony funktor zachowuje granicę. 

  Wiemy, że $Hom_{\mathbf{C}}(A, RB)\cong Hom_{\mathbf{D}}(LA, B)$. Niech $F$ będzie diagramem w kategorii $\mathbf{D}$, a $G$ jego granicą. 

  Ustalmy dowolny $A\in\ob\mathbf{C}$, który jest stożkiem granica diagramu $R\circ F$, tzn. komutuje diagram 
  \begin{center}\begin{tikzcd}
    & A\arrow[dl, "\Pi_i'" above left]\arrow[dr, "\Pi_j'"]\\ 
    %& RG\arrow[dl, "R\Pi_i" above left] \arrow[dr, "R\Pi_j"]\\ 
    RF(i) \arrow[rr, "RF(\alpha)" below] & & F(j) 
  \end{tikzcd}\end{center}
  W kategorii $\mathbf{D}$ obrazek wygląda następujące:
  \begin{center}
    \begin{tikzcd}
      & LA \arrow[d, dashed, "\exists!\lambda"] \arrow[ddl, "L\Pi_i'" above left, bend right=20] \arrow[ddr, "L\Pi_j'" above right, bend left=20] \\ 
      & G\arrow[dl, "\Pi_i" above left] \arrow[dr, "\Pi_j" above right]\\ 
      F(i)\arrow[rr, "F(\alpha)"] & & F(j) 
    \end{tikzcd}
  \end{center}
  Ze sprzężoności tych funktorów wiemy więc, że istnieje jedyna strzałka $A\to RG$ w kategorii $\mathbf{D}$. Czyli rysunek wygląda
  \begin{center}
    \begin{tikzcd}
      & A \arrow[d, dashed, "\exists!\lambda;"] \arrow[ddl, "\Pi_i'" above left, bend right=20] \arrow[ddr, "\Pi_j'" above right, bend left=20] \\ 
      & RG\arrow[dl, "R\Pi_i" above left] \arrow[dr, "R\Pi_j" above right]\\ 
      RF(i)\arrow[rr, "RF(\alpha)"] & & RF(j) 
    \end{tikzcd}
  \end{center}
  A więc $G$ po przeniesieniu przez $R$ nadal spełnia uniwersalną własność granicy, dla diagramu $R\circ F$.

  %{\large\color{red}OBRAZEK}

  %Musimy wziąć dowolny obiekt $A\in \mathbf{A}$ i sprawdzić, czy $\Pi_i':A\to (R\circ F)(I)$ sfaktoryzuje się w jedyny możliwy sposób na $R\circ R(\Pi_i)$. Musimy wziąć obiekt $LA\in\mathbf{B}$ i tutaj dostajemy jedyną strzałkę $LA\to X$, gdyż $X$ jest granicą. Ale sprzężoność $R$ z $L$ mówi, że mamy jedyność odpowiadania strzałek między elementami $\mathbf{A}$ a elementami $\mathbf{B}$.
\end{proof}

\subsection{Kategorie addytywne i abelowe}

\begin{definition}[kategoria addytywna]
  \buff{Kategoria addytywna} $\mathbf{A}$ to kategoria
  \begin{itemize}
    \item Dla każdej pary obiektów $A,B\in Ob\mathbf{A}$ na $Hom_{\mathbf{A}}(A, B)$ jest określona struktura grupy abelowej. Złożenia są biaddytywne:
      \begin{center}\begin{tikzcd}
        A\arrow[r, "g"] & B\arrow[r, "f", bend left=10]\arrow[r, "f'", bend right=10] & C\arrow[r, "h"] & D
      \end{tikzcd}\end{center}
      $$(f+f')\circ g=f\circ g+f'\circ g$$
      $$h\circ(f+f')=h\circ f+h\circ f'$$
    \item Istnieje \acc{obiekt zerowy} $0$ taki, że $Hom_{\mathbf{A}}(0, 0)=0$ jest grupą trywialną
    \item Dla dowolnej pary obiektów $A, B\in Ob\mathbf{A}$ istnieje obiekt $C$ (zwykle oznaczany $A\oplus B$), który jest ich \acc{produktem} i \acc{koproduktem}, tzn.: istnieją morfizmy
      \begin{center}\begin{tikzcd}
        A\arrow[r, "i_A"] & C\arrow[l, "P_A"]\arrow[r, "P_B"] & B\arrow[l, "i_B"]
      \end{tikzcd}\end{center}
      takie, że $P_a\circ i_A=id_A$ i $P_A\circ i_B=0$ (analogicznie gdy przestawimy $A$ i $B$). Dodatkowo, $i_AP_A+i_BP_B=id_C$.
  \end{itemize}
\end{definition}

Tłumacząc ostatni warunek, chcemy pokazać, że istnieje jedyna strałka $D\to C$:
\begin{center}\begin{tikzcd}
  D\arrow[dr, dashed]\arrow[drr, bend left=20, "f_b"]\arrow[ddr, "f_A", bend right=20] & &\\
                                                       & C\arrow[r, "P_B"]\arrow[d, "P_A"] & B\\
                                                       & A
\end{tikzcd}\end{center}
Zauważmy że $i_Af_A+i_Bf_B:D\to C$, wystarczy więc sprawdzić, czy taka definicja $D\to C$ sprawia, że diagram komutuje, tzn. złożyć ją z $P_A$ i $P_B$:
%({\large\color{red}TUTAJ ZDJĘCIE Z MNÓSTWEM OBLICZEŃ})
$$P_A(i_Af_A+i_Bf_B)=\underbrace{P_Ai_A}_{id_A}f_A+\underbrace{P_Ai_B}_{0}f_B=f_A$$
$$P_B(i_Af_A+i_Bf_B)=\underbrace{P_Bi_A}_{0}f_A+\underbrace{P_Bi_B}_{id_B}f_B=f_B$$
Jeśli istnieją dwa takie odwzorowania, to ich różnica $u$ zamykałaby diagram
\begin{center}\begin{tikzcd}
  D\arrow[drr, "0", bend left=20]\arrow[ddr, "0", bend right=20]\arrow[dr, "u"]\\
  & C\arrow[r, "P_B"]\arrow[d, "f_A"] & B\\
  & A
\end{tikzcd}\end{center}Zauważmy, że 
\begin{align*}
  u&=id_C\circ 0=\\
   &=i_AP_Au+i_BP_Bu=\\
   &=i_A0+i_B0=0+0=0
\end{align*}

Analogicznie pokazuje się dla koproduktu.

\begin{dygresja}[parę słów o zerach]
  Dla dowolnego obiektu $A\in Ob\mathbf{A}$ mamy $Hom(0, A)=0$ i $Hom(0, A)=0$, bo dla $f:A\to 0$ jest $id_0\circ f=f$, czyli $f=0\circ f$, a więc
  $$0f=(0+0)f=0f+0f\implies 0=0f\implies f=0$$
\end{dygresja}

\begin{example}
\item $\mathbf{AB}$
\item $R$-moduły
\item Presnopy grup abelowych na jakiejś przestrzeni topologicznej (lub kategorii) 

  $\mathbf{Pre-snop/AB}(X)$ i od razu zagubione w tym gąszczu snopy.
\end{example}

\begin{definition}[kategoria abelowa]
  Kategoria addytywna jest \buff{abelowa}, jeśli każdy morfizm ma jądro i kojądro i naturalny morfizm z koobrazu w obraz jest izomorfizmem.
\end{definition}

Definicja wyżej często jest formułowana w inny, równoważny, sposób.

\subsection{Monomorfizm, epimorfizm i jądra}

\begin{definition}[jądro, kojądro i przyjaciele]
  Kilka wyjaśnień:
\begin{itemize}
  \item Jądro $f$ to ekwalizator \begin{tikzcd}A\arrow[r, bend left=10, "f"]\arrow[r, bend right=10, "0" below] & B\end{tikzcd}. Inaczej, jest to \begin{tikzcd}K\arrow[r, "k"] &A\end{tikzcd} taki, że
    \begin{enumerate}
      \item \begin{tikzcd}K\arrow[r, "k"] & A\arrow[r, "f"] & B=0\end{tikzcd}
      \item Zachodzi własność uniwersalna:
        \begin{center}\begin{tikzcd}
          \forall\;K'\arrow[r, "k'"]\arrow[d, "\exists!\;u" left]\arrow[rr, "0", bend left=30] & A\arrow[r, "f"] & B\\
          K\arrow[ur, "k" below right]\arrow[urr, "0" below right, bend right=20]
        \end{tikzcd}\end{center}
    \end{enumerate}
  \item Kojądro $f$ to koekwalizator \begin{tikzcd}A\arrow[r, bend left=10, "f"]\arrow[r, bend right=10, "0" below] & B\end{tikzcd} jak w następującym diagramie:
    \begin{center}\begin{tikzcd}
      A\arrow[r, "f"]\arrow[rr, "0", bend left=30]\arrow[drr, "0", bend right=20] & B\arrow[dr, "c'"]\arrow[r, "c"] & C\arrow[d, "\exists!"]\\
    & & C'
    \end{tikzcd}\end{center}
  \item Niech $f:A\to B$, wówczas
    \begin{itemize}
      \item $\img f=\ker(B\to \coker\;f)$
      \item $\coim\;f=\coker(\ker f\to A)$
    \end{itemize}

    \begin{center}\begin{tikzcd}
      K\arrow[r, "k"] & A\arrow[r, "f"]\arrow[d] & B\arrow[r, "c"] & C\\
                      & \coim\;f\arrow[r, dashed, green] & \img f\arrow[u]
    \end{tikzcd}\end{center}
    Naturalne odwzorowanie zaznaczone przerywaną linią ma być izomorfizmem jeśli działaby w kategorii abelowej.
\end{itemize}
\end{definition}

\buff{Ekwalizator} strzałek \begin{tikzcd}A\arrow[r, bend left=10, "f"] \arrow[r, bend right=10, "g" below] & B\end{tikzcd} to obiekt $E$ oraz strzałka $e:E\to A$ taka, że $fe=ge$ oraz jeśli $U\xrightarrow{u}A$ również to spełnia, to istnieje jedyna strzałka $\phi:U\to E$ taka, że $u=e\phi$.


\begin{definition}[mono-, epi-]
  Morfizm $f:X\to Y$ jest
  \begin{itemize}
    \item \buff{monomorfizmem}, jeśli dla dowolnych dwóch odwzorowań $g_1,g_2:Z\to X$ zachodzi
      $$f\circ g_1=f\circ g_2\implies g_1=g_2$$
      W kategorii addytywnej można zamiast powyższego zażądać, żeby dla każdego $g:Z\to X$ $f\circ g=0\implies g=0$
    \item \buff{epimorfizmem} nazywamy morfizm $f:A\to B$ taki, że mając $h_1,h_2:B\to W$ zachodzi
      $$h_1\circ f=h_2\circ f\implies h_1=h_2$$
      W kategorii addytywnej można zamiast tego powiedzieć, że mając $f:A\to B$ i $h:B\to W$ to
      $$hf=0\implies h=0$$
  \end{itemize}
\end{definition}

Można pokazać, że jeśli $f$ jest monomorfizmem, to $\ker f=0$, a jeśli $f$ jest epimorfizmem, to $\coker f=0$.

\begin{lemma}
  Jądra są monomorfizmami, a kojądra są epimorfizmami.
\end{lemma}

\begin{proof}
  W przypadku jądra wystarczy zbadać diagram gdy mamy $kg=0$:
  \begin{center}\begin{tikzcd}[row sep=large]
    K\arrow[r, "k"] & A\arrow[r, "f"] & B\\
    Z\arrow[u, "g" right]\arrow[u, "0" left, bend left=20, green]\arrow[ur, "0"]\arrow[urr, "0" below right]
  \end{tikzcd}\end{center}
  i zauważyć, że jedyność odwzorowania $Z\to K$ wymaga, aby $g=0$. 

  Dla kojądra $A\xrightarrow{f} B\xrightarrow{c}C$ zakładamy $sc=0$, czyli rysujemy z uniwersalności kojądra:
  \begin{center}\begin{tikzcd}[row sep=large]
    A\arrow[r, "f"] & B\arrow[r, "c"]\arrow[rd, "sc=0" below left] & C\arrow[d, "\exists! s" right] \\ 
                    & & D
  \end{tikzcd}\end{center}
  Ponieważ $s$ jest jedyne takie, to $s=0$ a więc $c$ jest epimorfizmem.
\end{proof}

\begin{uwaga}\label{uwaga 5.1}
  Dla każdego morfizmu $f:A\to B$ w kategorii abelowej istnieje jedyny, z dokładnością do izomorfizmu, rozkład
  
  \begin{center}\begin{tikzcd}
    K\arrow[r, "k"] & A\arrow[r, "i" above, "epi" below] & I \arrow[r, "j" above, "mono" below] & B \arrow[r, "c"] & C
  \end{tikzcd}\end{center}
  
  w którym $k=\ker f$, $c=\coker f$, $i=\coker k$ oraz $j=\ker c$ i $f=j\cdot i$.
\end{uwaga}

\begin{proof}
  Załóżmy, że istnieją dwa takie rozkłady:
  
  \begin{center}\begin{tikzcd}
    K\arrow[dd, blue, dashed]\arrow[dr, "k"] & & I\arrow[dr, "j"]\arrow[dd, green, dashed] & & C\arrow[dd, red, dashed]\\
                                             & A\arrow[ur, "i"] \arrow[dr, "i'"] & & B\arrow[dr, "c'"] \arrow[ur, "c"]\\
    K' \arrow[ur, "k'"] & & I' \arrow[ur, "j'"] & & C'
  \end{tikzcd}\end{center}
  Strzałki {\color{blue}niebieska} i {\color{red}czerwona} są izomorfizmami wynikającymi z definicji kategorii abelowej. Strzałkę {\color{green}zieloną} dobbieramy w taki sposób, aby diagram
  
  \begin{center}\begin{tikzcd}
    I\arrow[dr, "j"]\arrow[dd, green, dashed]\\
    & B\\
    I'\arrow[ur, "j'"]
  \end{tikzcd}\end{center}
  komutował. Chcemy jeszcze pokazać, że lewa strona również komutuje, czyli zajmujemy się diagramem

  \begin{center}\begin{tikzcd}
                                      & I\arrow[dd, dashed, green]\arrow[dr, "j"] & \\
    A\arrow[ur, "i"]\arrow[dr, "i'"] &     & B \\
                                      & I' \arrow[ur, "j'"]
  \end{tikzcd}\end{center}

\end{proof}

\begin{lemma}\label{lemat 5.2}
  W kategorii abelowej, jeśli $f$ jest epimorfizmem, to $f=\coker\ker f$, a jeśli $f$ jest monomofizmem, to $f=\ker\coker f$.
\end{lemma}

\begin{proof}
  Zrobimy dowód dla epimorfizmu korzystając z rozkładu przedstawionego wyżej.

  \begin{center}\begin{tikzcd}
    K\arrow[r] & A\arrow[r] & I\arrow[r, "j"] & B\arrow[r, "0"] & 0
  \end{tikzcd}\end{center}
  
  wiemy, że $j$ jest $\ker(B\to 0)$, czyli funkcji zerowej. Czyli musi być $j=id_B$, możemy więc przerysować

  \begin{center}\begin{tikzcd}[row sep=small]
    K\arrow[r]\arrow[d, phantom, sloped, "\cong"] & A\arrow[r]\arrow[d, phantom, sloped, "\cong"] & I\arrow[r, "j"]\arrow[d, phantom, sloped, "\cong"] & B\arrow[r, "0"]\arrow[d, sloped, phantom, "\cong"] & 0\\
    K'\arrow[r] & A\arrow[r, "i"] & B\arrow[r, "id_B"] & B\arrow[r, "0"] & 0 
  \end{tikzcd}\end{center}
  ale przecież $i:A\to I$ było $i=\coker\ker f$, z drugiej strony ponieważ $A\to I\to B$ jest równe $f$, a w tym konkretnym przypadku jest to równe $A\to B\to B$ gdzie druga strzałka to $id_B$, to musi być $i:A\to I=f:A\to B$.
\end{proof}

\begin{uwaga}
  W kategorii addytywnej warunek z \ref{uwaga 5.1} jest równoważny stwierdzeniu, że każdy morfizm ma jądro i kojądro oraz zachodzi lemat \ref{lemat 5.2}
\end{uwaga}

\begin{example}
  \item Rozważmy kategorię abelowych grup topologicznych z warunkiem Hausdorffa. Tworzą one kategorię addytywną. Jądro $\ker f$ to algebraiczne jądro $f$ z dziedziczoną topologią, a $\coker f$ to tak naprawdę iloraz przez domknięcie obrazu $\overline{\img f}$.

  \begin{center}\begin{tikzcd}
    A\arrow[r, "f"] & B\arrow[r] & B/\overline{f[A]}
  \end{tikzcd}\end{center}

  Przez taką definicję $\coker$ mamy kategorię addytywną, która nie jest kategorią abelową.

  Wystarczy sprawdzić
  \begin{center}\begin{tikzcd}
    0\arrow[r] & \R^\delta \arrow[r] & \R \arrow[r] & 0
  \end{tikzcd}\end{center}
  gdzie $\R^\delta$ ma topologię dyskretną, a $\R$ traktujemy jako zwykłą przestrzeń euklidesową. Wtedy nie mamy naturalnego izomorfizmu między $\ker \coker f$ a $\coker\ker f$, bo jedna z nich ma odziedziczoną topologię dyskretną, a druga nie.
  %kojądrami {\large\color{red}JESZCZE RAZ PRZEMYŚLEĆ TEN PRZYKŁAD}

\item Podstawowym przykładem kategorii abelowej jest kategoria $R$-modułów. Bardzo często kiedy pracujemy w kategorii abelowej zachowujemy się jakbyśmy byli w kategorii $R$-modułów na mocy twierdzenia Freyd-Mitchella:

  \begin{dygresja}[twierdzenie Freyd-Mitchella]
    Mała kategoria belowa ma wierne, pełne i dokładne zanurzenie w kategorię $R$-modułów dla pewnego $R$.
  \end{dygresja}
\end{example}

