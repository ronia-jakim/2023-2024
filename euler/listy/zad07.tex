\textbf{\large\color{orange}Zadanie 7.} Grassmannian $Gr_{\C}(k, n)$ ma pewien podział na komórki, który możemy opisać za pomocą szufladek i groszków. Rozważmy $n$ szufladek, w których umieszczać będziemy $k$ groszków, co najwyżej po jednym w danej szufladzie. Takie rozmieszczenie groszków reprezentuje zbiór $k$-wymiarowych podprzestrzeni $\C^n$. Kolejne $l$ szufladek od lewej reprezentuje podprzestrzeń $\C^n$ rozpiętą przez pierwsze $l$ wektorów bazowych $e_1, e_2,..., e_l$, a liczba groszków leżących w $l$ pierwszych $l$ szufladkach to wymiar przekroju $k$-wymiarowej podprzestrzeni z tego zbioru z podprzestrzenią rozpiętą przez $e_1,..., e_l$.
\begin{enumerate}[label=(\alph*)]
  \item Pokaż, że konkretne rozmieszczenie groszków w szufladach reprezentuje przestrzeń $k$-wymiarowych podprzestrzeni $\C^n$ izomorficzna z $\C^m$, gdzie $m$ to liczba przesunięć groszków w lewo o jedną szufladkę dopóki to możliwe.
  \item Przestrzeń $\C^m$ z poprzedniego podpunktu to otwarta komórka wspomnianego rozkładu. Komórka odpowiadająca rozmieszczeniu groszków $A$ zawiera się w
domknięcu komórki odpowiadającej rozmieszczeniu $B$, gdy $A$ można otrzymać z $B$ poprzez kolejne przesunięcia groszków w lewo o jedną szufladkę.
Domknięcie komórki odpowiadającej rozmieszczeniu $A$ nazywamy $(A)$ rozmaitością Schuberta. Policz charakterystykę Eulera rozmaitości Schuberta.
Policz charakterystykę Eulera $Gr_{\C}(k, n)$ zliczając te komórki
\end{enumerate}

\dotfill 

\begin{enumerate}[label=\textbf{(\alph*)}]
  \item Zacznijmy od rozmieszczenia groszków tak, że nie możemy już żadnego przesunąć w lewo. To znaczy, że podprzestrzenie które są kodowane przez to ustawienie groszków kroją się niepusta z podprzestrzenią rozpinaną przez pierwszy wektor bazowy $e_1$, z podprzestrzenią rozpiętą przez dwa pierwsze wektory bazowe $e_1$, $e_2$ i tak dalej. W takim razie, typowa podprzestrzeń reprezentowana przez takie ustawienie jest generowana przez wektory
    \begin{align*}
      &e_1\\ 
      &a_1^1e_1+e_2\\ 
      &...\\ 
      &a_1^ke_1+a_2^ke_2+...+e_k
    \end{align*}
    Interesuje nas przestrzeń rozpinana przez takie wektory, więc w $i$-tym wektorze możemy usunąć część przychodzącą z $j<i$ wektorami. W ten sposób dostaniemy przestrzeń rozpiętą przez $e_1, e_2, ...,e_k$. Nie mamy żadnego parametru, więc jest to izomorficzne z punktem, czyli z $\C^0$.

    Załóżmy, że mamy $k$ groszków umieszczonych odpowiednio w szufladkach o numerze $m_1$, $m_2$, ..., $m_k$. Dzięki pierwszemu groszkowi możemy do naszej $k$-wymiarowej podprzestrzeni $\C^n$ wybrać wektor
    $$a_1^{m_1}e_1+a_2^{m_1}e_2+...+e_{m_1}.$$
    Kolejny groszek, na $m_2\neq m_1$ miejscu pozwoli nam dołożyć wektor
    $$a_1^{m_2}e_1+a_2^{m_2}e_2+...+a_{m_1}^{m_2}+...+e_{m_2}.$$
    Ze współczynników pojawiających się przy kolejnych $e_i$ możemy stworzyć macierz
    $$
    \begin{bmatrix}
      a_1^{m_1} & a_2^{m_1} & ... & 1 & 0 & ... & 0 & ... & 0\\ 
      a_1^{m_2} & a_2^{m_2} & ... & a_{m_1}^{m_2} & ... & ... & 1 & 0 & ...\\ 
      \vdots & \vdots & \vdots & \vdots & \vdots\\ 
      a_1^{m_k} & a_2^{m_k} & ... & a_{m_1}^{m_k} & ... & ... & ... & ... & 1 
    \end{bmatrix}
    $$
    która ma $m_k$ kolumn i $k$ wierszy. Możemy skorzystać z algorytmu eliminacji Gaussa, by dostać w pierwszych $k$ kolumnach kwadratową macierz górnotrójkątną z $1$ na przekątnej. 

    W pierwszym wierszu zostaje nam $(m_1-1)$ zmiennych, w drugim wierszu mamy $(m_2-2)$ nowych zmiennych i tak dalej. Sumarycznie dostajemy
    $$\sum_{i=1}^k(m_i-i)$$
    parametrów w takiej macierzy, co jest równe ilości potencjalnych przesunięć groszków: $i$-ty groszek może przejść przez co najwyżej $(m_i-i)$ szufladek, niekoniecznie za jednym zamachem.

    Pokazaliśmy, że jeśli możemy dokonać $m$ przesunięć groszków, to takie ustawienie możemy zapisać jako przestrzeń liniową przy pomocy $m$ parametrów.

  \item W tym podpunkcie pytamy tak naprawdę, na ile możliwości możemy dojść do "trywialnego" ułożenia zaczynając w ułożeniu $A$. Każde ułożenie pośrednie będziemy ewaluować $1$ lub $(-1)$ w zależności od tego, czy jest izomorficzne z $\C^{2k}$ (wtedy $+1$) czy z $\C^{2k+1}$ (wówczas $-1$).

    %Ułożenie groszków jest jednoznacznie wyznaczone przez rozmieszczenia groszków $m_1$,..., $m_k$. Naszym celem jest zliczenie na ile sposobów możemy odejmować od pojedynczych $m_i$ liczbę $1$ tak, by cały czas był zachowany porządek $m_1<...<m_k$ i aby końcowym wynikiem było $m_1=1<m_2=2<...<m_k=k$.

    %Możemy ten problem widzieć też jako $k$ sztuk szufladek tak, że w pierwszej jest $m_1$ groszków, $m_2$ jest $m_2-m_1$ groszków i tak dalej. W ostatniej, $k$-tej szufladce jest $0$ groszków -> naszym celem jest wrzucenie do $k$-tej szufladki wszystkich $(m_k-k)$ groszków, przesuwając je o jedno w prawo.

    Ułożenie $k$ groszków w $n$ szufladkach jest jednoznacznie wyznaczane przez położenia $m_1<m_2<...<m_k$ kolejnych groszków. Z takiego ułożenia możemy otrzymać inne przez przesuwanie w lewo dowolne ustawienie, w której na pozycjach $>m_{k-1}$ jest co najwyżej jeden groszek, na pozycjach $>m_{k-2}$ - co najwyżej dwa groszki i tak dalej.

    Możemy ułożyć $m_k$ zer w ciągu i $k$ spomiędzy nich zamienić na jedynki, symbolujące położenie groszków, i zastanawiać się, które ułożenia są niedozwolone. $k$ jedynki możemy wybrać bez restrykcji na $\binom{m_k}{k}$ sposobów. 

    Niedozwolone ułożenia posiadają co najmniej $(i+1)$ groszków na pozycjach $m_{k-i}$. Takich ułożeń jest $\binom{m_k-m_{k-i}}{i+1}\cdot\binom{m_k-i-1}{k-i-1}$. Dostajemy z zasady włączeń i wyłączeń wzór
    $$\sum_{i=0}(-1)^i\binom{m_k-m_{k-i}}{i+1}\cdot\binom{m_k-i-1}{k-i-1}$$
    na ilość możliwych przesunięć w lewo groszków z ustawienia $m_1<m_2<...<m_k$.

    Teoretycznie charakterystyka Eulera powinna wyjść $\binom{n}{k}$, czyli na ile sposobów możemy ułożyć $k$ groszków w $n$ szufladkach. Jest to równe ilości wierzchołków w grafie, który od $k$ groszków ustawionych skrajnie po prawej stronie przechodzi do $k$ groszków ustawionych skrajnie po lewej stronie na wszystkie możliwe sposoby.
\end{enumerate}
