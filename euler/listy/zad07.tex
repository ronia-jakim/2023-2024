\textbf{\large\color{orange}Zadanie 7.} Grassmannian $Gr_{\C}(k, n)$ ma pewien podział na komórki, który możemy opisać za pomocą szufladek i groszków. Rozważmy $n$ szufladek, w których umieszczać będziemy $k$ groszków, co najwyżej po jednym w danej szufladzie. Takie rozmieszczenie groszków reprezentuje zbiór $k$-wymiarowych podprzestrzeni $\C^n$. Kolejne $l$ szufladek od lewej reprezentuje podprzestrzeń $\C^n$ rozpiętą przez pierwsze $l$ wektorów bazowych $e_1, e_2,..., e_l$, a liczba groszków leżących w $l$ pierwszych $l$ szufladkach to wymiar przekroju $k$-wymiarowej podprzestrzeni z tego zbioru z podprzestrzenią rozpiętą przez $e_1,..., e_l$.
\begin{enumerate}[label=(\alph*)]
  \item Pokaż, że konkretne rozmieszczenie groszków w szufladach reprezentuje przestrzeń $k$-wymiarowych podprzestrzeni $\C^n$ izomorficzna z $\C^m$, gdzie $m$ to liczba przesunięć groszków w lewo o jedną szufladkę dopóki to możliwe.
  \item Przestrzeń $\C^m$ z poprzedniego podpunktu to otwarta komórka wspomnianego rozkładu. Komórka odpowiadająca rozmieszczeniu groszków $A$ zawiera się w
domknięcu komórki odpowiadającej rozmieszczeniu $B$, gdy $A$ można otrzymać z $B$ poprzez kolejne przesunięcia groszków w lewo o jedną szufladkę.
Domknięcie komórki odpowiadającej rozmieszczeniu $A$ nazywamy $(A)$ roz-
maitością Schuberta. Policz charakterystykę Eulera rozmaitości Schuberta.
Policz charakterystykę Eulera $Gr_{\C}(k, n)$ zliczając te komórki
\end{enumerate}
