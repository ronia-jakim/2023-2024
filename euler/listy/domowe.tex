\documentclass{article}

\usepackage{../../template}
\usepackage{tikz,pgfplots}
\usepackage{tikz-3dplot}

\usepackage{wrapfig}
\usetikzlibrary{calc,intersections}

\DeclareMathOperator{\Conf}{Conf}
\DeclareMathOperator{\conf}{conf}

\title{Charakterystyka Eulera\\{\normalsize Zadanie domowe}}
\author{Weronika Jakimowicz}
\date{}

\makeatletter
\renewcommand \dotfill {\leavevmode \cleaders \hb@xt@ .88em{\hss .\hss }\hfill \kern \z@}
\makeatother

\usetikzlibrary{decorations.markings}

\def\Singlearrow{{\arrow[scale=1.5,xshift={0.5pt+2.25\pgflinewidth}]{>}}}
\def\Doublearrow{{\arrow[scale=1.5,xshift=1.35pt +2.47\pgflinewidth]{>>}}}
\def\Triplearrow{{\arrow[scale=1.5,xshift=1.75pt +2.47\pgflinewidth]{>>>}}}

\newcommand{\groszki}[2]{
  \begin{scope}[shift={#1}]
      \draw (0, 0)--(2.5, 0);
      \draw (0, -0.5)--(2.5, -0.5);
      \foreach \i in {0, 1, 2, 3, 4, 5} \draw (\i/2, 0)--(\i/2, -0.5);
      \foreach \j in {#2} \fill[green!80!orange] (\j/2+0.25, -0.25) circle (3pt);
 \end{scope}
}

\begin{document}
\maketitle

\textbf{\large\color{orange}Zadanie 1.} Opisz grupę automorfizmów triangulacji $\R P^2$ o najmniejszej liczbie wierzchołków.

{\large\color{red}usunąć kolokwializmy, pokazać, że jądro $Aut(D)\to S_5=\Z_2$, na początku uzasadnić $2E=3T$, ładniej pokazać, że sześciany są trzymane przez automorfizmy}

\textbf{Ile wierzchołków?}

Zacznijmy od zauważenia, że patrzymy na $2$-rozmaitość, czyli dowolny punkt triangulacji ma otoczenie, które wygląda jak mała sfera w $\R^2$. Na $\R^2$ każda krawędź przylega do $2$ trójkątów, a każdy trójkąt ma $3$ krawędzie, stąd
$$2E=3T.$$

Wiemy, że jeśli $X$ ma triangulację o $V$ wierzchołkach, $E$ krawędziach i $T$ trójkątach, to 
$$\chi(X)=V-E+T.$$
Jak wcześniej zauważyliśmy, $2E=3T$, więc mamy
$$\chi(X)=V-E+\frac{2}{3}E=V-\frac{1}{3}E.$$
Ilość krawędzi szacujemy od góry przez ilość krawędzi w grafie pełnym: $E\leq \binom{V}{2}$ czyli
$$V=\chi(X)+\frac{1}{3}E\leq \chi(X)+\frac{V(V-1)}{6}$$
dla $\R P^2$ dostajemy więc ograniczenie
$$V\leq 1+\frac{V(V-1)}{6}$$
$$6\geq 6V-V^2+V=V(7-V)$$
Powyższa nierówność dla $V=6$ staje się równością. Tak samo dla $V=1$ mamy równość, ale z oczywistego powodu nie ma jednowierzchołkowej triangulacji na $\R P^2$. Pozostałe liczby naturalne z przedziału $(0, 7)$ nie mają szansy spełniać powyższe równanie (widać na obrazku)
\begin{center}
  \begin{tikzpicture}
  \begin{axis}[
    axis x line=center,
    axis y line=center,
    xtick={0, 2, 4, 6, 8},
    ytick={0, 2, 4, 6, 8, 10, 12},
    xlabel={$x$},
    ylabel={$y$},
    xlabel style={below right},
    ylabel style={left},
    xmin=-1.5,
    xmax=8.5,
    ymin=-1.5,
    ymax=13.5]
  \addplot [mark=none,domain=-1:7] {x*(7-x)};
  \addplot [domain=-1:7, dashed] {6};
  \addplot [dashed] coordinates {(6, -0.5) (6, 12)};
  \end{axis}
  \end{tikzpicture}
\end{center}

Z listy 1 wiemy, że $6$ wierzchołkowa triangulacja $\R P^2$ jest jedyna z dokładnością do izomorfizmu, czyli nie musimy się martwić którą triangulacje opisujemy.

\textbf{Plan działania}

Płaszczyzna rzutowa $\R P^2$ to $S^2$ wydzielona przez antypodyczne działanie $\Z_2$. W takim razie, $6$ wierzchołkowa triangulacja na $\R P^2$ przychodzi od triangulacji na $S^2$. Dwudziestościan ma $12$ wierzchołków i $20$ ścian i jest to interesująca nas triangulacja sfery. Łatwiejsze jest jednak badanie grupy automorfizmów bryły dualnej do dwudziestościanu - dwunastościanu o $12$ ścianach i $20$ wierzchołkach.

{\large\color{red}narysować na sferze z osiami symetrii}

Z dodecahedronu możemy dostać icosahedron - wystarczy postawić wierzchołek na każdej ścianie i połączyć odpowiednio wierzchołkami. W ten sam sposób można z icosahedronu wrócić do dodecahedronu. Stąd grupy automorfizmów obu tych brył będą równe i wystarczy popatrzeć na dodecahedron $D$:

\begin{wrapfigure}{l}{7cm}
\tdplotsetmaincoords{70}{110}
\begin{tikzpicture}[tdplot_main_coords]
	% Change the value of the number at {\escala}{##} to scale the figure up or down
	\pgfmathsetmacro{\escala}{2}
	% Coordinates of the vertices
%  \node(1) at (-\escala*1.37638, \escala*0., \escala*0.262866) {1};
%  \node(2) at (\escala*1.37638, \escala*0., -\escala*0.262866) {2};
%  \node(3) at (-\escala*0.425325, -\escala*1.30902, \escala*0.262866) {3};
%  \node(4) at (-\escala*0.425325, \escala*1.30902, \escala*0.262866) {4};
%  \node(5) at (\escala*1.11352, -\escala*0.809017, \escala*0.262866) {5};
%  \node(6) at (\escala*1.11352, \escala*0.809017, \escala*0.262866) {6};
%  \node(7) at (-\escala*0.262866, -\escala*0.809017, \escala*1.11352) {7};
%  \node(8) at (-\escala*0.262866, \escala*0.809017, \escala*1.11352) {8};
%  \node(9) at (-\escala*0.688191, -\escala*0.5, -\escala*1.11352) {9};
%  \node(10) at (-\escala*0.688191, \escala*0.5, -\escala*1.11352) {10};
%	%
%  \node(11) at (\escala*0.688191, -\escala*0.5, \escala*1.11352) {11};
%  \node(12) at (\escala*0.688191, \escala*0.5, \escala*1.11352) {12};
%  \node(13) at (\escala*0.850651, \escala*0., -\escala*1.11352) {13};
%  \node(14) at (-\escala*1.11352, -\escala*0.809017, -\escala*0.262866) {14};
%  \node(15) at (-\escala*1.11352, \escala*0.809017, -\escala*0.262866) {15};
%  \node(16) at (-\escala*0.850651, \escala*0., \escala*1.11352) {16};
%  \node(17) at (\escala*0.262866, -\escala*0.809017, -\escala*1.11352) {17};
%  \node(18) at (\escala*0.262866, \escala*0.809017, -\escala*1.11352) {18};
%  \node(19) at (\escala*0.425325, -\escala*1.30902, -\escala*0.262866) {19};
%  \node(20) at (\escala*0.425325, \escala*1.30902, -\escala*0.262866) {20};
	% Faces of the dodecahedron
 
	\coordinate(1) at (-\escala*1.37638, \escala*0., \escala*0.262866);
	\coordinate(2) at (\escala*1.37638, \escala*0., -\escala*0.262866);
	\coordinate(3) at (-\escala*0.425325, -\escala*1.30902, \escala*0.262866);
	\coordinate(4) at (-\escala*0.425325, \escala*1.30902, \escala*0.262866);
	\coordinate(5) at (\escala*1.11352, -\escala*0.809017, \escala*0.262866);
	\coordinate(6) at (\escala*1.11352, \escala*0.809017, \escala*0.262866);
	\coordinate(7) at (-\escala*0.262866, -\escala*0.809017, \escala*1.11352);
	\coordinate(8) at (-\escala*0.262866, \escala*0.809017, \escala*1.11352);
	\coordinate(9) at (-\escala*0.688191, -\escala*0.5, -\escala*1.11352);
	\coordinate(10) at (-\escala*0.688191, \escala*0.5, -\escala*1.11352);
	%
	\coordinate(11) at (\escala*0.688191, -\escala*0.5, \escala*1.11352);
	\coordinate(12) at (\escala*0.688191, \escala*0.5, \escala*1.11352);
	\coordinate(13) at (\escala*0.850651, \escala*0., -\escala*1.11352);
	\coordinate(14) at (-\escala*1.11352, -\escala*0.809017, -\escala*0.262866);
	\coordinate(15) at (-\escala*1.11352, \escala*0.809017, -\escala*0.262866);
	\coordinate(16) at (-\escala*0.850651, \escala*0., \escala*1.11352);
	\coordinate(17) at (\escala*0.262866, -\escala*0.809017, -\escala*1.11352);
	\coordinate(18) at (\escala*0.262866, \escala*0.809017, -\escala*1.11352);
	\coordinate(19) at (\escala*0.425325, -\escala*1.30902, -\escala*0.262866);
	\coordinate(20) at (\escala*0.425325, \escala*1.30902, -\escala*0.262866);

%	\draw[thick,cyan, opacity=0.75]  (2) -- (6) -- (12) -- (11) -- (5) -- (2);
%	\draw[thick,cyan, opacity=0.75]  (5) -- (11) -- (7) -- (3) -- (19) -- (5);
%	\draw[thick,cyan, opacity=0.75]  (11) -- (12) -- (8) -- (16) -- (7) -- (11);
%	\draw[thick,cyan, opacity=0.75]  (12) -- (6) -- (20) -- (4) -- (8) -- (12);
%	%
%	\draw[thick,cyan, opacity=0.75]  (6) -- (2) -- (13) -- (18) -- (20) -- (6);
%	\draw[thick,cyan, opacity=0.75]  (2) -- (5) -- (19) -- (17) -- (13) -- (2);
%	\draw[thick,cyan, opacity=0.75]  (4) -- (20) -- (18) -- (10) -- (15) -- (4);
%	\draw[thick,cyan, opacity=0.75]  (18) -- (13) -- (17) -- (9) -- (10) -- (18);
%	\draw[thick,cyan, opacity=0.75]  (17) -- (19) -- (3) -- (14) -- (9) -- (17);
%	%
%	\draw[thick,cyan, opacity=0.75]  (3) -- (7) -- (16) -- (1) -- (14) -- (3);
%	\draw[thick,cyan, opacity=0.75]  (16) -- (8) -- (4) -- (15) -- (1) -- (16);
%	\draw[thick,cyan, opacity=0.75]  (15) -- (10) -- (9) -- (14) -- (1) -- (15);
	%	

  \draw[orange, thick] (19)--(7)--(1)--(9)--(19);
  \draw[orange, thick, dashed] (18)--(2)--(12)--(4)--(18);
  \draw[orange, thick] (9)--(18)--(4)--(1);
  \draw[dashed, orange, thick] (7)--(12);
  \draw[dashed, orange, thick] (19)--(2);

  \draw[dashed, thick] (20)--(6)--(2)--(13);
  \draw[dashed, thick] (6)--(12)--(11)--(5)--(2)--(6);
  \draw[dashed, thick] (12)--(8);
  \draw[dashed, thick] (7)--(11);
  \draw[dashed, thick] (5)--(19);
  \draw[thick] (7)--(16)--(1)--(14)--(3)--(7);
  \draw[thick] (16)--(8)--(4)--(15)--(1);
  \draw[thick] (14)--(9)--(10)--(15);
  \draw[thick] (3)--(19)--(17)--(9);
  \draw[thick] (17)--(13)--(18)--(10);
  \draw[thick] (18)--(20)--(4);
\end{tikzpicture}

\end{wrapfigure}

Uzasadnimy teraz to, co prawi Wikipedia, mianowicie, że $Aut(D)=A_5\times \Z_2$.

\textbf{Czy zgadza się rząd?}

Niech $v\in D$ będzie wierzchołkiem dodecahedronu (odpowiada ścianie icosahedronu). 
\begin{itemize}
  \item $|Obr(v)|=20$, bo automorfizm może posłać wierzchołek na dowolny inny spośród $20$ które $D$ posiada.
  \item $|Stab(v)| = 3!=6$, gdyż są to permutacje $3$ sąsiadów tego wierzchołka przy trzymaniu $v$ w miejscu.
\end{itemize}
W takim razie dostajemy
$$|Aut(D)|=|Orb(v)|\cdot|Stab(v)|=20\cdot 6=120=|A_5\times \Z_2|.$$

\textbf{Pozbycie się $\Z_2$}

Wśród automorfizmów dodecahedronu $D$ mamy dwa "rodzaje" odwzorowań
\begin{itemize}
  \item rotacje i symetrie, które zachowują ruch wskazówek zegara przy numerowaniu sąsiadów dowolnego wierzchołka,
  \item odwzorowanie antypodyczne tudzież symetria względem punktu w samym środku $D$, która przewraca tę kolejność do góry nogami.
\end{itemize}
Ten drugi rodzaj odwzorowania będzie odpowiadać za czynnik $\Z_2$ w $Aut(D)$. Wystarczy więc zająć się samą grupą symetrii i rotacji i pokazać, że to $A_5$.

\textbf{Symetrie i obroty}

Sztuczką na pokazanie, że symetrie $D$ to $A_5$ jest zauważenie $5$ sześcianów w środku $D$. Sześciany możemy narysować idąc krokami:
\begin{itemize}
  \item weź krawędź w $D$
  \item połącz wszystkie sąsiady tej krawędzi w ścianę
  \item weź krawędź po przeciwnej stronie $D$
  \item połącz jej wszystkie sąsiady w ścianę
  \item połącz te dwie ściany w sześcian.
\end{itemize}
Z tej metody wytwarzania sześcianów można od razu wywnioskować, że automorfizm przeprowadza sześciany na sześciany, ponieważ sąsiedztwo wierzchołków musi być zachowane, a to ono było podstawą wyciskania sześcianów z $D$.

Ponumerujmy sześciany od $1$ do $5$ - możemy teraz je permutować. Najbardziej leniwym sposobem na zauważenie, że grupa uzyskana przez porządne permutacje tych sześcianów to $A_5$ jest podzielenie $|Aut(D)|=120$ przez $2$, które oznacza, że wyrzucamy antypodyzm (element rzędu $2$). Zostawia to nam $60$ automorfizmów, które będą permutować te sześciany i które powinniśmy móc włożyć w $S_5$. Jedyna (z dokładnością do izomorfizmu) podgrupa $S_5$ o $60$ elementach jest $A_5$ tak jak chcieliśmy.

Uzasadniliśmy, że $A_5\times\Z_2=Aut(\text{\emph{dodecahedron}})=Aut(\text{\emph{icosahedron}})$ bo tak jak już wspomniałam, bryły te są dualne. Po wydzieleniu $S^2$ z triangulacją będącą icosahedronem przez działanie antypodyczne dostajemy grupę automorfizmów triangulacji $\Delta \R P^2$ o $6$ wierzchołkach: 
$$Aut(\Delta\R P^2)=A_5\times\Z_2/\Z_2=A_5$$


\textbf{\large\color{orange}Zadanie 4.} Niech $X$ będzie przestrzenią topologiczną. Definiujemy przestrzeń konfiguracji $\Conf_n(X)$ jako przestrzeń położeń $n$ różnych punktów w $X$ (zakładamy, że dwa punkty nie mogą leżeć w tym samym miejscu. Definiujemy też przestrzeń konfiguracji $\conf_n(X)$ jako przestrzeń położeń $n$ nierozróżnialnych punktów w $X$, czyli
$$\Conf_n(X)=\{(x_1,...,x_n)\in X^n\;:\;x_i\neq x_j, \text{ gdy }i\neq j\}$$
$$\conf_n(X)=\frac{\Conf_n(X)}{S_n}$$
gdzie $S_n$ działa na $\Conf_n(X)$ przez permutacje współrzędnych.

Oblicz charaketrystykę Eulera $\conf_n(\mathbf{Y})$, gdzie $\mathbf{Y}$ to drzewo o $4$ wierzchołkach, z czego $3$ to liście, dla $n=2,3,4$.

\dotfill

\textbf{$\mathbf{n=2}$}

Zauważmy, że w każdym punkcie $\Conf_n(\mathbf{Y})$ leży niemalże identyczna kopia $\mathbf{Y}$ z tym, że brakuje w niej jednego punktu -> tego, który miałby obie współrzędne równe. Korzystając z addytywnej definicji charakterystyki Eulera, graf $\mathbf{Y}$ ma charakterystykę $\chi(\mathbf{Y})=1$. W takim razie, jeśli wyjmiemy z niego punkt to dostajemy graf z $\chi(\mathbf{Y}-\bullet)=1-1=0$.

Formuła Riemanna-Hurwitza mówi, że jeśli mamy funkcję $f:\Conf_n(\mathbf{Y})\to \mathbf{Y}$, to wtedy
$$\chi(\Conf_n(\mathbf{Y}))=\int_{\mathbf{Y}}\chi(f^{-1}(X))d\chi(X)$$
W tym konkretnym przypadku od razu w oczy rzuca się funkcja
$$f(x, y)=x,$$
w której przeciwobrazem dowolnego punktu jest $\mathbf{Y}$ bez punktu. W takim razie 
$$\chi(\Conf_n(\mathbf{Y}))=\chi(\mathbf{Y})\chi(\mathbf{Y}-\bullet)=1\cdot 0=0.$$

Zauważmy, że teraz możemy napisać funkcję $g:\Conf_n(\mathbf{Y})\to \conf_n(\mathbf{Y})$ która "skleja" dwa punkty będące swoimi permutacjami. W takim razie nad dowolnym punktem w $\conf_n(\mathbf{Y})$ wiszą dwa punkty w $\Conf_n(\mathbf{Y})$. W takim razie 
$$2\cdot\chi(\Conf_n(\mathbf{Y}))=\chi(\conf_n(\mathbf{Y}))$$
i $\chi(\conf_n(\mathbf{Y}))=0$.
\bigskip

$\mathbf{n=3}$ \dotfill

Rozważmy teraz funkcję 
$$f:\Conf_3(\mathbf{Y})\to \Conf_2(\mathbf{Y})$$ 
taką, że 
$$f(x, y, z)=(x, y).$$
Teraz nad każdym punktem $\Conf_2(\mathbf{Y})$ wisi kopia $\mathbf{Y}$ bez dwóch punktów. Charakterystyka $\mathbf{Y}$ bez dwóch punktów to $-2$, mamy więc
$$\chi(\Conf_3(\mathbf{Y}))=\int_{\Conf_2(\mathbf{Y})}\chi(f^{-1}(X))d\chi(X)=\chi(\Conf_2(\mathbf{Y}))\cdot\chi(\mathbf{Y}-\bullet-\bullet)=-2\cdot0=0.$$
Oczywiste jest również odwzorowanie $g:\Conf_3(\mathbf{Y})\to \conf_3(\mathbf{Y})$, które skleja $6$ punktów mających te same współrzędne w jeden punkt. W takim razie
$$\chi(\conf_3(\mathbf{Y}))=\frac{1}{4}\chi(\Conf_3(\mathbf{Y}))=\frac{1}{4}\cdot 0=0.$$
\bigskip

$\mathbf{n=4}$ \dotfill

Analogicznie jak wcześniej, funkcja $f:\Conf_4(\mathbf{Y})\to \Conf_3(\mathbf{Y})$ wraz z całką Riemanna-Hurwitza mówi, że
$$\chi(\Conf_4(\mathbf{Y}))=0.$$
Ponieważ odwzorowanie $g:\Conf_4(\mathbf{Y})\to \conf_4(\mathbf{Y})$ jest $4!$-krotne, to 
$$\chi(\conf_4(\mathbf{Y}))=\frac{0}{4!}=0.$$


\textbf{\large\color{orange}Zadanie 5.} Mapą na powierzchni $M$ nazywamy podział powierzchni na komórki homeomorficzne z dyskami, których przekroje są zawarte w ich brzegach. Z takim podziałem
mamy związany graf dualny, którego wierzchołki, to komórki, a krawędź istnieje
pomiędzy wierzchołkami, gdy odpowiadające im komórki mają niepusty przekrój.
Kolorowaniem mapy nazywać będziemy funkcję ze zbioru komórek w pewien skończony zbiór kolorów, która przyjmuje różne wartości na krojących się niepusto
komórkach.
\begin{enumerate}[label=(\alph*)]
  \item Jak mogą wyglądać mapy na powierzchniach? Czy da się uprościć je tak, by
graf dualny był $1$-szkieletem triangulacji? Rozważyć mapę o sześciu krajach
na butelce Kleina.
\item Twierdzenie o $k$ barwach dla powierzchni $M$ mówi, że każdą mapę na powierzchni $M$ można pokolorować co najwyżej $k$ kolorami. Udowodnić twierdzenie o $5$ barwach dla sfery $S^2$, o $6$ barwach dla $RP^2$ o $7$ barwach dla torusa
$T^2$ i o $6$ barwach dla butelki Kleina.
\end{enumerate}

\dotfill 

\begin{enumerate}[label=\textbf{(\alph*)}]
  \item W mapie pozwalamy, aby w jednym punkcie spotykały się co najwyżej $3$ kraje. Jeśli istnieje punkt, w którym spotykają się $4$ kraje, wtedy ten punkt zamieniamy na krawędź:
    \begin{center}\begin{tikzpicture}
      \fill [pattern={Dots[radius=1pt, distance=6pt, xshift=0.5mm]}, pattern color=pink!80!blue] (-1.5, -1.5) rectangle ++(1.5, 1.5);
      \fill [pattern={Stars[points=4, radius=2pt, distance=6pt, xshift=-1mm]}, pattern color=orange!40!red] (0, -1.5) rectangle ++(1.5, 1.5);
      \fill [pattern={Hatch[angle=45, distance=4pt]}, pattern color=green!70!orange] (-1.5, -3) rectangle ++(1.5, 1.5);
      \fill [pattern={Lines[distance=4pt, angle=45]}] (0, -3) rectangle ++(1.5, 1.5);

      \fill[opacity=0.6, white] (-1.5,-3) rectangle ++(3, 3);


      \draw[thick] (0, 0)--(0, -3);
      \draw[thick] (-1.5, -1.5)--(1.5, -1.5);


      \fill [pattern={Dots[radius=1pt, distance=6pt, xshift=0.2mm]}, pattern color=pink!80!blue] (4.7, 0)--(4.7, -1.3)--(4.3, -1.7)--(3, -1.7)--(3, 0)--cycle;
      \fill [pattern={Stars[points=4, radius=2pt, distance=6pt, xshift=0mm]}, pattern color=orange!40!red] (4.7, 0)--(4.7, -1.3)--(6, -1.3)--(6, 0)--cycle;
      \fill [pattern={Hatch[angle=45, distance=4pt]}, pattern color=green!70!orange] (3, -1.7)--(4.3, -1.7)--(4.3, -3)--(3, -3)--cycle;
      \fill [pattern={Lines[distance=4pt, angle=45]}] (4.3, -3)--(4.3, -1.7)--(4.7, -1.3)--(6, -1.3)--(6, -3)--cycle;

      \fill[opacity=0.6, white] (3, 0) rectangle (6, -3);

      \draw[thick] (4.7, 0)--(4.7, -1.3)--(6, -1.3);
      \draw[thick] (3, -1.7)--(4.3, -1.7)--(4.3, -3);
      \draw[thick] (4.3, -1.7)--(4.7, -1.3);

      \draw[style={decorate, decoration=snake}, thick] (1.7, -1.5)--(2.8, -1.5);
      \draw[thick, ->] (2.86, -1.5)--(2.88, -1.5);

      \coordinate (A1) at (-0.75, -0.75);
      \coordinate (A2) at (0.75, -.75);
      \coordinate (A3) at (0.75, -2.25);
      \coordinate (A4) at (-0.75, -2.25);

    \fill (A1) circle (2pt);
  \fill (A2) circle (2pt);
\fill(A3) circle (2pt);
\fill(A4) circle (2pt);
      
      \draw[very thick, blue](A1)--(A2)--(A3)--(A4)--(A1);

      \coordinate(B1) at (4, -0.75);
      \coordinate(B2) at (5.5, -0.75);
      \coordinate(B3) at (5, -2.25);
      \coordinate(B4) at (3.5, -2.25);

      \draw[very thick, blue] (B1)--(B2)--(B3)--(B4)--(B1);
      \draw[very thick, blue] (B1)--(B3);

    \fill (B1) circle (2pt);
  \fill (B2) circle (2pt);
\fill(B3) circle (2pt);
\fill(B4) circle (2pt);
    \end{tikzpicture}\end{center}
    W ten sposób z grafu $C_4$ dostajemy graf mający trójkąty jako ściany.

    W przypadku, gdy widzimy na mapie Andorę, to ignorujemy jedną granicę między Francją i Hiszpanią przy rysowaniu grafu dualnego: \def\r{1.5}
    \begin{center}\begin{tikzpicture}
      \fill[pattern={Lines[distance=4pt, angle=80]}, pattern color=orange] (-5, 3) rectangle (0, -3);
      \fill[pattern={Hatch[angle=45, distance=4pt]}, pattern color=green] (0, -3) rectangle (5, 3);
      \fill[white] (0,0) circle (\r);
      \fill[pattern={Dots[radius=1pt, distance=6pt, angle=45]}, pattern color=pink] (0,0) circle (\r);

      \draw[thick] (0,0) circle (\r);

      \draw[thick] (0, 3)--(0, \r);
      \draw[thick] (0, -\r)--(0, -3);

      \coordinate (A) at (-3.5, 2);
      \coordinate (B) at (3.5, 2);
      \coordinate (C) at (0, 0);

      \draw[very thick, blue] (A)--(C)--(B)--(A);

      \fill (A) circle (2pt);
      \fill (B) circle (2pt);
      \fill (C) circle (2pt);
    \end{tikzpicture}\end{center}
    Unikamy w ten sposób krawędzi wielokrotnych. % i dostajemy miejsce na dwuwymiarowy sympleks ($\triangle$).


    Niestety, nie każdy graf pochodzący od takich map na powierzchni da się rozszerzyć do triangulacji. Rozważmy na przykład mapę, do której graf dualny to $K_6$ na butelce Kleina:
    \begin{center}\begin{tikzpicture}[
line cap=round,
marr/.style={
  decoration={
    markings,
    mark=at position 0.5 with {#1}
    },
  postaction={decorate}
}
] 
    \begin{scope}[shift={(-5, 0)}]

    \fill[green] (-2, 2) rectangle (0,0);
    \fill[yellow] (-2, 0) rectangle (0, -2); 
    \fill[blue] (2, 2)--(2, 1)--(0,0)--(0, 2)--cycle;
    \fill[purple] (0, 0)--(2, 1)--(2, -2)--(-0.5, -2)--(0,0);
    \fill[pink] (2, 1)--(2, -1)--(1, -1)--(0,0)--(1, 1)--cycle;

    \fill[orange] (0,0) circle (0.7);

    \foreach \i in {1,...,5}
    \fill (\i*360/5:1.5) coordinate (n\i) circle(2 pt);
      %\ifnum \i>1 foreach \j in {\i,...,1}{(n\i) edge (n\j)} \fi;
    \fill (0,0) circle (2pt);
    \foreach \i in {1,...,5} \draw (0, 0)--(n\i);

    \draw[black, thick] (n1)--(0.1, 2);
    \draw[black, thick] (-0.1, -2)--(n4);

    \draw[black, thick] (n1)--(0.7, 2);
    \draw[black, thick] (-0.7, -2)--(n3);

    %\draw[blue, thick] (n1)--(2, 1.15);
    %\draw[blue, thick] (-2, 1.15)--(n2);
    
    \draw[black, thick] (n2)--(-2, 0.45);
    \draw[black, thick] (2, 0.45)--(n5);

    \draw[black, thick] (n5)--(2, -0.45);
    \draw[black, thick] (-2, -0.45)--(n3);

    \draw[black, thick] (n4)--(0.7, -2);
    \draw[black, thick] (-0.7, 2)--(n2);

    \draw (n1)--(n2)--(n3)--(n4)--(n5)--(n1);
  \fill[opacity=0.7, white] (-2, 2) rectangle (2, -2);

\path
  coordinate (n1) at (-2,-2)
  coordinate (n2) at (-2,2)
  coordinate (n3) at (2,-2)
  coordinate (n4) at (2,2);
\foreach \from/\to in {n1/n3,n4/n2}
    \draw[marr=\Singlearrow] (\from) -- (\to);
\foreach \from/\to in {n1/n2,n3/n4}
    \draw[marr=\Doublearrow] (\from) -- (\to);

\end{scope}



\path
  coordinate (n1) at (-2,-2)
  coordinate (n2) at (-2,2)
  coordinate (n3) at (2,-2)
  coordinate (n4) at (2,2);
\foreach \from/\to in {n1/n3,n4/n2}
    \draw[marr=\Singlearrow] (\from) -- (\to);
\foreach \from/\to in {n1/n2,n3/n4}
    \draw[marr=\Doublearrow] (\from) -- (\to);

    \foreach \i in {1,...,5}
    \fill (\i*360/5:1.5) coordinate (n\i) circle(2 pt);
      %\ifnum \i>1 foreach \j in {\i,...,1}{(n\i) edge (n\j)} \fi;
    \fill (0,0) circle (2pt);
    \foreach \i in {1,...,5} \draw (0, 0)--(n\i);

    \fill[pattern={Dots[angle=45, radius=1pt, distance=6pt]}, pattern color=orange] (n4)--(n5)--(2, -0.45)--(2, -2)--(0.7, -2)--cycle;
    \fill[pattern={Dots[angle=45, radius=1pt, distance=6pt]}, pattern color=orange] (n3)--(-0.7, -2)--(-2, -2)--(-2, -0.45)--cycle;
    \fill[pattern={Dots[angle=45, radius=1pt, distance=6pt]}, pattern color=orange] (n1)--(n5)--(2, 0.45)--(2, 2)--(0.7, 2)--cycle;
    \fill[pattern={Dots[angle=45, radius=1pt, distance=6pt]}, pattern color=orange] (n2)--(-2, 0.45)--(-2, 2)--(-0.7, 2)--(n2)--cycle;

    \draw[blue, thick] (n1)--(0.1, 2);
    \draw[blue, thick] (-0.1, -2)--(n4);

    \draw[blue, thick] (n1)--(0.7, 2);
    \draw[blue, thick] (-0.7, -2)--(n3);

    %\draw[blue, thick] (n1)--(2, 1.15);
    %\draw[blue, thick] (-2, 1.15)--(n2);
    
    \draw[blue, thick] (n2)--(-2, 0.45);
    \draw[blue, thick] (2, 0.45)--(n5);

    \draw[blue, thick] (n5)--(2, -0.45);
    \draw[blue, thick] (-2, -0.45)--(n3);

    \draw[blue, thick] (n4)--(0.7, -2);
    \draw[blue, thick] (-0.7, 2)--(n2);

    \draw (n1)--(n2)--(n3)--(n4)--(n5)--(n1);

    %\node at (n1) {$1$};

    \end{tikzpicture}\end{center}
    Obszar zacieniowany kółeczkami po prawej stronie rysunku zawiera $5$ wierzchołków zamiast $3$ zwyczajowo obecnych w $\triangle$. Nie możemy tego pięciokąta przekroić by otrzymać trójkąty, bo $K_6$ jest już grafem pełnym i takie działanie dałoby wielokrotną krawędź.


  \item 

    \textbf{Torus} $\mathbf{T^2}$\dotfill

    Zacznijmy od tego, że $7$ kolorów na torusie jest koniecznych. Wynika to z faktu, że triangulacja torusa o minimalnej ilości wierzchołków to $K_7$.

    Po pierwsze, minimalna ilość wierzchołków w triangulacji na torusie to $7$. Ponieważ torus jest rozmaitością $2$ wymiarową, to
    $$2E=3T$$
    $$\frac{2}{3}E=T$$
    z formuły Gaussa-Bonnet wiemy, że
    $$0=V-E+F=V-E+\frac{2}{3}E=V-\frac{1}{3}E.\quad (\star)$$
    Górne szacowanie na ilość wierzchołków to
    $$E\leq \binom{V}{2}=\frac{V(V-1)}{2},$$
    co po podstawieniu do $(\star)$ daje
    $$0=V-\frac{1}{3}E\geq V-\frac{V(V-1)}{6}=\frac{7V-V^2}{6}=\frac{V(7-V)}{6}$$
    co implikuje, że $V\geq 7$.

    \begin{center}\begin{tikzpicture}[
line cap=round,
marr/.style={
  decoration={
    markings,
    mark=at position 0.5 with {#1}
    },
  postaction={decorate}
}
] 
\path
  coordinate (n1) at (-2,-2)
  coordinate (n2) at (-2,2)
  coordinate (n3) at (2,-2)
  coordinate (n4) at (2,2);
\foreach \from/\to in {n1/n3,n2/n4}
    \draw[marr=\Singlearrow] (\from) -- (\to);
\foreach \from/\to in {n1/n2,n3/n4}
    \draw[marr=\Doublearrow] (\from) -- (\to);


\coordinate (A1) at (-2, -2);
\coordinate (A2) at (-2, 2);
\coordinate (A3) at  (2, -2);
\coordinate (A4) at (2,2);

\coordinate (B1) at (-2, {2-4/3});
\coordinate (B2) at (2, {2-4/3});

\coordinate (C1) at (-2, {-2+4/3});
\coordinate (C2) at (2, {-2+4/3});

\coordinate (D1) at ({2-4/3}, 2);
\coordinate (D2) at ({2-4/3}, -2);

\coordinate (E1) at ({-2+4/3}, 2);
\coordinate (E2) at ({-2+4/3}, -2);

\coordinate (F) at ({2-4/3}, {2-4/3});
\coordinate (G) at ({-2+4/3}, {-2+4/3});

\draw (A1)--(G)--(F)--(A4);
\draw (D1)--(F)--(B2);
\draw (C1)--(G)--(E2);

\draw (C2)--(D2)--(G)--(B2);
\draw (D2)--(B2);

\draw (F)--(C1)--(E1)--(B1);
\draw (F)--(E1);

    \fill[green] (-2, -2) circle (3pt);
    \fill[green] (-2, 2) circle (3pt);
    \fill[green] (2, -2) circle (3pt);
    \fill[green] (2,2) circle (3pt);

    \fill[orange] (-2, {2-4/3}) circle (3pt);
    \fill[blue] (-2, {-2+4/3}) circle (3pt);

    \fill[orange] (2, {2-4/3}) circle (3pt);
    \fill[blue] (2, {-2+4/3}) circle (3pt);

    \fill[pink] ({2-4/3}, 2) circle (3pt);
    \fill ({-2+4/3}, 2) circle (3pt);

    \fill[pink] ({2-4/3}, -2) circle (3pt);
    \fill ({-2+4/3}, -2) circle (3pt);

    \fill[yellow] ({2-4/3}, {2-4/3}) circle (3pt);
    \fill[purple] ({-2+4/3}, {-2+4/3}) circle (3pt);

    \end{tikzpicture}\end{center}

    Pokażemy teraz, przy pomocy indukcji po ilości wierzchołków, że dla każdego grafu $G$ na torusie wystarczy $7$ kolorów, by go pomalować. 

    Przypadek bazowy, to znaczy $|G|\leq 7$, jest dość oczywisty. Załóżmy teraz, że każdy graf o co najwyżej $n$ wierzchołkach potrafimy pokolorować $7$ kolorami. Niech $G$ będzie grafem na torusie, który ma $(n+1)$ wierzchołków. Rozważamy przypadki:
    \begin{enumerate}[label=\arabic*.]
      \item Istnieje wierzchołek $v\in G$ taki, że $\deg(v)\leq6$.

        Możemy wtedy wierzchołek $v$ wyjąć, tzn. rozważyć graf $G'=G\setminus v$ w którym ze zbioru wierzchołków usunięty został $v$, a ze zbioru krawędzi usunięto wszystkie krawędzie $e$ takie, że $e\cap v\neq \emptyset$.

        Na mocy założenia indukcyjnego graf $G'$ możemy pokolorować $7$ kolorami. Sąsiedzi wierzchołka $v$, jako że było ich $6$ sztuk, korzystają z maksymalnie $6$ kolorów. Możemy więc wybrać kolor, który nie jest przez nich użyty i pomalować nim $v$.

      \item Jeśli wszystkie wierzchołki mają stopień co najmniej $7$, to wtedy mamy 
        $$E=\frac{1}{2}\sum_{v\in V}\deg(v)\geq \frac{7V}{2}\implies \frac{2}{7}E\geq V$$ 
        krawędzi. 

        Graf $G$ niekoniecznie jest triangulacją, ale na pewno nie zawiera przecinających się krawędzi. Możemy więc nieco zmodyfikować to, co wiemy o zależności między liczbą krawędzi a liczbą ścian. Jesteśmy na rozmaitości dwuwymiarowej, więc jedna krawędź trafia do dwóch ścian. Każda ściana z kolei ma co najmniej $3$ krawędzie. Dostajemy więc zależność
        $$2E\geq 3T\implies \frac{2}{3}E\geq T.$$

        Charakterystyka Eulera torusa wynosi $0$, więc możemy użyć formuły Gaussa-Bonneta
        \begin{align*}
          0&=V-E+T\leq V-E+\frac{2}{3}E=\\ 
                 &=V-\frac{1}{3}E\leq\frac{2}{7}E-\frac{1}{3}E=\\ 
                 &=\frac{6-7}{21}E=-\frac{E}{21}
        \end{align*}
        z tego wynika, że
        $$0\geq {E}$$
        co daje sprzeczność z $E>0$. W takim razie w grafie narysowanym na torusie zawsze znajdziemy wierzchołek stopnia $\leq 6$.
    \end{enumerate}

    \textbf{Płaszczyzna rzutowa $\boldsymbol{\R}\mathbf{ P^2}$} \dotfill

    Zaczniemy znowu od pokazania, że istnieje na $\R P^2$ graf, który potrzebuje $6$ kolorów do bycia pomalowanym.

    Triangulacja $\R P^2$ o najmniejszej liczbie wierzchołków to $K_6$. Wnioskujemy to z formuły Gaussa-Bonnet uzupełnionej o fakt, że $\R P^2$ jest rozmaitością wymiaru $2$
    $$1=V-E+T=V-\frac{1}{3}E$$
    dokładamy jeszcze górne ograniczenie na liczbę krawędzi, czyli $E\leq\binom{V}{2}$, by dostać
    $$1=V-\frac{1}{3}E\geq V-\frac{V(V-1)}{6}=\frac{V(7-V)}{6}$$
    $$0\leq \frac{6+V(V-7)}{6}$$
    $V=6$ jest najmniejszym dodatnim rozwiązaniem tej nierówności.

    Dla $V=6$ wymagamy $E=15$, czyli $6$-wierzchołkowa triangulacja $\R P^2$ jest grafem $K_6$:
    %\def{\v1}{pink}\def{\v2}{green}\def{\v3}{blue}\def{\v4}{pink}\def{\v5}{green}\def{\v6}{blue}
    \begin{center}\begin{tikzpicture}
      \draw[->, thick] (-2, 0) arc (180:0:2);
      \draw[->, thick] (2, 0) arc (0:-180:2);

      \foreach \c [count=\i] in {orange, green, yellow, orange, green, yellow} {
        \fill[\c] (\i*360/6+360/12:2) circle (2pt); 
        \coordinate (n\i) at (\i*360/6+360/12:2);
      }

      \foreach \i in {1,...,3} {
        \fill (\i*360/3 + 30:0.7) circle (2pt);
        \coordinate (m\i) at (\i * 120 + 30:0.7);
      }

      \draw (m1)--(m2)--(m3)--(m1);
      \draw (m1)--(n1)--(m3);
      \draw (m1)--(n3)--(m2);
      \draw (m2)--(n5)--(m3);
      \draw (m1)--(n2);
      \draw (m2)--(n4);
      \draw (m3)--(n6);

      \foreach \c [count=\i] in {orange, green, yellow, orange, green, yellow} 
        \fill[\c] (\i*360/6+360/12:2) circle (3pt); 

      \foreach \c [count=\i] in {pink!70!purple, blue, black} 
      \fill[\c] (\i*360/3 + 30:0.7) circle (3pt);
    \end{tikzpicture}\end{center}

    Załóżmy teraz, że wszystkie grafy o co najwyżej $n$ wierzchołkach narysowane na $\R P^2$ potrafimy pomalować używając $6$ kolorów. Niech $G$ będzie grafem o $(n+1)$ wierzchołkach. 
    \begin{enumerate}[label=\arabic*.]
      \item Jeśli istnieje wierzchołek $v\in G$ taki, że $\deg(v)\leq 5$, to podobnie jak w przypadku torusa, możemy ten wierzchołek wyjąć, pokolorować graf $G'=G\setminus v$ i znajdziemy dla $v$ kolor niewykorzystywany przez jego sąsiadów.
      \item W tym przypadku zakładamy, że wszystkie wierzchołki $v\in G$ mają stopień $\deg(v)\geq 6$. Tak jak i wcześniej, mamy
        $$2E\geq 3T$$
        $$2E\geq 6V.$$
        Płaszczyzna rzutowa ma charakterystykę Eulera $1$, w takim razie
        $$1=V-E+T$$
        $$3=3V-3E+3T\leq E-3E+2E=0$$
        co jest sprzecznością. W takim razie w grafie narysowanym na $\R P^2$ zawsze znajdziemy wierzchołek $\deg(v)\leq 5$ i wykonamy kroki z pierwszego punktu.
    \end{enumerate}
%    Dalej mamy rozmaitość $2$ wymiarową, więc $2E=3T$. Z formuły Gaussa-Bonneta dostajemy
%    $$1=\sum_{v\in V}(1-\frac{e_v}{v})=V-\frac{1}{3}E.$$
%
%    Mapy mające $\leq 6$ krajów nie są problemem, więc weźmy mapę o $(n+1)$ krajach. Znowu, robimy z niej graf dualny, który musi być planarny. Możemy więc odrysować krawędzie tak, by otrzymać triangulację. Wiemy, że w grafie zachodzi
%    $$E=\frac{1}{2}\sum_{v\in V}deg(v).$$
%    Co, jeśli nasz $(n+1)$ wierzchołkowy graf ma tylko wierzchołki stopnia $6$ lub więcej? Wtedy
%    $$E\geq \frac{1}{2}\sum_{v\in V}6=\frac{1}{2}\cdot 6V=3V.$$
%    Wracając z tym szacowaniem do formuły G-B dostalibyśmy
%    $$1=V-\frac{1}{3}E\leq V-\frac{1}{3}\cdot 3V=V-V=0$$
%    co jest zdecydowaną sprzecznością. Stąd musi istnieć wierzchołek stopnia $5$ lub mniej i możemy go wyciąć, pomalować resztę i wybrać mu kolor różny od koloru użytego przez co najwyżej $5$ jego sąsiadów.
%
    \textbf{Sfera $\mathbf{S^2}$} \dotfill

    W przypadku sfery próżno szukać grafu, którego nie pokolorujemy $4$ kolorami - prawdziwe jest twierdzenie o $4$ barwach.

    Oczywiście grafy, które mają nie więcej niż $5$ wierzchołków pokolorujemy bez problemu. Załóżmy więc, że każdy graf o co najwyżej $n$ wierzchołkach możemy pokolorować i niech $G$ będzie $(n+1)$-wierzchołkowym grafem.

    \begin{enumerate}[label=\arabic*]
      \item Jeśli znajdziemy wierzchołek stopnia $\leq 4$ lub mniej to robimy to co w przypadku torusa i płaszczyzny rzutowej.
      \item Jeśli istnieje wierzchołek $v\in G$ stopnia $5$.
        \begin{enumerate}
          \item 
        Jeśli $v$ ma $2$ sąsiadów, którzy nie są ze sobą połączeni, wtedy po wyjęciu $v$ możemy złączyć ich w jeden wierzchołek:
        \begin{center}\begin{tikzpicture}
        \foreach \i in {1,...,5} {
          \fill (\i*72+18:1.5) circle (3pt);
          \draw (0,0)--(\i*72+18:1.5);
        \draw (\i*72+18:1.5)--(\i*72+18:2);
        \draw (\i*72+18:1.5)--(\i*72+10:2);
        \draw (\i*72+18:1.5)--(\i*72+26:2);
        }

        \fill[orange] (18:1.5) circle (3.01pt);
        \fill[orange] (18-2*72:1.5) circle (3.01pt);


          \fill[green] (0,0) circle (3pt);
          \node[green] at (0.3, -0.1) {$v$};

          \draw[style={decorate, decoration=snake}, ->] (1.8, 0)--(2.7, 0);

          \begin{scope}[shift={(4.5, 0)}]
            \foreach \i in {1,...,5} {
              \fill (\i*72+18:1.5) circle (3pt);
              \draw[dashed] (0,0)--(\i*72+18:1.5);
        \draw (\i*72+18:1.5)--(\i*72+18:2);
        \draw (\i*72+18:1.5)--(\i*72+10:2);
        \draw (\i*72+18:1.5)--(\i*72+26:2);
            }
            \draw[thick] (18:1.5)--(18-2*72:1.5);
          \end{scope}

          \draw[style={decorate, decoration=snake}, ->] (6.3, 0)--(7.2, 0);
          
          \begin{scope}[shift={(9, 0)}]
            \foreach \i in {1,...,2} {
              \fill (\i*72+18:1.5) circle (3pt);
              \draw[dashed] (0,0)--(\i*72+18:1.5);
        \draw (\i*72+18:1.5)--(\i*72+18:2);
        \draw (\i*72+18:1.5)--(\i*72+10:2);
        \draw (\i*72+18:1.5)--(\i*72+26:2);
            }
          \fill (18-72:1.5) circle (3pt);
          \draw [dashed] (18-2*72:1.5)--(0,0)--(18:1.5);
          \draw[dashed] (18:1.5) circle (3pt);
          \draw[dashed] (18-2*72:1.5) circle (3pt);
            \draw[thick, dotted] (18:1.5)--(18-2*72:1.5);
          %\fill(18-72:0.45) circle (3pt);

          \begin{scope}[shift={(18-72:0.45)}]
          \fill (0,0) circle (3pt);
          \draw (0,0)--(72-35:0.5);
          \draw (0,0)--(104-35:0.5);
          \draw (0,0)--(40-35:0.5);

          \draw (0,0)--(72-35+180:0.5);
          \draw (0,0)--(104-35+180:0.5);
          \draw (0,0)--(40-35+180:0.5);
          \end{scope}

        \draw (18-72:1.5)--(-72+18:2);
        \draw (-72+18:1.5)--(-72+10:2);
        \draw (-72+18:1.5)--(-72+26:2);
          \end{scope}
        \end{tikzpicture}\end{center}
        Po takiej operacji nadal mamy graf narysowany na $S^2$, bo tylko wciągnęliśmy w pustkę pozostawioną przez $v$ krawędzie.

        Taki graf jak na skrajnie prawym rysunku ma $(n-2)$ wierzchołki, więc możemy go pomalować $5$ kolorami. Dwóch sąsiadów $v$ zlepiliśmy w jedno i pokolorowaliśmy tym samym kolorem, więc po ich rozczepieniu nadal mogą mieć ten sam kolor. $v$ miało $5$ sąsiadów, którzy używają nie więcej niż $4$ kolorów - możemy pozostały kolor użyć na $v$.

      \item Jeśli wszyscy sąsiedzi $v$ są ze sobą połączeni, to wtedy $G$ ma podgraf będący $K_5$. 

        Sfera to tak naprawdę punkt z doczepionym dyskiem $D^2$, czyli jeśli wytniemy wokół $v$ dysk zawierający wszystkich jego sąsiadów, to możemy go rozprostować i patrzeć na ten podgraf narysowany na płaszczyźnie. Jednak twierdzenie Kuratowskiego (o grafach) mówi, że na płaszczyźnie możemy narysować graf $\iff$ nie zawiera podgrafu izomorficznego z $K_5$ ani z $K_{3,3}$. Stąd ten przypadek jest niemożliwy.
    \end{enumerate}
  \item Jeśli każdy wierzchołek $v\in G$ ma stopień $>5$, to dostaniemy sprzeczność z formułą Gaussa-Bonnet. Ilość krawędzi szacujemy z jeden strony przez stopień wierzchołków:
    $$2E=\sum_{v\in V}\deg(v)\geq\sum_{v\in V}6=6V$$
    $$E\geq 3V$$
    a z drugiej przez fakt, że każda krawędź leży w dwóch ścianach, a każda ściana ma co najmniej $3$ krawędzie:
    $$2E\geq 3T.$$
    Wstawiając to do formuły Gaussa-Bonnet z pamięcią, że $\chi(S^2)=2$, dostajemy
    $$2=V-E+T$$
    $$6=3V-3E+3T\leq E-3E+2E=0$$
    czyli sprzeczność.
    \end{enumerate}

    \textbf{Butelka Kleina} \dotfill

  Rozważmy graf $G$ na butelce Kleina. Jeśli $|G|\leq 6$, to bez problemy pokolorujemy go $6$ kolorami (fakt, że potrzebujemy co najmniej $6$ kolorów wynika z rysunku w ptk (a) zadania). 

  Niech więc $G$ będzie grafem na butelce Kleina. 
  \begin{enumerate}[label=\arabic*]
    \item Jeśli istnieje wierzchołek $v\in G$ taki, że $\deg(v)\leq 5$, to możemy ten wierzchołek wyjąć, pokolorować $6$ kolorami graf $G\setminus \{v\}$, wierzchołek $v$ pomalować kolorem, który nie pojawia się wśród jego $\leq 5$ sąsiadów i włożymy go z powrotem do $G$ nie psując kolorowania.

    \item Zadanie komplikuje się natomiast, jeśli $G$ jest $6$-regularny. Wybierzmy wierzchołek $v\in G$. Ma on $6$ sąsiadów, czyli jest wierzchołkiem $6$ trójkątów w triangulacji którą stał się $G$:
  \begin{center}\begin{tikzpicture}
    \draw (-1.5, 0)--({-1.5*cos(60)}, {1.5*sin(60)})--(0,0)--({1.5*cos(60)}, {1.5*sin(60)})--(1.5, 0)--(0,0)--({1.5*cos(60)}, {-1.5*sin(60)})--({-1.5*cos(60)}, {-1.5*sin(60)})--(0,0)--cycle;
    \draw ({-1.5*cos(60)}, {1.5*sin(60)})--({1.5*cos(60)}, {1.5*sin(60)});
    \draw (1.5, 0)--({1.5*cos(60)}, {-1.5*sin(60)});
    \draw ({-1.5*cos(60)}, {-1.5*sin(60)})--(-1.5,0);

      %({1.5*cos(60)}, {-1.5*sin(60)})--({-1.5*cos(60)}, {-1.5*sin(60)})--cycle;
\filldraw (1.5, 0) circle (2pt);
\filldraw(-1.5, 0) circle (2pt);
\filldraw({-1.5*cos(60)}, {1.5*sin(60)}) circle (2pt);
\filldraw({1.5*cos(60)}, {1.5*sin(60)}) circle (2pt);
\filldraw({-1.5*cos(60)}, {-1.5*sin(60)}) circle (2pt);
\filldraw({1.5*cos(60)}, {-1.5*sin(60)}) circle (2pt);

\filldraw[orange](0,0) circle(2pt);
\node at (0.3, -0.15) {$\color{orange}v$};
  \end{tikzpicture}\end{center}
  Jeżeli istnieje para sąsiadów, która nie jest ze sobą połączona, to możemy pomalować je na jeden kolor. Wtedy sąsiedzi $v$ wykorzystują tylko $5$ kolorów i ostatni, szósty, pozostaje wolny do kolorowania $v$. 

  Jeśli natomiast wszyscy sąsiedzi $v$ są ze sobą połączeni, to oznacza, że mamy $K_7$ zanurzone w $G$, które z kolei jest narysowane na butelce Kleina. Wiemy, że $K_7$ nie można narysować na butelce Kleina, więc dochodzimy do sprzeczności w tym punkcie.
\item Jeśli w $G$ istnieje wierzchołek stopnia $7$ a pozostałe wierzchołki mają stopień $\deg(v)\geq6$, to możemy użyć Gaussa-Bonnet, by dostać sprzeczność.

  Tak jak wcześniej, szacujemy ilość krawędzi na dwa sposoby:
  $$2E= \sum_{v\in V}\deg(v)\geq 6(V-1)+7=6V+1$$
  $$2E\geq 3T$$
  wstawiamy to do formuły Gaussa-Bonneta i dostajemy
  $$0=V-E+T$$
  $$0=6V-6E+6T\leq (2E-1)-6E+4E=6E-6E-1=-1$$
  $$0\leq -1$$
  co jest sprzecznością.
\end{enumerate}
\end{enumerate}


\textbf{\large\color{orange}Zadanie 6.} Niech $\mathds{K}$ będzie ciałem. Grassmannian $Gr_{\mathds{K}}(k,n)$ to przestrzeń $k$-wymiarowych podprzestrzeni $\mathds{K}^n$. Jeśli $\mathds{K}=\R,\C$ to jest to rozmaitość. Oblicz charakterystykę Eulera Grassmannianu $Gr_{\C}(k,n)$ korzystając z uogólnionej formuły Riemanna-Hurwitza i działania torusa $T^n$ na $\C^n$.

\dotfill 

Każdą $k$-wymiarową podprzestrzeń $\C^n$ możemy utożsamić z macierzą rzutu ortogonalnego na tę podprzestrzeń, która ma rząd $k$. Jeśli $V\in Gr_{\C}(k,n)$, to macierz z nią utożsamioną będziemy oznaczać $A_V$.

\textbf{\color{green}Działanie torusa - definicja}

Działanie torusa $T^n$ na $\C^n$ to mnożenie przez macierze diagonalne z wyrazami długości $1$ na przekątnej. Możemy to działanie przenieść na Grassmannian - torus będzie działał na podprzestrzeń $k$-wymiarową $V$ macierzą $T$:
$$TA_VT^{-1}.$$

\textbf{\color{green}Macierze ortogonalne są hermitowskie}

Macierze rzutu zawsze spełniają $A^2=A$. Jeśli ten rzut jest ortogonalny, to dodatkowo wymagamy, aby $A^T=A$. Macierz rzutu ortogonalnego na zespolonej przestrzeni wektorowej jest jednocześnie macierzą hermitowską, tzn. dla wszystkich $x,y$ spełnia $\langle x, Ay\rangle = \langle Ax, y\rangle$:
\begin{align*}
  \langle Ax, y\rangle -\langle x, Ay\rangle &= \langle Ax, y\rangle -\langle x-Ax+Ax, Ay\rangle=\\ 
  &=\langle Ax, y\rangle -\langle Ax, Ay\rangle + \langle Ax-x, Ay\rangle=\\ 
  &=\langle Ax, (I-A)y\rangle + \langle (A-I)x, Ay\rangle=0,
\end{align*}
bo dla $A$ ortogonalnej $\langle Ax, (I-A)y\rangle =0$. Skoro wiemy już, że $A_V$ jest zawsze macierzą hermitowską, to wiemy, że $\overline{A_V^T}=A_V^T$.

\textbf{\color{green}Działanie torusa jest dobrze określone}

Wiemy też, że każda hermitowska macierz jest diagonalizowalna przez unitarne macierze z rzeczywistymi wartościami własnymi. Dzięki tej własności macierzy hermitowskich działanie torusa zdefiniowane wcześniej jest dobrze określone.

Jeśli $V\in Gr_{\C}(k,n)$ i $A_V=PDP^{-1}$, to po działaniu na $A_V$ macierzą z torusa też jest diagonalizowalna przez macierze unitarne:
$$TA_VT^{-1}=TPDP^{-1}T^{-1}=(TP)D(TP)^{-1},$$
gdyż $T$ również jest macierzą unitarną, czyli $TP$ też takie jest.

\textbf{\color{green}Orbity działania torusa i $\boldsymbol{\chi}$}

Torus $T^n$ jest produktem $n$ okręgów $S^1$. Jeśli działamy okręgiem na przestrzeni, to punkty mają albo trywialny stabilizator, albo trywialną orbitę. Stąd, jeśli $T^n$ działa na $Gr_{\C}(k,n)$ to punkty mają albo trywialną orbitę, albo ich orbita to produkt okręgów.

Jeśli więc określimy odwzorowanie $f:Gr_{\C}(k,n)\to Gr_{\C}(k,n)/T^n$, to wzór Riemanna-Hurwitza podpowiada, że tylko punkty stacjonarne liczą się do charakterystyki Eulera Grassmannianu:
$$\chi(Gr_{\C}(k,n)=\int_{Gr_{\C}(k,n)/T^n}\chi(f^{-1}(X))d\chi(X)$$
będzie zerować się na podzbiorach będących okręgami, bo $\chi(S^1)=0$. Natomiast punkty stałe, których orbity są punktami, będą liczyć się do charakterystyki Eulera po jeden raz. Stąd
$$\chi( Gr_{\C}(k,n))=\#\{\text{punkty stałe działania }T^n\}$$

\textbf{\color{green}Jak wyglądają punkty stałe?}

Punkt stały działania torusa musi spełniać
$$TA_VT^{-1}=A_V$$
$$TA_V=A_VT$$
dla wszystkich możliwych $T$.

{\slshape Macierz $A_V$ jest punktem stałym działania torusa $\iff$ jest macierzą diagonalną o $k$ jedynkach i $(n-k)$ zerach na przekątnej.
}

$\implies$

Załóżmy nie wprost, że istnieje macierz $A_V$ hermitowska, która nie jest diagonalna, ale jest punktem stałym działania $T^n$. Taka macierz musi posiadać niediagonalną macierz unitarną $P$ w diagonalizacji $A_V=PDP^{-1}$. 
Jako, że $A_V$ jest punktem stałym, to dla każdego $T$ przychodzącego z torusa musi zachodzić
$$TPDP^{-1}T^{-1}=PDP^{-1}$$
$$D=(P^{-1}TP)D(P^{-1}TP)^{-1}.$$
Ostatnia równość nie ma szans być spełniona dla wszystkich unitarnych, diagonalnych $T$ - sprzeczność.

$\impliedby$

Wszystkie macierze diagonalne $A_V$ mające $k$ jedynek i $(n-k)$ zer na przekątnej będą punktami stałymi działania torusa. Macierz 
$$TA_V$$ 
to "obcięcie" $T$ do $k$ wyrazów na przekątnej, a domnożenie do tego $T^{-1}$ zwróci $1$ w kolumnach, które zostały zachowane oryginalnie przez $A_V$. Stąd 
$$TA_VT^{-1}=A_V.$$

\textbf{\color{green}Ile jest punktów stałych?}

Jeśli macierz rzutu ortogonalnego jest macierzą diagonalną, to musi ona mieć tylko $1$ i $0$ na przekątnej. Z jej macierzy możemy wywnioskować, że jeśli $i_1,...,i_k$ to numery kolumn, które mają niezerowy wyraz, to przestrzeń na którą jest wykonywany rzut wynosi 
$$Lin\{e_{i_1},...,e_{i_k}\}.$$ 
W takim razie, obrazem wektorów $e_{i_j}$ musi być on sam. Czyli na przekątnej będą $1$ tam, gdzie jest kolumna odpowiadająca wektorom z bazy przestrzeni i $0$ dla pozostałych wektorów.

Stąd, albo przestrzeń z Grassmannianu jest punktem stałym działania, albo nie jest diagonalna. Macierzy z $k$ jedynkami i $(n-k)$ zerami na przekątnej jest $\binom{n}{k}$, czyli charakterystyka Eulera Grassmannianu wynosi
$$\chi(Gr_{\C}(k,n))=\binom{n}{k}.$$


\textbf{\large\color{orange}Zadanie 7.} Grassmannian $Gr_{\C}(k, n)$ ma pewien podział na komórki, który możemy opisać za pomocą szufladek i groszków. Rozważmy $n$ szufladek, w których umieszczać będziemy $k$ groszków, co najwyżej po jednym w danej szufladzie. Takie rozmieszczenie groszków reprezentuje zbiór $k$-wymiarowych podprzestrzeni $\C^n$. Kolejne $l$ szufladek od lewej reprezentuje podprzestrzeń $\C^n$ rozpiętą przez pierwsze $l$ wektorów bazowych $e_1, e_2,..., e_l$, a liczba groszków leżących w $l$ pierwszych $l$ szufladkach to wymiar przekroju $k$-wymiarowej podprzestrzeni z tego zbioru z podprzestrzenią rozpiętą przez $e_1,..., e_l$.
\begin{enumerate}[label=(\alph*)]
  \item Pokaż, że konkretne rozmieszczenie groszków w szufladach reprezentuje przestrzeń $k$-wymiarowych podprzestrzeni $\C^n$ izomorficzna z $\C^m$, gdzie $m$ to liczba przesunięć groszków w lewo o jedną szufladkę dopóki to możliwe.
  \item Przestrzeń $\C^m$ z poprzedniego podpunktu to otwarta komórka wspomnianego rozkładu. Komórka odpowiadająca rozmieszczeniu groszków $A$ zawiera się w
domknięcu komórki odpowiadającej rozmieszczeniu $B$, gdy $A$ można otrzymać z $B$ poprzez kolejne przesunięcia groszków w lewo o jedną szufladkę.
Domknięcie komórki odpowiadającej rozmieszczeniu $A$ nazywamy $(A)$ rozmaitością Schuberta. Policz charakterystykę Eulera rozmaitości Schuberta.
Policz charakterystykę Eulera $Gr_{\C}(k, n)$ zliczając te komórki
\end{enumerate}

\dotfill 

\begin{enumerate}[label=\textbf{(\alph*)}]
  \item Zacznijmy od rozmieszczenia groszków tak, że nie możemy już żadnego przesunąć w lewo. To znaczy, że podprzestrzenie które są kodowane przez to ustawienie groszków kroją się niepusta z podprzestrzenią rozpinaną przez pierwszy wektor bazowy $e_1$, z podprzestrzenią rozpiętą przez dwa pierwsze wektory bazowe $e_1$, $e_2$ i tak dalej. W takim razie, typowa podprzestrzeń reprezentowana przez takie ustawienie jest generowana przez wektory
    \begin{align*}
      &e_1\\ 
      &a_1^1e_1+e_2\\ 
      &...\\ 
      &a_1^ke_1+a_2^ke_2+...+e_k
    \end{align*}
    Interesuje nas przestrzeń rozpinana przez takie wektory, więc w $i$-tym wektorze możemy usunąć część przychodzącą z $j<i$ wektorami. W ten sposób dostaniemy przestrzeń rozpiętą przez $e_1, e_2, ...,e_k$. Nie mamy żadnego parametru, więc jest to izomorficzne z punktem, czyli z $\C^0$.

    Załóżmy, że mamy $k$ groszków umieszczonych odpowiednio w szufladkach o numerze $m_1$, $m_2$, ..., $m_k$. Dzięki pierwszemu groszkowi możemy do naszej $k$-wymiarowej podprzestrzeni $\C^n$ wybrać wektor
    $$a_1^{m_1}e_1+a_2^{m_1}e_2+...+e_{m_1}.$$
    Kolejny groszek, na $m_2\neq m_1$ miejscu pozwoli nam dołożyć wektor
    $$a_1^{m_2}e_1+a_2^{m_2}e_2+...+a_{m_1}^{m_2}+...+e_{m_2}.$$
    Ze współczynników pojawiających się przy kolejnych $e_i$ możemy stworzyć macierz
    $$
    \begin{bmatrix}
      a_1^{m_1} & a_2^{m_1} & ... & 1 & 0 & ... & 0 & ... & 0\\ 
      a_1^{m_2} & a_2^{m_2} & ... & a_{m_1}^{m_2} & ... & ... & 1 & 0 & ...\\ 
      \vdots & \vdots & \vdots & \vdots & \vdots\\ 
      a_1^{m_k} & a_2^{m_k} & ... & a_{m_1}^{m_k} & ... & ... & ... & ... & 1 
    \end{bmatrix}
    $$
    która ma $m_k$ kolumn i $k$ wierszy. Możemy skorzystać z algorytmu eliminacji Gaussa, by dostać w pierwszych $k$ kolumnach kwadratową macierz górnotrójkątną z $1$ na przekątnej. 

    W pierwszym wierszu zostaje nam $(m_1-1)$ zmiennych, w drugim wierszu mamy $(m_2-2)$ nowych zmiennych i tak dalej. Sumarycznie dostajemy
    $$\sum_{i=1}^k(m_i-i)$$
    parametrów w takiej macierzy, co jest równe ilości potencjalnych przesunięć groszków: $i$-ty groszek może przejść przez co najwyżej $(m_i-i)$ szufladek, niekoniecznie za jednym zamachem.

    Pokazaliśmy, że jeśli możemy dokonać $m$ przesunięć groszków, to takie ustawienie możemy zapisać jako przestrzeń liniową przy pomocy $m$ parametrów.

  \item 
\end{enumerate}


\end{document}
