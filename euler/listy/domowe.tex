\documentclass{article}

\usepackage{../../template}
\usepackage{tikz,pgfplots}
\usepackage{tikz-3dplot}

\usepackage{wrapfig}

\title{Charakterystyka Eulera\\{\normalsize Zadanie domowe}}
\author{Weronika Jakimowicz}
\date{}

\begin{document}
\maketitle

\textbf{\large\color{orange}Zadanie 1.} Opisz grupę automorfizmów triangulacji $\R P^2$ o najmniejszej liczbie wierzchołków.

Wiemy, że jeśli $X$ ma triangulację o $V$ wierzchołkach, $E$ krawędziach i $T$ trójkątach, to 
$$\chi(X)=V-E+T$$
a ponieważ $2E=3T$, to możemy podstawić
$$\chi(X)=V-E+\frac{2}{3}E=V-\frac{1}{3}E.$$
Ilość krawędzi szacujemy od góry przez ilość krawędzi w grafie pełnym: $E\leq \binom{V}{2}$ czyli
$$V=\chi(X)+\frac{1}{3}E\leq \chi(X)+\frac{V(V-1)}{6}$$
dla $\R P^2$ dostajemy więc ograniczenie
$$V\leq 1+\frac{V(V-1)}{6}$$
$$6\geq 6V-V^2+V=V(7-V)$$
Powyższa nierówność dla $V=6$ staje się równością. Tak samo dla $V=1$ mamy równość, ale z oczywistego powodu nie ma jednowierzchołkowej triangulacji na $\R P^2$. Pozostałe liczby naturalne z przedziału $(0, 7)$ nie mają szansy spełniać powyższe równanie
\begin{center}
  \begin{tikzpicture}
  \begin{axis}[
    axis x line=center,
    axis y line=center,
    xtick={0, 2, 4, 6, 8},
    ytick={0, 2, 4, 6, 8, 10, 12},
    xlabel={$x$},
    ylabel={$y$},
    xlabel style={below right},
    ylabel style={left},
    xmin=-1.5,
    xmax=8.5,
    ymin=-1.5,
    ymax=13.5]
  \addplot [mark=none,domain=-1:7] {x*(7-x)};
  \addplot [domain=-1:7, dashed] {6};
  \addplot [dashed] coordinates {(6, -0.5) (6, 12)};
  \end{axis}
  \end{tikzpicture}
\end{center}

Z listy 1 wiemy, że $6$ wierzchołkowa triangulacja $\R P^2$ jest jedyna z dokładnością do izomorfizmu, czyli nie musimy się martwić którą triangulacje opisujemy.

Działając podobnie jak przy próbie skonstruowania triangulacji na butelce Klein, zauważmy, że $\R P^2$ to $S^2$ wydzielona przez działanie $x\sim -x$. Żeby dostać $6$ wierzchołkową triangulację na $\R P^2$ możemy więc zacząć od znalezienia $12$ wierzchołkowej triangulacji na $S^2$, którą będzie np. dwunastościan:
\begin{wrapfigure}{l}{7cm}
\tdplotsetmaincoords{70}{110}
\begin{tikzpicture}[tdplot_main_coords]
	% Change the value of the number at {\escala}{##} to scale the figure up or down
	\pgfmathsetmacro{\escala}{2}
	% Coordinates of the vertices
%  \node(1) at (-\escala*1.37638, \escala*0., \escala*0.262866) {1};
%  \node(2) at (\escala*1.37638, \escala*0., -\escala*0.262866) {2};
%  \node(3) at (-\escala*0.425325, -\escala*1.30902, \escala*0.262866) {3};
%  \node(4) at (-\escala*0.425325, \escala*1.30902, \escala*0.262866) {4};
%  \node(5) at (\escala*1.11352, -\escala*0.809017, \escala*0.262866) {5};
%  \node(6) at (\escala*1.11352, \escala*0.809017, \escala*0.262866) {6};
%  \node(7) at (-\escala*0.262866, -\escala*0.809017, \escala*1.11352) {7};
%  \node(8) at (-\escala*0.262866, \escala*0.809017, \escala*1.11352) {8};
%  \node(9) at (-\escala*0.688191, -\escala*0.5, -\escala*1.11352) {9};
%  \node(10) at (-\escala*0.688191, \escala*0.5, -\escala*1.11352) {10};
%	%
%  \node(11) at (\escala*0.688191, -\escala*0.5, \escala*1.11352) {11};
%  \node(12) at (\escala*0.688191, \escala*0.5, \escala*1.11352) {12};
%  \node(13) at (\escala*0.850651, \escala*0., -\escala*1.11352) {13};
%  \node(14) at (-\escala*1.11352, -\escala*0.809017, -\escala*0.262866) {14};
%  \node(15) at (-\escala*1.11352, \escala*0.809017, -\escala*0.262866) {15};
%  \node(16) at (-\escala*0.850651, \escala*0., \escala*1.11352) {16};
%  \node(17) at (\escala*0.262866, -\escala*0.809017, -\escala*1.11352) {17};
%  \node(18) at (\escala*0.262866, \escala*0.809017, -\escala*1.11352) {18};
%  \node(19) at (\escala*0.425325, -\escala*1.30902, -\escala*0.262866) {19};
%  \node(20) at (\escala*0.425325, \escala*1.30902, -\escala*0.262866) {20};
	% Faces of the dodecahedron
 
	\coordinate(1) at (-\escala*1.37638, \escala*0., \escala*0.262866);
	\coordinate(2) at (\escala*1.37638, \escala*0., -\escala*0.262866);
	\coordinate(3) at (-\escala*0.425325, -\escala*1.30902, \escala*0.262866);
	\coordinate(4) at (-\escala*0.425325, \escala*1.30902, \escala*0.262866);
	\coordinate(5) at (\escala*1.11352, -\escala*0.809017, \escala*0.262866);
	\coordinate(6) at (\escala*1.11352, \escala*0.809017, \escala*0.262866);
	\coordinate(7) at (-\escala*0.262866, -\escala*0.809017, \escala*1.11352);
	\coordinate(8) at (-\escala*0.262866, \escala*0.809017, \escala*1.11352);
	\coordinate(9) at (-\escala*0.688191, -\escala*0.5, -\escala*1.11352);
	\coordinate(10) at (-\escala*0.688191, \escala*0.5, -\escala*1.11352);
	%
	\coordinate(11) at (\escala*0.688191, -\escala*0.5, \escala*1.11352);
	\coordinate(12) at (\escala*0.688191, \escala*0.5, \escala*1.11352);
	\coordinate(13) at (\escala*0.850651, \escala*0., -\escala*1.11352);
	\coordinate(14) at (-\escala*1.11352, -\escala*0.809017, -\escala*0.262866);
	\coordinate(15) at (-\escala*1.11352, \escala*0.809017, -\escala*0.262866);
	\coordinate(16) at (-\escala*0.850651, \escala*0., \escala*1.11352);
	\coordinate(17) at (\escala*0.262866, -\escala*0.809017, -\escala*1.11352);
	\coordinate(18) at (\escala*0.262866, \escala*0.809017, -\escala*1.11352);
	\coordinate(19) at (\escala*0.425325, -\escala*1.30902, -\escala*0.262866);
	\coordinate(20) at (\escala*0.425325, \escala*1.30902, -\escala*0.262866);

%	\draw[thick,cyan, opacity=0.75]  (2) -- (6) -- (12) -- (11) -- (5) -- (2);
%	\draw[thick,cyan, opacity=0.75]  (5) -- (11) -- (7) -- (3) -- (19) -- (5);
%	\draw[thick,cyan, opacity=0.75]  (11) -- (12) -- (8) -- (16) -- (7) -- (11);
%	\draw[thick,cyan, opacity=0.75]  (12) -- (6) -- (20) -- (4) -- (8) -- (12);
%	%
%	\draw[thick,cyan, opacity=0.75]  (6) -- (2) -- (13) -- (18) -- (20) -- (6);
%	\draw[thick,cyan, opacity=0.75]  (2) -- (5) -- (19) -- (17) -- (13) -- (2);
%	\draw[thick,cyan, opacity=0.75]  (4) -- (20) -- (18) -- (10) -- (15) -- (4);
%	\draw[thick,cyan, opacity=0.75]  (18) -- (13) -- (17) -- (9) -- (10) -- (18);
%	\draw[thick,cyan, opacity=0.75]  (17) -- (19) -- (3) -- (14) -- (9) -- (17);
%	%
%	\draw[thick,cyan, opacity=0.75]  (3) -- (7) -- (16) -- (1) -- (14) -- (3);
%	\draw[thick,cyan, opacity=0.75]  (16) -- (8) -- (4) -- (15) -- (1) -- (16);
%	\draw[thick,cyan, opacity=0.75]  (15) -- (10) -- (9) -- (14) -- (1) -- (15);
	%	

  \draw[orange, thick] (19)--(7)--(1)--(9)--(19);
  \draw[orange, thick, dashed] (18)--(2)--(12)--(4)--(18);
  \draw[orange, thick] (9)--(18)--(4)--(1);
  \draw[dashed, orange, thick] (7)--(12);
  \draw[dashed, orange, thick] (19)--(2);

  \draw[dashed, thick] (20)--(6)--(2)--(13);
  \draw[dashed, thick] (6)--(12)--(11)--(5)--(2)--(6);
  \draw[dashed, thick] (12)--(8);
  \draw[dashed, thick] (7)--(11);
  \draw[dashed, thick] (5)--(19);
  \draw[thick] (7)--(16)--(1)--(14)--(3)--(7);
  \draw[thick] (16)--(8)--(4)--(15)--(1);
  \draw[thick] (14)--(9)--(10)--(15);
  \draw[thick] (3)--(19)--(17)--(9);
  \draw[thick] (17)--(13)--(18)--(10);
  \draw[thick] (18)--(20)--(4);
\end{tikzpicture}

\end{wrapfigure}

Zaczniemy od uzasadnienia, że $Aut(\text{20-ścian})=A_5\times \Z_2$ (bo Wikipedia nie jest tutaj uważanym źródłem informacji \frownie). 

Oznaczmy $20$-ścian przez $D$. Wiemy, że jeśli $v$ jest dowolnym wierzchołkiem $D$, to $|Orb(v)=20|$, bo wierzchołek może przejść na dowolny inny. Pozostaje znaleźć $|Stab(v)|$. W $Stab(v)$ mamy permutacje wierzchołków przyległych do $v$, czyli permutacje zbioru o $3$ elementach. W takim razie $|Stab(v)|=|S_3|=6$. Daje to $|Aut(D)|=|Orb(v)||Stab(v)|=20\cdot 6=120$. Jak na razie zgadza się.

Zauważmy, że automorfizmem $D$ są zwykłe obroty i rotacje, ale też symetria przez punkt w samym środku (odwzorowanie antypodalne, co jest przyjemne). Ten drugi automorfizm powinien odpowiadać za czynnik $\Z_2$ - to chyba on mówi czy sąsiedzi wierzchołka są ponumerowani zgodnie z ruchem wskazówek zegara czy też przeciwnie. W takim razie wystarczy pokazać, że symetrie i obroty $D$ tworzą grupę $A_5$.

Sztuczką na pokazanie, że symetrie $D$ to $A_5$ jest zauważenie $5$ sześcianów w środku $D$. Sześciany możemy wyciągnąć biorąc dowolną krawędź i łącząc wszystkich sąsiadów końców tej krawędzi. Powtarzając tę samą akcję z krawędzią po przeciwnej stronie $D$ dostajemy drugą ścianę i wystarczy je teraz połączyć. Jeden sześcian możemy otrzymać na $6$ sposobów, a w $D$ jest $30$ krawędzi. Mamy więc $5$ grup krawędzi, które dają ten sam sześcian.

Ponumerujmy sześciany od $1$ do $5$ - możemy teraz je permutować. Najbardziej leniwym sposobem na zauważenie, że grupa uzyskana przez porządne permutacje tych sześcianów to $A_5$ jest podzielenie $|Aut(D)|=120$ przez $2$, które oznacza, że wyrzucamy symetrię względem punktu w środku $D$ (element rzędu $2$). Zostawia to nam $60$ automorfizmów, które będą permutować te sześciany i które powinniśmy móc włożyć w $S_5$. Jedyna (z dokładnością do izomorfizmu) podgrupa $S_5$ o $60$ elementach to $A_5$.

\end{document}
