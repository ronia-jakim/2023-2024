\textbf{\large\color{orange}Zadanie 4.} Niech $X$ będzie przestrzenią topologiczną. Definiujemy przestrzeń konfiguracji $\Conf_n(X)$ jako przestrzeń położeń $n$ różnych punktów w $X$ (zakładamy, że dwa punkty nie mogą leżeć w tym samym miejscu. Definiujemy też przestrzeń konfiguracji $\conf_n(X)$ jako przestrzeń położeń $n$ nierozróżnialnych punktów w $X$, czyli
$$\Conf_n(X)=\{(x_1,...,x_n)\in X^n\;:\;x_i\neq x_j, \text{ gdy }i\neq j\}$$
$$\conf_n(X)=\frac{\Conf_n(X)}{S_n}$$
gdzie $S_n$ działa na $\Conf_n(X)$ przez permutacje współrzędnych.

Oblicz charaketrystykę Eulera $\conf_n(\mathbf{Y})$, gdzie $\mathbf{Y}$ to drzewo o $4$ wierzchołkach, z czego $3$ to liście, dla $n=2,3,4$.

\dotfill

W każdym punkcie $\Conf_n(\mathbf{Y})$ wisi prawie identyczna kopia $\mathbf{Y}$ z tym, że brakuje w niej jednego punktu. Możemy więc napisać funkcję
$$f:\Conf_n(\mathbf{Y})\to \mathbf{Y}$$
$$f(x, y)=x$$
wtedy przeciwobrazem dowolnego punktu w $\mathbf{Y}$ jest $\mathbf{Y}$ bez jednego punktu, oznaczmy tę przestrzeń przez $\mathbf{Y}'$.

Korzystając ze wzoru
