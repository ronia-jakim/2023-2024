\section{Preliminaria}

Zacznijmy od przyjżenia się, czym jest \buff{charakterystyka Eulera}, $\color{orange}\chi(X)$, dla poznanych już przestrzeni $X$:

\begin{itemize}
  \item dla przestrzeni skończonych mamy $\chi(X)=|X|$
  \item jeśli zajmujemy się przestrzenią wektorową nad ciałem $K$, to $\chi(X)=\dim_K(X)$.
\end{itemize}

Poza tym, będziemy przyglądać się kompleksom symplikacyjnym (poniżej) oraz kompleksom łańcuchowym (czyli uogólnieniom przestrzeni wektorowej).

\subsection{Kompleksy symplikacyjne}

\begin{definition}[Kompleks symplikacyjny]
  Rozważmy zbiór (wierzchołków) $V$. Zbiór $\set{K}\subseteq 2^V$ nazywamy \buff{kompleksem symplikacyjnym} na zbiorze $V$, jeśli

  \begin{itemize}
    \item dla każdego $v\in V$ mamy $\{v\}\in \set{K}$
    \item dla każdych $B\subseteq A\subseteq V$ zachodzi $A\in\set{K}\implies B\in\set{K}$
  \end{itemize}
\end{definition}

Będziemy głównie zajmować się $|V|<\infty$. Dla wygody często $v$ będziemy utożsamiać z $\{v\}$.

Rodzinę $\set{L}\subseteq\set{K}$ nazywamy \acc{podkompleksem} kompleksu $\set{K}$, jeśli jest on kompleksem symplikacyjnym na zbiorze wierzchołków $V(\set{L})=\bigcup_{L\in\set{L}}L$.

\begin{definition}[Sympleks]
  Elementy $\sigma\in\set{K}$ oraz podkompleksy zawierające wszystkie niepuste podzbiory $\sigma$ są \buff{sympleksami}. Wymiar sympleksu to $\dim(\sigma)=|\sigma|-1$.
\end{definition}

\begin{example}
  \item $\set{K}=\set{P}(V)$ jest sympleksem $|V|-1$-wymiarowym i oznaczamy jako $\Delta^V$ lub $\Delta^n$ gdy $|V|=n+1$.
  \item \buff{Brzegiem} sympleksu $\Delta^V$ nazywamy podkompleks złożony z właściwych podzbiorów $V$ i oznaczamy go $\partial\Delta^V$. Sympleksy brzegu $\partial\Delta^n$ (lub, w zależności od konwencji, $(n-1)$-wymiarowe sympleksy) nazywamy \acc{ścianami} sympleksu $\Delta^n$.
  \item Podkompleks $\set{K}$ złożony z co najmniej $(k+1)$-elementów nazywamy $k$-szkieletem kompleksu $\set{K}$
  \item Grafy symplikacyjne to $1$-wymiarowe kompleksy będące grafami bez pętli i bez wielokrotnych krawędzi.
  \item \acc{Stożkiem} nad kompleksem $\set{K}=\{\sigma_1,...,\sigma_n\}$ o wierzchołku
\end{example}

