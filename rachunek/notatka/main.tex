\documentclass{article}

\usepackage[back_on]{../../template}

\title{Rachunek prawdopodobieństwa 2R}
\author{Kycia}
\date{}

\begin{document}
\maketitle

\section{Testowa sekcja}
\subsection{Testowa subsekcja}

{\color{blue}Jakiś tak przykładowy} tekst {\color{red}jara jara jara jaaara}. {\color{green}Bardzo dużo} tekstu, {\color{yellow}żeby sobie} sprawdzić {\color{orange}jak czcionka sobie} radzi. {\color{purple}Baaardzo obfity} tekst. Tuptuś. TRALALALALALAL LAL LA LALA ja sobie piszę i tak sobie piszę świetną i wybitną notatkę. i wydaje się, że tutaj o dziwo działa porządnie robienie wyjustowania. Ciekawa sprawa.

\begin{itemize}
  \item Weles jest czarny
  \item Kycia jest gruba
  \item Brzuszek mnie boli
\end{itemize}

\end{document}
