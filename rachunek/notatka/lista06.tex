\subsection{Zadania}

\setcounter{problem}{0}

\begin{problem}
  PÓŹNIEJ
\end{problem}

\begin{problem}
  Niech $\{X_n\}$ będzie martyngałem całkowalnym z kwadratem ($\expected{X_n^2}<\infty$ dla każdego $n\in\N$). Pokaż, że
  $$\expected{(X_n-X_m)^2}{\set{F}_m}=\expected{X_n^2}{\set{F}_m}-X_m^2$$
  {\color{red}Dodatkowe założenia: $n>m$.}
\end{problem}

\begin{solution}
  Po pierwsze, wypadałoby pokazać, że $X_nX_m$ jest całkowalne
  $$\expected{|X_nX_m|}\leq\expected{|X_n|^2}^{1/2}\expected{|X_m|^2}^{1/2}<\infty$$
  W takim razie, wiedząc, że $X_m$ jest zawsze mierzalna względem $\set{F}_m$ wiemy, że $\expected{X_nX_m}{\set{F}_m}=X_m\expected{X_n}{\set{F}_m}$ i teraz jeśli $n>m$ to i $\expected{X_n}{\set{F}_m}=X_m$, co pokazaliśmy już dawno temu na liście. Przechodząc z tą wiedzą do pisania, mamy;
  \begin{align*}
    \expected{(X_n-X_m)^2}{\set{F}_m}&=\expected{X_n^2-2X_nX_m+X_m^2}{\set{F}_m}=\\ 
                                     &=\expected{X_n^2}{\set{F}_m}+\expected{X_m^2}{\set{F}_m}-2X_m\expected{X_n}{\set{F}_m}=\\ 
                                     &=\expected{X_n^2}{\set{F}_m}+X_m^2-2X_m^2=\expected{X_n^2}{\set{F}_m}-X_m^2
  \end{align*}
\end{solution}

\begin{problem}
  Niech $\{Z_n\}$ będzie procesem Galtona-Watsona dla którego $\expected{Z_1}=\mu$ oraz $Var(Z_1)=\sigma^2<\infty$. Rozważmy martyngał $M_n=\mu^{-n}Z_n$.
  \begin{enumerate}[label=(\alph*)]
    \item Pokaż, że
      $$\expected{M_n^2}=1+\sigma^2\sum_{k=2}^{n+1}\mu^{-k}$$
    \item Uzasadnij, że jeśli $\mu>1$, to $M_n$ jest zbieżny w $L^2$
    \item Uzasadnij, że jeśli $\mu<1$, to $M_n$ nie jest zbieżny w $L^2$.
  \end{enumerate}
\end{problem}

\begin{solution}
  Proces Galtona-Watsona pojawił się w rozdziale \ref{procel galtona-watsona}, gdy chcieliśmy obserwować pantofelki rozmnażające się bezpłciowo, niezależnie od siebie. Rozważaliśmy zmienne losowe $Y_{n,k}$ takie oraz ciąg
  $$Z_1=1$$
  $$Z_{n+1}=\sum_{k=1}^{Z_n}Y_{n+1,k}$$
  gdzie $Z_n$ to liczba nowych pantofelków w $n$-tej generacji, a $Y_{n,k}$ to liczba potomstwa w $n$-tej generacji zrodzona przez $k$-tego pantofelka w $n-1$ generacji.

  \begin{enumerate}[label=(\alph*)]
    \item Wiem już, że
      $$\expected{Z_1}=\expected{\sum_{k=1}^{Z_0}Y_{n,k}}=\expected{Y_{1,1}}=\mu$$
      oraz (korzystając z tego co pokazaliśmy na wykładzie):
      $$\expected{Z_{n+1}}=\expected{Y_{n,k}}\expected{Z_n}=\mu\expected{Z_n}=\mu^{n+1}.$$

      Całość pokażemy za pomocą indukcji. Jeśli $n=1$, to mamy
      $$\expected{M_1^2}=\expected{\mu^{-2}Z_1^2}=\mu^{-2}\expected{Z_1^2}=\mu^{-2}[Var(Z_1)+\expected{Z_1}^2]=\sigma^2\mu^{-2}+1$$
      tak jak chcieliśmy.

      Zróbmy teraz krok indukcyjny, czyli $n\implies n+1$. Będziemy korzystać z zadania 2, więc chcemy wyliczyć $M_{n+1}-M_n$
      \begin{align*}
        M_{n+1}-M_n&=\mu^{-n-1}\sum_{k=1}^{Z_{n}}Y_{n,k}-\mu^{-n}Z_n=\mu^{-n-1}\sum_{k=1}^{Z_n}(Y_{n,k}-\mu)
      \end{align*}
      Wstawiając do równości w zadaniu 2 (po scałkowaniu), dostajemy
      \begin{align*}
        \expected{M_{n+1}^2}&=\expected{(M_{n+1}-M_n)^2}+\expected{M_{n}^2}=\\ 
                            &=\expected{\left( \mu^{-n-1}\sum_{k=1}^{Z_n}(Y_{n,k}-\mu) \right)^2} +1+\sigma^2\sum_{k=2}^{n+1}\mu^{-k}=\\ 
                            &=\mu^{-2n-2}\expected{\left(\sum_{k=1}^{Z_n}(Y_{n,k}-\mu)\right)^2}+1+\sigma^2\sum_{k=2}^{n+1}\mu^{-k}=(\star)
      \end{align*}
      Pozostaje pokazać, że $\expected{\left(\sum_{k=1}^{Z_n}(Y_{n,k}-\mu)\right)^2}=\mu^{n}\sigma^2$.

      Rozważmy funkcję 
      \begin{align*}
        h(z)&=Var(\sum_{k=1}^zY_{n+1,k})=\expected{\left(\sum_{k=1}^zY_{n+1,k}-z\mu\right)^2}=\\ 
            &=\sum_{k=1}^zVar(Y_{n+1,k})=zVar(Y_{1,1})=zVar(Z_1)=z\sigma^2
      \end{align*}
      i zauważmy, że wwo
      $$\expected{\left(\sum_{k=1}^{Z_n}(Y_{n+1,k}-\mu)\right)^2}{\set{F}_n}=h(Z_n)=Z_n\sigma^2$$
      czyli całkując obie strony otrzymujemy
      $$\expected{\left(\sum_{k=1}^{Z_n}(Y_{n+1,k}-\mu)\right)^2}=\expected{Z_n\sigma^2}=\mu^n\sigma^2$$
      i to jest tym co chcieliśmy, bo wracając do kroku indukcyjnego
      $$(\star)=\mu^{-n-2}\mu^n \sigma^2+1+\sigma^2\sum_{k=2}^{n+1}\mu^{-k}=1+\sigma^2\sum_{k=2}^{n+2}\mu^{-k}$$
    \item Aby $M_n$ zbiegało w $L^2$, musimy znaleźć funkcję $X$ taką, że $\expected{|M_n-X|^2}\to 0$. Zauważmy, że $Z_n$ zawsze przyjmuje skończone wartości, czyli $M_n=\mu^{-n}Z_n\to 0$, bo $\mu^{-n}\to 0$. W takim razie mamy $X=0$ jest potencjalną granicą $M_n$. Wystarczy wyliczyć
      \begin{align*}
        \expected{|M_n|^2}&=1+\sigma^2\sum_{k=2}^{n+1}\mu^{-k}=1+\sigma^2\sum_{k=0}^{n-1}\mu^{-k-2}=\\ 
                          &=1+\sigma^2\mu^{-2}\sum_{k=0}^{n-1}\mu{-k}=1+\sigma^2\mu^{-2}\sum_{k=0}^{n-1}\mu^{-k}=\\ 
                          &=1+\sigma^2\mu^{-2}\frac{1-\mu^{-n}}{1-\mu^{-1}}=1+\sigma^2\mu^{-1}\frac{1-\mu^{-n}}{\mu-1}\leq 1+\sigma^2\mu^{-1}\frac{1}{\mu-1}<\infty
      \end{align*}
      czyli stosuje się do tego twierdzenie \ref{tw 7.5}, ponieważ spełniony jest warunek $\sup\expected{|M_n|^p}<\infty\iff$ istnieje $M_\infty$ że $M_n\to M_\infty$ w $L^p$ dla $p=2$.
    \item Argument jest prawie taki sam jak wyżej, z tym, że
      $$\expected{|M_n|^2}=1+\sigma^2\mu^{-2}\frac{1-\mu^{-n}}{1-\mu^{-1}}=1+\sigma^2\mu^{-1}\frac{1-\mu^{-n}}{\mu-1}$$
      jest cały czas rosnące, czyli $\sup{|M_n|^p}=\infty$ i wówczas musimy mieć fałsz w warunku 2, czyli istnieniu $M_\infty$ takiego, że $M_n\to M_\infty$ w $L^p$.
  \end{enumerate}
\end{solution}
