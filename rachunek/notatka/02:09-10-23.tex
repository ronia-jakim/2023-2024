\section{09.10.23 Własności WWO}

Na tym wykładzie zajmiemy się dowodzeniem własności wwo, w tym pokażemy jej istnienie i jedyność.

\subsection{Istnienie i jedyność}

\begin{lemma}[WWO jest całkowalna]
  To znaczy, że mając całkowalną zmienną losową $X$ oraz $\sigma$-ciało $\set{G}\subseteq\set{F}$, to zachodzi $\expected{\left|\expected{X}{\set{G}}\right|}<\infty$.
\end{lemma}

\begin{proof}
  Rozważmy zbiór
  $$A=\{\omega\;:\;\expected{X}{\set{G}}(\omega)>0\}=\{\omega\;:\;\expected{X}{\set{G}}\in(0, \infty)\}=\left[\expected{X}{\set{G}}\right]^{-1}((0,\infty))$$
  jako przeciwobraz zbioru $(0,\infty)\in Bor(\R)$ przez funkcję $\set{G}$-mierzalną $\expected{X}{\set{G}}$ wiemy, że $A\in \set{G}$. Ponieważ $\expected{X}{\set{G}}$ jest wwo $X$ pod warunkiem $\set{G}$, to musi warunek (W2): $$0\leq\expected{\expected{X}{\set{G}}\mathds{1}_A}=\expected{X\mathds{1}_A}\leq\expected{|X|\mathds{1}_A}<\infty$$
  bo $X$ jest całkowalna.

  Analogicznie postępujemy dla zbioru $A^c$:
  $$0\leq\expected{-\expected{X}{\set{G}}\mathds{1}_A}=\expected{-X\mathds{1}_{A^c}}\leq\expected{|X|\mathds{1}_{A^c}}<\infty.$$
  
  Zauważmy, że 
  $$\left|
    \expected{X}{\set{G}}
    \right| = \expected{X}{\set{G}}\mathds{1}_A-\expected{X}{\set{G}}\mathds{1}_{A^c}$$
  Dodając obie te nierówności (i korzystając z liniowości wartości oczekiwanej) uzyskujemy
  $$0\leq \expected{\expected{X}{\set{G}}\mathds{1}_A}-\expected{\expected{X}{\set{G}}\mathds{1}_{A^c}}=\expected{\expected{X}{\set{G}}\mathds{1}_A-\expected{X}{\set{G}}\mathds{1}_{A^c}}=\expected{\left|\expected{X}{\set{G}}\right|}<\infty$$
\end{proof}

\begin{lemma}[jedyność p.w.]
  Niech $\set{G}\subseteq{F}$ będzie $\sigma$-ciałem. Jeśli $Y$ i $Y'$ są obie wersjami $\expected{X}{\set{G}}$, to $Y=Y'$ p.w..
\end{lemma}

\begin{proof}
  Ustalmy $\varepsilon>0$ i rozważmy zdarzenie
  $$A_\varepsilon=\{Y-Y'>\varepsilon\}\in \set{G}$$
  które jest $\set{G}$-mierzalne, bo $Y$ i $Y'$ takie są.

  \begin{align*}
    \varepsilon\prob{A_\varepsilon}+\expected{Y'\mathds{1}_{A_\varepsilon}}&=\expected{\varepsilon\mathds{1}_{A_\varepsilon}}+\expected{Y'\mathds{1}_{A_\varepsilon}}=\\
      &=\expected{(\varepsilon+Y')\mathds{1}_{A_\varepsilon}}\leq\\
      &\overset{\star}{\leq}\expected{Y\mathds{1}_{A_\varepsilon}}\overset{(W2)}{=}\expected{X\mathds{1}_{A_\varepsilon}}=\\
      &=\expected{Y'\mathds{1}_{A_\varepsilon}}
  \end{align*}
  gdzie $\star$ wynika z tego, że na zbiorze $A_\varepsilon$ $Y>Y'+\varepsilon$.

  Dostajemy więc, że
  $$\varepsilon\prob{A_\varepsilon}+\expected{Y'\mathds{1}_{A_\varepsilon}}\leq\expected{Y'\mathds{1}_{A_\varepsilon}}$$
  co po przeniesieniu $\expected$ na jedną stronę daje
  $$\varepsilon\prob{A_\varepsilon}\leq0$$
  a ponieważ $\varepsilon>0$, to musi być $\prob{A_\varepsilon}=0$.

  Wówczas 
$$\prob{Y>Y'}=\underbrace{\prob{(\exists\;n)\;Y\geq Y'+\frac{1}{n}}}_{\prob{A_\frac{1}{n}}}=\prob{\bigcup A_\frac{1}{n}}=\lim\prob{A_{\frac{1}{n}}}=0$$
  ponieważ $A_\frac{1}{n}\subseteq A_{\frac{1}{n+1}}$.

  Zamieniając miejscami $Y$ i $Y'$ w dowodzie dostaniemy $\prob{Y'>Y}=0$, czyli obie możliwości są miary zero.
\end{proof}

\begin{theorem}[o istnieniu WWO]\label{istnienie wwo}
  Niech $\set{G}\subseteq\set{F}$ będzie $\sigma$-ciałem, a $X$ jest całkowalną zmienną losową. Istnieje zmienna losowa $Y$ spełniająca oba postulaty wwo $X$ pod warunkiem $\set{G}$.
\end{theorem}

Jest to Twierdzenie \ref{poprawnosc wwo} z poprzedniego wykładu.

Zanim jednak przejdziemy do dowodu \ref{istnienie wwo}, przypomnijmy \acc{twierdzenie Radona-Nikodyma} z teorii miary:
\begin{dygresja}[twierdzenie Radona-Nikodyma]
  Niech $\mu$ i $\nu$ będą $\sigma$-miarami na przestrzeni $(\Omega,\set{G})$ takimi, że $\nu$ jest \acc{absolutnie ciągła} względem $\mu$ [$\nu\ll\mu$], tzn $\mu(A)=0\implies\nu(A)=0$. Wówczas istnieje $\set{G}$-mierzalna funkcja $f:\Omega\to\R$ taka, że
  $$\nu(A)=\int_Af(x)\mu(dx)$$
\end{dygresja}
Funkcję $f$ jak wyżej często oznaczamy $f=\frac{d\nu}{d\mu}$ i nazywamy \acc{pochodną Radona-Nikodyma}.

\begin{proof}
  Wracając do dowodu twierdzenia \ref{istnienie wwo}. Najpierw pokażemy prostszy przykład, gdy $X\geq0$, a potem uogólnimy go do dowolnego $X$.

  Załóżmy, że $X\geq0$ p.w. Wtedy możemy rozważyć miary $\mu=\prob\restriction\set{G}$ oraz $\nu(A)=\expected{X\mathds{1}_A}$. Od razu widać, że w takim ułożeniu $\nu\ll\mu$, więc na mocy twierdzenia Radona-Nikodyma istnieje $f$ $\set{G}$-mierzalna taka, że
  $$\expected{f\mathds{1}_A}=\int_Af(\omega)\mu(d\omega)=\nu(A)-\expected{X\mathds{1}_A}.$$
  Funkcja $f$ spełnia (W1) z definicji wwo, bo jest $\set{G}$-mierzalna, a (W2) jest potwierdzone przez rachunek wyżej. Czyli $f$ jest wwo $X$ pod warunkiem $\set{G}$.

  Niech teraz $X$ będzie dowolną zmienną losową. Możemy ją rozbić jako 
  $$X=X^+-X^-,$$ 
  gdzie $X^+=\max(0,X)\geq0$ oraz $X^-=-\min(0,X)\geq0$. Do obu tych zmiennych możemy zastosować pierwszą część dowodu, by dostać zmienne $\expected{X^+}{\set{G}}$ oraz $\expected{X^-}{\set{G}}$. Wystarczy zauważyć, że dzięki liniowości $\expected$ możemy w prosty sposób pokazać
  $$\expected{X}{\set{G}}=\expected{X^+}{\set{G}}-\expected{X^-}{\set{G}}$$
\end{proof}

\subsection{Własności wwo}

\begin{theorem}[o arytmetyce wwo]
  Niech $\set{G},\set{G}_1,\set{G}_2\subseteq\set{F}$ będą $\sigma$-ciałami, a $X,X_1,X_2$ całkowalnymi zmiennymi losowymi
  \begin{enumerate}[topsep=8pt, parsep=8pt]
    \item $\expected{\expected{X}{\set{G}}}=\expected{X}$
    \item Jeśli $X\geq0$, to również $\expected{X}{\set{G}}\geq0$
    \item $\expected{aX_1+bX_2}{\set{G}}=a\expected{X_1}{\set{G}}+b\expected{X_2}{\set{G}}$
    \item $\left| \expected{X}{\set{G}} \right| \leq \expected{\left| X\right| }{\set{G}}$
    \item Jeśli $\set{G}_1\subseteq\set{G}_2$, to wówczas
      $$\expected{\expected{X}{\set{G}_1}}{\set{G}_2}=\expected{\expected{X}{\set{G}_2}}{\set{G}_1}=\expected{X}{\set{G}_1}$$
    \item Jeśli $Y$ jest $\set{G}$-mierzalna i $XY$ jest całkowalna, to $\expected{XY}{\set{G}}=Y\expected{X}{\set{G}}$, czyli $Y$ możemy traktować jako stałą.
  \end{enumerate}
\end{theorem}
