\section{09.10.23 : Własności WWO}

Na tym wykładzie zajmiemy się dowodzeniem własności wwo, w tym pokażemy jej istnienie i jedyność.

\subsection{Poprawność: istnienie i jedyność}

\begin{lemma}[WWO jest całkowalna]
  To znaczy, że mając całkowalną zmienną losową $X$ oraz $\sigma$-ciało $\set{G}\subseteq\set{F}$, to zachodzi $\expected{\left|\expected{X}{\set{G}}\right|}<\infty$.
\end{lemma}

\begin{proof}
  Rozważmy zbiór
  $$A=\{\omega\;:\;\expected{X}{\set{G}}(\omega)>0\}=\{\omega\;:\;\expected{X}{\set{G}}\in(0, \infty)\}=\left[\expected{X}{\set{G}}\right]^{-1}((0,\infty))$$
  jako przeciwobraz zbioru $(0,\infty)\in Bor(\R)$ przez funkcję $\set{G}$-mierzalną $\expected{X}{\set{G}}$ wiemy, że $A\in \set{G}$. Ponieważ $\expected{X}{\set{G}}$ jest wwo $X$ pod warunkiem $\set{G}$, to musi warunek (W2): $$0\leq\expected{\expected{X}{\set{G}}\mathds{1}_A}=\expected{X\mathds{1}_A}\leq\expected{|X|\mathds{1}_A}<\infty$$
  bo $X$ jest całkowalna.

  Analogicznie postępujemy dla zbioru $A^c$:
  $$0\leq\expected{-\expected{X}{\set{G}}\mathds{1}_A}=\expected{-X\mathds{1}_{A^c}}\leq\expected{|X|\mathds{1}_{A^c}}<\infty.$$
  
  Zauważmy, że 
  $$\left|
    \expected{X}{\set{G}}
    \right| = \expected{X}{\set{G}}\mathds{1}_A-\expected{X}{\set{G}}\mathds{1}_{A^c}$$
  Dodając obie te nierówności (i korzystając z liniowości wartości oczekiwanej) uzyskujemy
  $$0\leq \expected{\expected{X}{\set{G}}\mathds{1}_A}-\expected{\expected{X}{\set{G}}\mathds{1}_{A^c}}=\expected{\expected{X}{\set{G}}\mathds{1}_A-\expected{X}{\set{G}}\mathds{1}_{A^c}}=\expected{\left|\expected{X}{\set{G}}\right|}<\infty$$
\end{proof}

\begin{lemma}[jedyność p.w.]
  Niech $\set{G}\subseteq{F}$ będzie $\sigma$-ciałem. Jeśli $Y$ i $Y'$ są obie wersjami $\expected{X}{\set{G}}$, to $Y=Y'$ p.w..
\end{lemma}

\begin{proof}
  Ustalmy $\varepsilon>0$ i rozważmy zdarzenie
  $$A_\varepsilon=\{Y-Y'>\varepsilon\}\in \set{G}$$
  które jest $\set{G}$-mierzalne, bo $Y$ i $Y'$ takie są.

  \begin{align*}
    \varepsilon\prob{A_\varepsilon}+\expected{Y'\mathds{1}_{A_\varepsilon}}&=\expected{\varepsilon\mathds{1}_{A_\varepsilon}}+\expected{Y'\mathds{1}_{A_\varepsilon}}=\\
      &=\expected{(\varepsilon+Y')\mathds{1}_{A_\varepsilon}}\leq\\
      &\overset{\star}{\leq}\expected{Y\mathds{1}_{A_\varepsilon}}\overset{(W2)}{=}\expected{X\mathds{1}_{A_\varepsilon}}=\\
      &=\expected{Y'\mathds{1}_{A_\varepsilon}}
  \end{align*}
  gdzie $\star$ wynika z tego, że na zbiorze $A_\varepsilon$ $Y>Y'+\varepsilon$.

  Dostajemy więc, że
  $$\varepsilon\prob{A_\varepsilon}+\expected{Y'\mathds{1}_{A_\varepsilon}}\leq\expected{Y'\mathds{1}_{A_\varepsilon}}$$
  co po przeniesieniu $\expected$ na jedną stronę daje
  $$\varepsilon\prob{A_\varepsilon}\leq0$$
  a ponieważ $\varepsilon>0$, to musi być $\prob{A_\varepsilon}=0$.

  Wówczas 
$$\prob{Y>Y'}=\underbrace{\prob{(\exists\;n)\;Y\geq Y'+\frac{1}{n}}}_{\prob{A_\frac{1}{n}}}=\prob{\bigcup A_\frac{1}{n}}=\lim\prob{A_{\frac{1}{n}}}=0$$
  ponieważ $A_\frac{1}{n}\subseteq A_{\frac{1}{n+1}}$.

  Zamieniając miejscami $Y$ i $Y'$ w dowodzie dostaniemy $\prob{Y'>Y}=0$, czyli obie możliwości są miary zero.
\end{proof}

\begin{theorem}[o istnieniu WWO]\label{istnienie wwo}
  Niech $\set{G}\subseteq\set{F}$ będzie $\sigma$-ciałem, a $X$ jest całkowalną zmienną losową. Istnieje zmienna losowa $Y$ spełniająca oba postulaty wwo $X$ pod warunkiem $\set{G}$.
\end{theorem}

Jest to Twierdzenie \ref{poprawnosc wwo} z poprzedniego wykładu.

Zanim jednak przejdziemy do dowodu \ref{istnienie wwo}, przypomnijmy \acc{twierdzenie Radona-Nikodyma} z teorii miary:
\begin{dygresja}[twierdzenie Radona-Nikodyma]
  Niech $\mu$ i $\nu$ będą $\sigma$-miarami na przestrzeni $(\Omega,\set{G})$ takimi, że $\nu$ jest \acc{absolutnie ciągła} względem $\mu$ [$\nu\ll\mu$], tzn $\mu(A)=0\implies\nu(A)=0$. Wówczas istnieje $\set{G}$-mierzalna funkcja $f:\Omega\to\R$ taka, że
  $$\nu(A)=\int_Af(x)\mu(dx)$$
\end{dygresja}
Funkcję $f$ jak wyżej często oznaczamy $f=\frac{d\nu}{d\mu}$ i nazywamy \acc{pochodną Radona-Nikodyma}.

\begin{proof}
  Wracając do dowodu twierdzenia \ref{istnienie wwo}. Najpierw pokażemy prostszy przykład, gdy $X\geq0$, a potem uogólnimy go do dowolnego $X$.

  Załóżmy, że $X\geq0$ p.w. Wtedy możemy rozważyć miary $\mu=\prob\restriction\set{G}$ oraz $\nu(A)=\expected{X\mathds{1}_A}$. Od razu widać, że w takim ułożeniu $\nu\ll\mu$, więc na mocy twierdzenia Radona-Nikodyma istnieje $f$ $\set{G}$-mierzalna taka, że
  $$\expected{f\mathds{1}_A}=\int_Af(\omega)\mu(d\omega)=\nu(A)-\expected{X\mathds{1}_A}.$$
  Funkcja $f$ spełnia (W1) z definicji wwo, bo jest $\set{G}$-mierzalna, a (W2) jest potwierdzone przez rachunek wyżej. Czyli $f$ jest wwo $X$ pod warunkiem $\set{G}$.

  Niech teraz $X$ będzie dowolną zmienną losową. Możemy ją rozbić jako 
  $$X=X^+-X^-,$$ 
  gdzie $X^+=\max(0,X)\geq0$ oraz $X^-=-\min(0,X)\geq0$. Do obu tych zmiennych możemy zastosować pierwszą część dowodu, by dostać zmienne $\expected{X^+}{\set{G}}$ oraz $\expected{X^-}{\set{G}}$. Wystarczy zauważyć, że dzięki liniowości $\expected$ możemy w prosty sposób pokazać
  $$\expected{X}{\set{G}}=\expected{X^+}{\set{G}}-\expected{X^-}{\set{G}}$$
\end{proof}

\subsection{Własności wwo}

\begin{theorem}[o arytmetyce wwo]\label{o arytmetyce wwo}
  Niech $\set{G},\set{G}_1,\set{G}_2\subseteq\set{F}$ będą $\sigma$-ciałami, a $X,X_1,X_2$ całkowalnymi zmiennymi losowymi
  \begin{enumerate}[topsep=8pt, parsep=8pt]
    \item $\expected{\expected{X}{\set{G}}}=\expected{X}$
    \item Jeśli $X\geq0$, to również $\expected{X}{\set{G}}\geq0$
    \item $\expected{aX_1+bX_2}{\set{G}}=a\expected{X_1}{\set{G}}+b\expected{X_2}{\set{G}}$
    \item $\left| \expected{X}{\set{G}} \right| \leq \expected{\left| X\right| }{\set{G}}$
    \item Jeśli $\set{G}_1\subseteq\set{G}_2$, to wówczas
      $$\expected{\expected{X}{\set{G}_1}}{\set{G}_2}=\expected{\expected{X}{\set{G}_2}}{\set{G}_1}=\expected{X}{\set{G}_1}$$
      To znaczy, że mając informacje o $X$ w dwóch zawartych w sobie ciałach, to mniejsze zawsze wygrywa.
    \item Jeśli $Y$ jest $\set{G}$-mierzalna i $XY$ jest całkowalna, to $\expected{XY}{\set{G}}=Y\expected{X}{\set{G}}$, czyli $Y$ możemy traktować jako stałą.
  \end{enumerate}
\end{theorem}

\begin{proof}
  \begin{enumerate}
    \item Wystarczy wstawić $G=\Omega$ w warunek (W2):
      $$\expected{\expected{X}{\set{G}}}=\expected{\expected{X}{\set{G}}\mathds{1}_\Omega}\overset{(W2)}{=}
      \expected{X\mathds{1}_\Omega}=\expected{X}$$
    \item Wynika z dowodu twierdzenia o istnieniu, bo $\frac{d\nu}{d\mu}=\expected{X}{\set{G}}$. Gdyby $A=\{\omega\;:\;\expected{X}{\set{G}}<0\}$, to wówczas
      $$\expected{X\mathds{1}_A}=\nu(A)=\int_A\expected{X}{\set{G}}(\omega)\mu(d\omega)=\expected{\expected{X}{\set{G}}\mathds{1}_A}<0$$
        ale przecież $X\geq0\implies\expected{X\mathds{1}_A}\geq0$, więc $A=\emptyset$.
      \item Można to zrobić na dwa sposoby: licząc wszystko pokolei, albo można sprawdzić, czy $Y=a\expected{X_1}{\set{G}}+b\expected{X_2}{\set{G}}$ spełnia warunki wwo tej samej zmiennej co $\expected{aX_1+bX_2}{\set{G}}$. Wówczas obie te zmienne są równe prawie wszędzie.

        Warunek $\set{G}$-mierzalności dla $Y$ jest spełniony, bo $Y$ jest kombinacją liniową dwóch funkcji $\set{G}$-mierzalnych. Wystarczy więc sprawdzić warunek (W2). W tym celu ustalmy $A\in\set{G}$.
        \begin{align*}
          \expected{Y\mathds{1}_A}&\overset{\star}{=}a\expected{\expected{X_1}{\set{G}}\mathds{1}_A}+b\expected{\expected{X_2}{\set{G}}\mathds{1}_A}=\\
                                  &=a\expected{X_1\mathds{1}_A}+b\expected{X_2\mathds{1}_A}=\\
                                  &=\expected{(aX_1+bX_2)\mathds{1}_A}=\expected{\expected{aX_1+bX_2}{\set{G}}\mathds{1}_A}
        \end{align*}
      \item Wiemy, że $-|X|\leq X\leq|X|$. Korzystając z punktu 2 dostajemy
        $$0\leq X+|X|\implies 0\leq \expected{|X|}{\set{G}}+\expected{X}{\set{G}}\implies -\expected{|X|}{\set{G}}\leq\expected{X}{\set{G}}$$
        $$0\leq |X|-X\implies 0\leq \expected{|X|}{\set{G}}-\expected{X}{\set{G}}\implies \expected{X}{\set{G}}\leq\expected{|X|}{\set{G}}$$
        Po złożeniu tych dwóch nierówności:
        $$-\expected{|X|}{\set{G}}\leq\expected{X}{\set{G}}\leq\expected{|X|}{\set{G}}$$
        wiemy, że $-|\expected{X}{\set{G}}|\leq\expected{X}{\set{G}}\leq|\expected{X}{\set{G}}|$, więc musi być
        $$|\expected{X}{\set{G}}|\leq\expected{|X|}{\set{G}}.$$
      \item Zaczniemy od sprawdzenia, że $\expected{\expected{X}{\set{G}_2}}{\set{G}_1}=\expected{X}{\set{G}_1}$. Wybierzmy $A\in\set{G}_1\subseteq\set{G}_2$:
        $$\expected{\expected{X}{\set{G}_1}\mathds{1}_A}=\expected{X\mathds{1}_A}=\expected{\expected{X}{\set{G}_2}\mathds{1}_A}$$
        co potwierdza warunek (W2). $\set{G}_1$-mierzalność $\expected{X}{\set{G}_2}$ jest oczywista, gdyż $\set{G}_1\subseteq\set{G}_2$, $\expected{X}{\set{G}_2}$ jest $\set{G}_2$-mierzalne, a po obcięciu do $\set{G}_1$ dostajemy funkcję $\set{G}_1$-mierzalną.

        Pozostaje nam sprawdzić czym jest $\expected{\expected{X}{\set{G}_1}}{\set{G}_2}$. Roboczo nazwiemy $Y=\expected{X}{\set{G}_1}$. Jest to funkcja $\set{G}_1$-mierzalna, ale dzięki $\set{G}_1\subseteq\set{G}_2$ mamy też $\set{G}_2$-mierzalność. W takim razie (tak jak w jednym z przykładów z pierwszego wykładu) $\expected{Y}{\set{G}_2}=Y$. Pisząc bez używania litery $Y$ dostajemy
        $$\expected{\expected{X}{\set{G}_1}}{\set{G}_2}=\expected{X}{\set{G}_1}$$
    \item Ćwiczenie, a poniżej moja próba.

      Jeśli $Y$ jest $\set{G}$-mierzalne, to $Y\expected{X}{\set{G}}$ też takie jest jako iloczyn dwóch funkcji $\set{G}$-mierzalnych. Pozostaje sprawdzić warunek (W2).

      Zacznijmy od $Y=\sum a_i\mathds{1}_{A_i}$ dla $A_i\in\set{G}$, czyli od funkcji prostej. Wybierając $A\in\set{G}$ możemy ograniczyć się do zbiorów $A_i$, gdyż są one rozłączne i na dowolnym innym zbiorze $Y=0$. Mamy więc
      \begin{align*}
        \expected{\expected{XY}{\set{G}}\mathds{1}_{A_i}}&\overset{(W2)}{=}\expected{XY\mathds{1}_{A_i}}=\expected{a_iX\mathds{1}_{A_i}}=a_i\expected{X\mathds{1}_{A_i}}=\\
                                                          &=a_i\expected{\expected{X}{\set{G}}\mathds{1}_{A_i}}=\expected{(a_i\expected{X}{\set{G}})\mathds{1}_{A_i}}=\\
                                                          &=\expected{(Y\expected{X}{\set{G}})\mathds{1}_{A_i}}
      \end{align*}
      Czyli $\expected{XY}{\set{G}}=Y\expected{X}{\set{G}}$ dla przypadku gdy $Y$ jest funkcją prostą.

      Jeśli teraz $Y$ jest dowolną nieujemną funkcją mierzalną, to istnieje ciąg funkcji prostych
      $$s_1\leq s_2\leq...\leq s_n\leq...\quad \lim s_i=Y$$
      Wówczas dla dowolnego $A\in\set{G}$
      \begin{align*}
        \expected{\expected{XY}\mathds{1}_A}&=\expected{XY\mathds{1}_A}=\expected{X\lim s_i\mathds{1}_A}\overset{\star}{=}\lim\expected{X s_i\mathds{1}_A}=\\
                                            &\overset{\star\star}{=}\lim \expected{s_i\expected{X}{\set{G}}\mathds{1}_A}=\expected{\lim s_i\expected{X}{\set{G}}\mathds{1}_A}=\\
                                            &=\expected{Y\expected{X}{\set{G}}\mathds{1}_A}
      \end{align*}
      $\star$ można zrobić na mocy twierdzenia o monotoniczności ciągu $s_i$ dla zwykłej $\expected$, natomiast $\star\star$ stosuje poprzedni przypadek $Y$.

      Pozostaje przypadek, gdy $Y$ jest dowolną $\set{G}$-mierzalną zmienną losową. Wówczas możemy rozbić $Y=Y^+-Y^-$ i skorzystać z liniowości wwo:
      $$\expected{XY}{\set{G}}=\expected{XY^+}{\set{G}}-\expected{XY^-}{\set{G}}=Y^+\expected{X}{\set{G}}-Y^-\expected{X}{\set{G}}=Y\expected{X}{\set{G}}$$
  \end{enumerate}
\end{proof}

\begin{theorem}[o zbieżności i ciągłości]
  Niech $\set{G}\subseteq\set{F}$ będzie $\sigma$-ciałem, a $X,X_1, X_2,...$ będzie ciągiem całkowalnych zmiennych losowych. Wówczas
  \begin{enumerate}[topsep=8pt, parsep=8pt]
    \item Jeśli $0\leq X_1\leq X_2\leq ...$ oraz $X_n\nnearrow X$, to $\expected{X_n}{\set{G}}\nnearrow\expected{X}{\set{G}}$ p.w. (twierdzenie o zbieżności monotonicznej)
    \item Jeśli $X\geq0$, to $\expected{\liminf_n X_n}{\set{G}}\leq\liminf_n\expected{X_n}{\set{G}}$ (lemat Fatou).
    \item Jeśli $|X_n|\leq Y$ oraz $Y$ jest całkowalny i $X_n\to X$ p.w., to $\expected{X_n}{\set{G}}\to\expected{X}{\set{G}}$ (twierdzenie o zbieżności ograniczonej)
  \end{enumerate}
\end{theorem}

\begin{proof}
  \begin{enumerate}
    \item Zauważamy, że ciąg $\expected{X_n}{\set{G}}$ jest niemalejący i ograniczony przez $\expected{X}{\set{G}}$ (na mocy punktu 2 z poprzedniego twierdzenia).

      Niech $Y=\lim\expected{X_n}{\set{G}}$. Wystarczy, że pokażemy $Y=\expected{X}{\set{G}}$ p.w., czyli sprawdzimy warunki (W1) i (W2). Oczywiście, warunek (W1) wynika z faktu, że granica ciągu funkcji $\set{G}$-mierzalnych jest nadal $\set{G}$-mierzalna. Dla sprawdzenia warunku (W2) wybierzmy $A\in\set{G}$
      \begin{align*}
        \expected{Y\mathds{1}_A}&=\expected{\lim\expected{X_n}{\set{G}}\mathds{1}_A}\overset{\star}{=}\lim\expected{\expected{X_n}{\set{G}}\mathds{1}_A}=\\
                                &=\lim\expected{X_n\mathds{1}_A}=\expected{\lim X_n\mathds{1}_A}=\expected{X\mathds{1}_A}
      \end{align*}
      czyli $Y=\expected{X}{\set{G}}$ p.w.
    \item Zaczniemy od dwóch obserwacji:
      \begin{itemize}
        \item Dla ciągu $\{a_n\}$ $\lim\inf a_n$ to najmniejszy z jego punktów skupienia, równoważnie:
          $$\liminf_na_n=\lim_{k\to\infty}\inf_{n\geq k}a_n$$
        \item[{\color{green}\PHrosette}] Dla dowolnej przeliczalnej rodziny zmiennych losowych $\{Z_n\}_{n\in T}$ i dla dowolnego $t\in T$ mamy
          $$\inf_{s\in T}Z_s\leq Z_t$$
          $$\expected{\inf_{s\in T} Z_s}\leq\expected{Z_t}$$
          $$\expected{\inf_{s\in T} Z_s}\leq\inf_{t\in T}\expected{Z_t}$$
          (co jest tak naprawdę wersją lematu Fatou dla $\expected$ z RP1R).
      \end{itemize}
      Stosując obserwację \PHtunny w przejściach $\star$, obserwację \PHrosette w przejsciu $\star\star$ oraz ptk 1. ($\inf_{n>k}X_n\leq \inf_{n>k+1}X_n$) w $\star\star\star$, dostajemy
      \begin{align*}
        \expected{\liminf_n X_n}{\set{G}}&\overset{\star}{=}\expected{\lim_k\inf_{n>k}X_n}{\set{G}}\overset{\star\star\star}{=}\lim_k\expected{\inf_{n>k}X_n}{\set{G}}\leq\\
                                         &\overset{\star\star}{\leq}\lim_k\inf_{n>k}\expected{X_n}{\set{G}}\overset{\star}{=}\liminf_n\expected{X_n}{\set{G}}
      \end{align*}
    \item Rozważmy zmienne $X_n'=Y+X_n$. Ponieważ $|X_n|\leq Y$, to $Y+X_n\geq 0$.
      \begin{align*}
        \expected{Y+\liminf X_n}{\set{G}}&=\expected{\liminf(X_n+Y)}{\set{G}}\overset{3.}{\leq}\liminf\expected{Y+X_n}{\set{G}}=\\
                                         &=\expected{Y}+\liminf\expected{X_n}{\set{G}}
      \end{align*}
      To daje nam, że $\expected{\liminf X_n}{\set{G}}\leq\liminf\expected{X_n}{\set{G}}$.

      Postępując analogicznie dla $X_n"=Y-X_n$ (które dalej jest $\geq0$) dostaniemy $\expected{\limsup X_n}{\set{G}}\geq\limsup\expected{X_n}{\set{G}}$:
      \begin{align*}
        \expected{Y-\limsup X_n}{\set{G}}&=\expected{\limsup(Y-X_n)}{\set{G}}=\\
                                         &=-\expected{\liminf(X_n-Y)}{\set{G}}\overset{3.}{\geq}-\liminf\expected{X_n-Y}{\set{G}}=\\
                                         &=\limsup\expected{Y-X_n}{\set{G}}
      \end{align*}
      Ale wiemy, że $\liminf X_n=X$ oraz $\limsup X_n=X$, czyli
      $$\liminf\expected{X_n}{\set{G}}\geq\expected{\liminf X_n}{\set{G}}=\expected{X}{\set{G}}=\expected{\limsup{X_n}}{\set{G}}\geq\limsup\expected{X_n}{\set{G}}$$
      ale przecież $\liminf\expected{X_n}{\set{G}}\leq\limsup\expected{X_n}{\set{G}}$, czyli musi być
      $$\liminf\expected{X_n}{\set{G}}=\expected{X}{\set{G}}=\limsup\expected{X_n}{\set{G}}$$
      i ponieważ $\liminf=\limsup=\lim$ to mamy
      $$\lim\expected{X_n}{\set{G}}=\expected{X}{\set{G}}.$$
  \end{enumerate}
\end{proof}

\begin{theorem}\label{2.6 twierdzenie}
  Niech $\set{G}\subseteq\set{F}$ będzie $\sigma$-ciałem. Załóżmy, że
  \begin{itemize}
    \item $X$ jest $\set{G}$-mierzalna
    \item $Y$ jest niezależna od $\set{G}$
    \item funkcja $\psi:\R^2\to\R$ jest mierzalna taka, że
      $$\expected{|\psi(X,Y)|}<\infty.$$
  \end{itemize}
  Wówczas 
  $$\expected{\psi(X,Y)}{\set{G}}=\Psi(X),$$ 
  gdzie funkcja $\Psi:\R\to\R$ jest zdefiniowana jako $\Psi(x)=\expected{\psi(x,Y)}$.
\end{theorem}

\begin{proof}
%  Jeśli $\psi(x,y)=xy$, to korzystając z punktu 3. twierdzenia \ref{o arytmetyce wwo} dostajemy
%  $$\psi(x)=\expected{xY}=x\expected{Y}$$

  Tak jak w dowodzie ostatniego punktu twierdzenia \ref{o arytmetyce wwo} zaczniemy od funkcji $\phi$ prostych i stopniowo przejdziemy do dowolnych funkcji mierzalnych. 

  Zaczniemy od funkcji $\phi$ postaci $\phi(x, y)=\mathds{1}_{A\times B}(x, y)=\mathds{1}_A(x)\cdot\mathds{1}_B(y)$ dla pewnych $A, B\in Bor(\R)$.

  Po pierwsze zauważmy, że jeżeli $X$ jest $\set{G}$-mierzalne, to $\mathds{1}_A(X)$ też takie jest. Analogicznie, jeśli $Y$ jest niezależna od $\set{G}$, to $\mathds{1}_B(Y)$ też jest niezależne i wtedy $\expected{\mathds{1}_B(Y)}{\set{G}}=\expected{\mathds{1}_B(Y)}$. Korzystając z \ref{o arytmetyce wwo} w przejściu $\star$, dostajemy
  \begin{align*}
    \expected{\phi(X, Y)}{\set{G}}&=\expected{\mathds{1}_A(X)\mathds{1}_B(Y)}{\set{G}}=\\ 
                                  &\overset{\star}{=}\mathds{1}_A(X)\expected{\mathds{1}_B(Y)}{\set{G}}=\\ 
                                  &=\mathds{1}_A(X)\expected{\mathds{1}_B(Y)}
  \end{align*}
  Z drugiej strony mamy
  \begin{align*}
    \Psi(x)=\expected{\phi(x, Y)}&=\expected{\mathds{1}_A(x)\mathds{1}_B(Y)}=\\ 
                                &=\mathds{1}_A(x)\expected{\mathds{1}_B(Y)}
  \end{align*}
  czyli $\Psi(X)=\mathds{1}_A(X)\expected{\mathds{1}_B(Y)}$ tak jak chcieliśmy.

  Chcemy przejść teraz do funkcji postaci $\phi(x, y)=\mathds{1}_C(x, y)$ dla $C\in Bor(\R^2)$. Skorzystamy przy tym z lematu o $\pi-\lambda$ układach:

  \begin{dygresja}[lemat o $\pi-\lambda$ układach]
    Niech $P$ będzie $\pi$-układem (tzn. $A, B\in P\implies A\cap B\in P$) oraz niech $P\subseteq L$ będzie $\lambda$-układem ($\Omega\in L$, $A\subseteq B\in L\implies B\setminus A\in L$ i $A_1\subseteq A_2\subseteq...\in L\implies \bigcup A_i\in L$). 

    Wówczas $L$ jest $\sigma$-ciałem i w szczególności zawiera $\sigma$-ciało generowane przez $\pi$-układ $P$.
  \end{dygresja}

  Rozważmy zbiór
  $$D=\{C\in Bor\R^2\;:\;\expected{\mathds{1}_C(X,Y)}{\set{G}}=\Psi_C(X)  \}.$$
  Oczywiście, zbiór wszystkich "kwadratów" $A\times B$ dla $A,B\subseteq\R$ jest $\pi$-układem zbiorów z $\R^2$ i zgodnie z tym co już pokazaliśmy, zawiera się on w $D$. Chcemy więc pokazać, że $D$ jest $\lambda$-układem.

  \begin{enumerate}
    \item $\Omega\in D$
      
      Jest to prawdą, bo w tym przypadku $\Omega=\R\times\R$, czyli podlega pod przypadek udowodniony wyżej.

    \item $A\subseteq B\in D\implies B\setminus A\in D$

      Niech $A\subseteq B\in D$, wówczas
      $$\mathds{1}_{B\setminus A}(X, Y)=\mathds{1}_B(X, Y)-\mathds{1}_A(X, Y)$$
      czyli wówczas
      \begin{align*}
        \expected{\mathds{1}_{B\setminus A}(X, Y)}{\set{G}}&=\expected{\mathds{1}_B(X, Y)-\mathds{1}_A(X, Y)}{\set{G}}=\\ 
                                                           &=\expected{\mathds{1}_B(X, Y)}{\set{G}}-\expected{\mathds{1}_A(X, Y)}{\set{G}}=\\ 
                                                           &=\Psi_B(X)-\Psi_A(Y)=\expected{\mathds{1}_B(X, Y)}-\expected{\mathds{1}_A(X, Y)}=\\ 
                                                           &=\expected{\mathds{1}_{B\setminus A}(X, Y)}=\Psi(X)
      \end{align*}
      i tym samym dostajemy $B\setminus A\in D$
    \item $A_1\subseteq A_2\subseteq...\implies\bigcup A_i\in D$

      Wystarczy zauważyć, że przy wstępującym ciągu zbiorów $A_i$ mamy $\mathds{1}_{A_n}\geq \mathds{1}_{A_{n-1}}$ oraz $\mathds{1}_{\bigcup A_i}=\lim\mathds{1}_{A_i}$, a następnie zastosować twierdzenie o zbieżności monotonicznej:
      \begin{align*}
        \expected{\bigcup \mathds{1}_{A_i}(X, Y)}{\set{G}}&=\expected{\lim\mathds{1}_{A_i}(X, Y)}{\set{G}}=\\ 
                                                          &=\lim\expected{\mathds{1}_{A_i}(X, Y)}{\set{G}}=\\ 
                                                          &=\lim\Psi_{A_i}(X)=\lim\expected{\mathds{1}_{A_i}(X, Y)}=\\ 
                                                          &= \expected{\lim\mathds{1}_{A_i}(X, Y)}=\\ 
                                                          &=\expected{\mathds{1}_{\bigcup A_i}(X, Y)}
      \end{align*}
  \end{enumerate}

  W ten sposób pokazaliśmy już, że twierdzenie jest prawdziwe dla funkcji prostych $\phi:\R^2\to \R$. Przejdziemy teraz do przypadku, gdy $\phi$ jest nieujemną funkcją mierzalną, czyli istnieje ciąg funkcji prostych $s_1\leq s_2\leq ...\leq s_n\leq...\leq \phi$ taki, że $\phi=\lim s_i$.
  \begin{align*}
    \expected{\phi(X, Y)}{\set{G}}&=\expected{\lim s_i(X, Y)}{\set{G}}=\\ 
                                  &\overset{\heartsuit}{=}\lim\expected{s_i(X, Y)}{\set{G}}=\\ 
                                  &=\lim\expected{s_i(X, Y)}=\\ 
                                  &=\expected{\lim s_i(X, Y)}=\expected{\phi(X, Y)}
  \end{align*}
  W przejściu $\heartsuit$ skorzystaliśmy ponownie z twierdzeniu o zbieżności monotonicznej.

  Pozostaje jedynie przypadek, gdy $\phi$ jest dowolną funkcją mierzalną. Wtedy możemy zapisać $\phi=\phi^+-\phi^-$ dla $\phi^+$ oraz $\phi^-$ nieujemnych. Wtedy, korzystając z wcześniej już pokazanych form funkcji mierzalnych dostajemy
  \begin{align*}
    \expected{\phi(X, Y)}{\set{G}}&=\expected{\phi^+(X, Y)-\phi^-(X, Y)}{\set{G}}=\\ 
                                  &=\expected{\phi^+(X, Y)}{\set{G}}-\expected{\phi^-(X, Y)}{\set{G}}=\\ 
                                  &=\expected{\phi^+(X, Y)}-\expected{\phi^-(X, Y)}=\\ 
                                  &=\expected{\phi^+(X, Y)-\phi^-(X, Y)}=\expected{\phi(X, Y)}
  \end{align*}

\end{proof}
