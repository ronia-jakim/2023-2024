\subsection{Zadania}

\setcounter{problem}{0}

\begin{problem}
  Załóżmy, że $\{X_n\}_{n\in\N}$ jest ciągiem niezależnych zmiennych losowych o takim samym rozkładzie, średniej $0$ i skończonej wariancji. Rozważmy filtrację $\mathds{F}=\{\set{F}_n\}$ zadaną przez $\set{F}_n=\sigma(X_0,X_1,...,X_n)$. Udowodnij, że ciąg
  $$Z_n=X_0X_1+X_1X_2+....+X_{n-1}X_n, \quad Z_0=0$$
  jest $\mathds{F}$-martyngałem.
\end{problem}

\begin{solution}
  Chcemy pokazać, że
  $$\expected{Z_{n+1}}{\set{F}_{n}}=Z_n$$
  dla dowolnego $n\in\N$.
  \begin{align*}
    \expected{Z_{n+1}}{\set{F}_n}&=\expected{Z_n+X_nX_{n+1}}{\set{F}_n}=\\ 
                                 &=\expected{Z_n}{\set{F}}+\expected{X_nX_{n+1}}{\set{F}_n}=\\ 
                                 &=Z_n+X_n\expected{X_{n+1}}{\set{F}_n}
  \end{align*}
  ponieważ $X_n$ jest $\set{F}_n$-mierzalne oraz
  $$\expected{|X_nX_{n+1}|}\leq\expected{X_n^2}^{1/2}\expected{X_{n+1}^2}^{1/2}<\infty$$
  gdzie nierówność wynika z nierówności Cauchy'ego-Schwarza, a $\expected{X_n^2}=Var(X_n)<\infty$.

  Zauważmy teraz, że $X_{n+1}$ jest niezależne od $\set{F}_n$, gdyż $X_n$ jest niezależne od każdej ze zmiennych $X_1,...,X_n$. W takim razie, $\expected{X_{n+1}}{\set{F}_n}=\expected{X_{n+1}}=0$, a więc ostatecznie dostajemy
  $$\expected{Z_{n+1}}{\set{F}_n}=Z_n+X_n\expected{X_{n+1}}{\set{F}_n}=Z_n+X_n\cdot 0=Z_n$$
  Czyli $Z_n$ faktycznie jest martyngałem.
\end{solution}

\begin{problem}
  Ustalmy $\theta\in\R$. Niech $X_1,X_2,...$ będzie ciągiem niezależnych zmiennych losowych o takim samym rozkładzie takich, że 
  $$\expected{e^{\theta X_1}}<\infty.$$
  Pokaż, że 
  $$M_n=\expected{e^{\theta X_1}}^{-n}\prod_{j=1}^ne^{\theta X_j}$$
  jest $\mathds{F}$-martyngałem dla filtracji $\mathds{F}=\{\set{F}_n\}$ danej przez $\set{F}_n=\sigma(X_1,...,X_n)$.
\end{problem}

\begin{solution}
  Zacznijmy od obserwacji, że
  $$M_{n+1}=\expected{e^{\theta X_1}}^{-n-1}\prod_{j=1}^{n+1}e^{\theta X_j}=M_n\cdot\expected{e^{\theta X_1}}^{-1}e^{\theta X_{n+1}}$$
  w takim razie, wwo $M_{n+1}$ to jest
  \begin{align*}
    \expected{M_{n+1}}{\set{F}_n}&=\expected{e^{\theta X_1}}^{-1}\cdot \expected{M_n\cdot e^{\theta X_{n+1}}}{\set{F}_n}
  \end{align*}
  Od razu widać, że $M_n$ jest mierzalne względem $\set{F}_n$, bo zależy tylko od zmiennych $X_1,...,X_n$ które $\set{F}_n$ generują. Chcemy teraz sprawdzić, czy $\expected{|M_n\cdot e^{\theta X_{n+1}}|}<\infty$, wówczas możemy wyciągnąć $M_n$ przed wwo.
  \begin{align*}
    \expected{|M_n\cdot e^{\theta X_{n+1}}|}&=\expected{\left |\expected{e^{\theta X_1}}^{-1}\cdot \prod_{j=1}^{n} e^{\theta X_j}\cdot e^{\theta X_{n+1}} \right| }=\\ 
                                            &=|\expected{e^{\theta X_1}}|^{-n}\cdot\expected{ \prod_{j=1}^{n+1}e^{\theta X_j} }=\\ 
                                            &=|\expected{e^{\theta X_1}}|^{-n}\cdot \prod_{j=1}^{n+1}\expected{e^{\theta X_j}}=\\ 
                                            &=\expected{e^{\theta X_{n+1}}}=\expected{e^{\theta X_1}}<\infty
  \end{align*}
  ponieważ jeśli $\{X_n\}$ są niezależne, to $e^{X_n}$ też są niezależne, a dla niezależnych $X,Y$ zachodzi $\expected{XY}=\expected{X}\expected{Y}$.
  W takim razie dostajemy
  $$\expected{M_{n+1}}{\set{F}_n}=\expected{e^{\theta X_1}}^{-1}\expected{M_n\cdot e^{\theta X_{n+1}}}{\set{F}_n}=M_n\expected{e^{\theta X_1}}^{-1}\expected{e^{\theta X_{n+1}}}{\set{F}_{n}}$$
  ale ponieważ $\set{F}_n$ nie zawiera ani grama informacji o $X_{n+1}$, to $e^{\theta X_{n+1}}$ jest niezależne od $\set{F}_n$, więc
  $$\expected{e^{\theta X_{n+1}}}{\set{F}_n}$$
    a to już daje to co chcieliśmy.
\end{solution}

\begin{problem}
  Niech $\{Y_n\}$ będzie ciągiem niezależnych zmiennych losowych o średniej $0$ i wariancji $\sigma^2$. Pokaż, że ciąg 
  $$X_n=\left(\sum_{k=1}^nY_k\right)^2-n\sigma^2$$
  jest martyngałem.
\end{problem}

\begin{solution}
  Zacznijmy od wyrażenia $X_{n+1}$ przy użyciu $X_n$
  \begin{align*}
    X_{n+1}&=\left( \sum_{k=1}^{n+1}Y_{k} \right)^2 - (n+1)\sigma^2= \left( \sum_{k=1}^nY_k +Y_{n+1} \right)^2 - (n+1)\sigma^2=\\ 
           &=\left(\sum_{k=1}^n Y_k \right)^2 -n\sigma ^2 + 2 Y_{n+1}\left( \sum_{k=1}^n Y_k \right) +Y_{n+1}^2 -\sigma^2=\\ 
           &=X_n+2\left( \sum_{k=1}^nY_{n+1}Y_k \right) -\sigma^2 + Y_{n+1}^2
  \end{align*}
  Rozważmy teraz filtrację $\mathds{F}=\{\set{F}_n\}$ dla ciągu $\set{F}_n=\sigma(Y_1,...,Y_n)$. Wtedy mamy
  \begin{align*}
    \expected{X_{n+1}}{\set{F}_n}&=\expected{X_n+2 \left( \sum_{k=1}^nY_{n+1}Y_k \right) -\sigma^2+Y_{n+1}^2}{\set{F}_n}=\\
                                 &=\expected{X_n}{\set{F}_n}+2\sum\expected{Y_{n+1}Y_n}{\set{F}_n}-\sigma^2+\expected{Y_{n+1}^2}{\set{F}_n}=(\star)
  \end{align*}
  Zmienna $X_n$ jest $\set{F}_n$-mierzalna bo korzysta tylko z informacji zapisywanych przez $Y_1,...,Y_n$. Zmienna $Y_{n+1}$ jest niezależna od $\set{F}_n$, bo zmienne $\{Y_n\}$ są niezależne. Zmienna $Y_{n+1}Y_k$ jest całkowalna dla dowolnego $k$, bo $\expected{|Y_{n+1}Y_k|}=\expected{|Y_{n+1}|}\expected{|Y_k|}=0$. W takim razie wracając do równości wyżej, można napisać
  \begin{align*}
    (\star)&=X_n+2\sum Y_{k}\expected{Y_{n+1}}{\set{F}_n}-\sigma^2+\expected{Y_{n+1}^2}=\\ 
           &=X_n+2\sum Y_k\expected{Y_{n+1}}-\sigma^2+\sigma^2=X_n+2\sum Y_k\cdot 0=X_n
  \end{align*}
  W takim razie ciąg $\{X_n\}$ jest $\mathds{F}$-martyngałem.
\end{solution}

\begin{problem}
  Niech $\{X_n\}$ będzie $\mathds{F}$-martyngałem. Pokaż, że dla każdych naturalnych $m,n$
  $$\expected{X_{m+n}}{\set{F}_n}=X_n.$$
\end{problem}

\begin{solution}
  Dla $m=1$ działa z definicji martyngału.

  Zakładamy teraz, że $\expected{X_{m+n}}{\set{F}_n}=X_n$ i chcemy to samo dostać dla $m+1$ (indukcja).

  Z własnośći wwo \ref{o arytmetyce wwo} wiemy, że jeśli mamy dwa zawarte w sobie $\sigma$-ciała $\set{G}_1\subseteq\set{G}_2$, to
  $$\expected{\expected{X}{\set{G}_2}}{\set{G}_1}=\expected{X}{\set{G}_1}$$
  tutaj bierzemy $\set{G}_1=\set{F}_n$ natomiast $\set{G}_2=\set{F}_{n+m}$. Mamy więc
  $$\expected{\expected{X_{m+n+1}}{\set{F}_{m+n}}}{\set{F}_n}=\expected{X_{m+n+1}}{\set{F}_n}$$
  Z faktu, że $\{X_n\}$ jest martyngałem to mamy
  $$\expected{X_{m+n+1}}{\set{F}_{n+m}}=X_{m+n}$$
  czyli przechodząc już do sedna sprawy,
  \begin{align*}
    \expected{X_{m+n+1}}{\set{F}_n}&=\expected{\expected{X_{m+n+1}}{\set{F}_{m+n}}}{\set{F}_n}=\\ 
                                   &=\expected{X_{m+n}}{\set{F}_n}=X_n
  \end{align*}
  bo ostatnie przejście wynika z założenia indukcyjnego.
\end{solution}

\begin{problem}
  Niech $\{X_n\}$ będzie nadmartyngałem takim, że $\expected{X_n}=\expected{X_0}<\infty$ dla każdego $n\in\N$. Pokaż, że $\{X_n\}$ jest martyngałem.
\end{problem}

\begin{solution}
  Nadmartyngał spełnia
  $$\expected{X_{n+1}}{\set{F}_n}\leq X_n$$

\end{solution}
