\subsection{Zadania}

\begin{problem}
  Uzasadnij, że jeżeli $\{X_n\}$ są niezależnymi całkowalnymi zmiennymi losowymi o tym samym rozkładzie, a $T$ jest czasem zatrzymania względem filtracji $\set{F}_n=\sigma(X_1,...,X_n)$, takim że $\expected{T}<\infty$, to 
  $$\expected{S_T}=\expected{T}\cdot\expected{X_1}$$
  gdzie $S_n=X_1+X_2+...+X_n$.
\end{problem}

\begin{solution}
  Na liście drugiej pojawiło się zadanie, w który mieliśmy ciąg niezależnych zmiennych losowych o tym samym rozkładzie oraz zmienną $N$ niezależną od tego ciągu. Naszym zadaniem było policzenie $\expected{S_N}{N}$, a następnie $\expected{S_N}$. Tutaj działamy na dość podobnym założeniu z tym, że nasze $N$ (czyli $T$) nie jest niezależne od danego ciągu $X_n$.

  Liczenie od razu $\expected{S_T}$ nie brzmi jak dobra zabawa, więc spróbujmy policzyć $\expected{S_T}{T}$. Wybierzmy $G\in\sigma(T)$.
\end{solution}
