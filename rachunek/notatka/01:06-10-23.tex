\section{06.10.23 : Warunkowa wartość oczekiwana}

\subsection{Prawdopodobieństwo warunkowe}

Tak jak zwykle do tej pory, pracować będziemy na przestrzeni probabilistycznej $(\Omega,\set{F},\prob)$.

Przypomnijmy definijcję \buff{prawdopodobieństwa warunkowego} z Rachunku Prawdopodobieństwa 1 (i z liceum). Dla zdarzenia $A\in\set{F}$ takiego, że $\prob{A}\in(0, 1)$ definiujemy prawdopodobieństwo warunkowe jako
$$\prob{B\;|\;A}=\frac{\prob{A\cap B}}{\prob{A}}.$$

Wartość ta informuje nas o zajściu $B$ wtedy, gdy jesteśmy pewni, że $A$ zaszło. Ale co, jeśli nasza wiedza dotycząca $A$ jest mniej pewna? To znaczy, \acc{co jeśli $\prob{A}=0$}? Dość naturalne wydaje się rozważenie zdarzenia przeciwnego i zsumowania obu prawdopodobieństw:
$$\mathds{1}_A\prob{B\;|\;A}+\mathds{1}_{A^c}\prob{B\;|\;A^c}.$$
Zauważmy od razu, że wyrażenie $\mathds{1}_A\prob{B\;|\;A}$ jest zmienną losową.

W przypadku, gdy mamy dwa zbiory, $A_1,A_2\in\set{F}$, i chcemy zbadać $\prob{B\;|\;A_1\cap A_2}$ możemy powyższe rozumowanie rozszerzyć na wszystkie możliwe kombinacje $A_1$, $A_2$ i ich dopełnień:
$$\mathds{1}_{A_1\cap A_2}\prob{B\;|\;A_1\cap A_2}+\mathds{1}_{A_1\cap A_2^c}\prob{B\;|\;A_1\cap A_2^c}+\mathds{1}_{A_1^c\cap A_2}\prob{B\;|\;A_1^c\cap A_2}+\mathds{1}_{A_1^c\cap A_2^c}\prob{B\;|\;A_1^c\cap A_2^c}.$$
Działanie jak wyżej daje pełną informacje o każdym zdarzeniu z ciała generowanego przez zdarzenia $A_1$ i $A_2$. Nazywamy je \buff{rozbiciem} względem $\sigma$-ciała generowanego przez $A_1$ i $A_2$.

Analogicznie możemy zdefiniować $\expected{X\;|\;A}$ dla całkowalnej zmiennej losowej $X$ (tzn. $\expected{|X|}<\infty$):
$$\expected{X\;|\;A}=\int_\Omega X(\omega)\prob{d\omega\;|\;A}=\frac{1}{\prob{A}}\expected{X\mathds{1}_A},$$
gdzie całka wyżej tłumaczy się na całkę po $X$ względem miary $\prob{B\;|\;A}$. 

Uzasadnimy, dlaczego wzór wyżej jest zasadną definicją prawdopodobieństwa warunkowego przy ograniczonej wiedzy o zdarzeniu $A$.

\subsection{Konstrukcja warunkowej wartości oczekiwanej}

Zanim zdefiniujemy \buff{warunkową wartość oczekiwaną [wwo]} zmiennej losowej $X$, zaczniemy od przyjrzenia się bliżej motywacji i konstrukcji stojącej za tym pojęciem.

Niech $Z$ będzie całkowalną zmienną losową przyjmującą przeliczalnie wiele wartości. Zdefiniujmy funkcję
    $$h(z)=\begin{cases}\expected{X\;|\; Z=z}&\prob{Z=z}>0\\0&wpp\end{cases}$$
    oraz zmienną losową $Y=h(Z)$. Weźmy dowolny $C\in Bor(\R)$ i zbadajmy $\expected{Y\mathds{1}_{\{Z\in C\}}}$. Zaczniemy od skorzystania z faktu, że $Z$ przyjmuje przeliczalnie wiele wartości, więc możemy zapisać sumę po nich wszystkich
    \begin{align*}
      \expected{Y\mathds{1}_{\{Z\in C\}}}&=\sum_{z\in C}h(z)\prob{Z=z}=\\
                                         &\overset{\star}{=}\sum_{z\in C}\expected{X\mathds{1}_{\{Z=z\}}}\frac{1}{\prob{Z=z}}\prob{Z=z}=\\
                                         &=\sum_{z\in C}\expected{X\mathds{1}_{\{Z=z\}}}=\\
                                         &=\expected{\sum_{z\in C}X\mathds{1}_{\{Z=z\}}}=\\
                                         &=\expected{X\mathds{1}_{\{Z\in C\}}}
    \end{align*}
    Równość $\star$ wynika ze sposobu w jaki zdefiniowaliśmy $\expected{X}{A}$ w poprzednim podrozdziale.

    Zauważmy, że dowolne zdarzenie $F\in\sigma(Z)$ jest postaci $F=\{z\in C\}$ dla pewnego $C\in Bor(\R)$. Wyprowadziliśmy więc równość:
    $$\expected{Y\mathds{1}_F}=\expected{X\mathds{1}_F}\quad F\in\sigma(Z).$$
    Pozostaje zapytać, co z tej zależności wynika?

    Dla $F=\Omega$ dostajemy
    $$\expected{h(Z)}=\expected{Y}=\expected{X}.$$

    \begin{dygresja}
      W tym miejscu kuszące byłoby rozpisanie $Y=h(Z)$ wprost z definicji, tzn. $h(Z)=\expected{X}{Z=Z}$, ale jest to całkowitą brednią. W definicji funkcji $h$ podanej na samym początku przykładu $z$ jest teoretycznym punkcikiem, natomiast przy definiowaniu $Y=h(Z)$ ów $Z$ jest już obserwowaną przez nas, konkretną zmienną losową. W takim razie, bardziej poprawny byłby zapis
    $$h(Z(\omega))=\expected{X}{\{\omega'\;:\;Z(\omega')=Z(\omega)\}}.$$
    \end{dygresja}

\begin{example}
  \item Ze zbioru $\{1, 2,...,10\}$ losujemy w sposób jednostajny liczbę i oznaczamy ją jako $N$. W drugim losowaniu, również w sposób jednostajny, wybieramy liczbę ze zbioru $\{1,...,N\}$ i nazywamy ją $M$. Chcemy znaleźć średnią wartość liczby $M$. Oczywiście, nie jest trudno zrobić to metodami poznanymi na poprzednich przygodach probabilistycznych, jednak w tym przypadku użyjemy konstrukcji wyżej.

    Funkcja $h$ będzie wyglądać następująco:
    $$h(n)=\expected{M}{N=n}=\sum_{1\leq i\leq n}\frac{i}{n}=\frac{1}{n}\cdot\frac{n(n+1)}{2}=\frac{n+1}{2}$$
    czyli $h(N)=\frac{N+1}{2}$.

    Stosując notację jak wyżej, mamy
    $$\begin{matrix}Z=N\\X=M\end{matrix}$$
    czyli podstawiając do wzoru:
    \begin{align*}
      \expected{M}&=\expected{h(N)}=\expected{\frac{N+1}{2}}=\\
                  &=\frac{1}{2}\left(\expected{N}+1\right)=\frac{1}{2}\left(\sum_{1\leq i\leq 10}\frac{i}{10}+1\right)=\\
                  &=\frac{1}{2}\left(\frac{11}{2}+1\right)\frac{13}{4}
    \end{align*}
\end{example}

Rozbicie jak wyżej można w elegancki sposób zamienić w bardziej abstrakcyjną definicję warunkowej wartości oczekiwanej.

\begin{definition}\label{wwo-definicja}
  Niech $\set{G}\subseteq \set{F}$ będzie $\sigma$-ciałem, a $X$ całkowalną zmienną losową. 

  Zmienną losową $Y$ nazywamy \buff{warunkową wartością oczekiwaną} [wwo] $X$ pod warunkiem $\set{G}$, jeśli następujące warunki są spełnione:
  \begin{itemize}[leftmargin=50pt]
    \item[(W1)] $Y$ jest $\set{G}$-mierzalne
    \item[(W2)] $(\forall\;G\in\set{G})\;\expected{X\mathds{1}_G}=\expected{Y\mathds{1}_G}$
  \end{itemize}
\end{definition}

Nasuwają się teraz pytania o poprawność $Y$ zdefiniowanego jak wyżej. Czy zawsze istnieje i czy jest on jedyny?

\begin{example}
  \item Niech $\set{G}=\sigma(Z)$, gdzie $Z$ jest zmienną losową przyjmującą przeliczalnie wiele wartości. Wówczas $Y=h(Z)$ dla $h(z)=\expected{X}{Z=z}$ jest wwo $X$ względem $\set{G}$.
\end{example}

\begin{theorem}[poprawność wwo]\label{poprawnosc wwo}
Dla $\sigma$-ciała $\set{G}\subseteq\set{F}$ i całkowalnej zmiennej losowej $X$ \buff{istnieje jedyna zmienna losowa $Y$} będąca wwo $X$ względem $\set{G}$. Będziemy ją oznaczać
$$\expected{X}{\set{G}}=Y.$$

    Jeśli $Y, Y'$ są wwo $X$ względem $\set{G}$, to $Y=Y'$ prawie wszędzie.
\end{theorem}

\begin{proof}
  Dowód na następnym wykładzie.
\end{proof}

\begin{uwaga}
  O wwo $X$ pod warunkiem $\set{G}$ należy myśleć jako o przybliżeniu $X$ na podstawie informacji zawartych w $\set{G}$ (więcej na wykładzie 3).
\end{uwaga}

\begin{example}
\item Jeśli $X$ i $\set{G}$ są niezależne, to znaczy dla każdego $B\in Bor(\R)$ i dla każdego $G\in\set{G}$ zachodzi
  $$\prob{X\in B, G}=\prob{X\in B}\prob{G},$$
  to wtedy $\expected{X}{\set{G}}=\expected{X}=Y$.

  Warunek (W1) jest oczywiście spełniony, bo $Y$ jest funkcją stałą, więc jego przeciwobraz to całość lub $\emptyset$ (czyli jest $\set{G}$-mierzalny). Warunek (W2) sprawdzamy dla dowolnego $G\in\set{G}$:
  $$\expected{X\mathds{1}_G}=\expected{X}\expected{\mathds{1}_G}=\expected{\expected{X}\mathds{1}_G}=\expected{Y\mathds{1}_G}.$$

\item Rozważmy pokrycie $\Omega$ rozłącznymi zbiorami $\{A_n\}_{n\in\N}$, gdzie $A_i\in\set{F}$ dla każdego $i$. Niech $\set{G}=\sigma(A_i\;:\;i\in\N)$ będzie $\sigma$-ciałem rozpinanym przez to pokrycie. Wówczas
  $$\expected{X}{\set{G}}=\sum_{i\in\N}\mathds{1}_{A_i}\expected{X}{A_i}$$
  Spełnianie pierwszego warunku jest oczywiste, bo mamy doczynienia z funkcją prostą. Warunek (W2) wystarczy sprawdzić dla atomów, czyli $G=A_i$, bo wszystkie zmienne losowe $\set{G}$-mierzalne są stałe na $A_i$.
  \begin{align*}
    \expected{\left[\sum\mathds{1}_{A_i}\expected{X}{A_i}\right]\mathds{1}_{A_j}}&=\expected{\expected{X}{A_j}\mathds{1}_{A_j}}=\\
                                                                                 &=\expected{\mathds{1}_{A_j}\frac{\expected{X\mathds{1}_{A_j}}}{\prob{A_j}}}=\\
                                                                                 &=\expected{\mathds{1}_{A_j}}\frac{\expected{X\mathds{1}_{A_j}}}{\prob{A_j}}=\expected{X\mathds{1}_{A_j}},
  \end{align*}
  gdyż $\expected{\mathds{1}_{A_j}}=\prob{A_j}$.
\item Jeśli w przykładzie wyżej weźmiemy $A_1=A$, $A_2=A^c$ i $A_3=A_4=...=\emptyset$ oraz $\set{G}=\sigma(A)$, to dostajemy to samo co na samym początku tego wykładu:
  $$\expected{X}{\set{G}}-\mathds{1}_A\expected{X}{A}+\mathds{1}_{A^c}\expected{X}{A^c}.$$
\end{example}

\subsection{Prawdopodobieństwo warunkowe}

\begin{definition}[prawdopodobieństwo warunkowe]\label{prawdopodobienstwo warunkowe}
  Dla $\sigma$-ciała $\set{G}\subseteq\set{F}$ definiujemy \buff{prawdopodobieństwo warunkowe} pod warunkiem $\set{G}$ jako
  $$\prob{A}{\set{G}}=\expected{\mathds{1}_A}{\set{G}}$$
\end{definition}

Prawdopodobieństwo $\prob{A}{\set{G}}$ jest zmienną losową taką, że:
\begin{itemize}
  \item $\prob{A}{\set{G}}$ jest $\set{G}$-mierzalna (ze względu na wwo w definicji)
  \item $\expected{\prob{A}{\set{G}}\mathds{1}_G}=\expected{\mathds{1}_A\mathds{1}_G}=\prob{A\cap G}$
\end{itemize}

\begin{example}
\item Niech $E_1,E_2$ będą niezależnymi zmiennymi losowymi z rozkładem $exp(1)$. Chcemy się zastanowić jak wygląda prawdopodobieństwo
  $$\prob{E_1+E_2>t}{\sigma(E_1)}$$
  dla $t>0$. Ponieważ liczymy to prawdopodobieństwo względem $\sigma(E_1)$, to tak naprawdę wszystkie informacje o $E_1$ mamy w ręku, gdyż tę zmienną obserwujemy. Czyli $E_1$ możemy w takim przypadku potraktować jako zwykłą stałą i zgadnąć, że
  $$\prob{E_1+E_2>t}{\sigma(E_1)}=e^{-(t-E_1)}.$$

  Dla pewności, przerachujemy cały ten przykład wprost z definicji, żeby przekonać się że strzał był poprawny.

  Niech $G\in\sigma(E_1)$, wtedy zgodnie z wcześniejszą obserwacją istnieje pewne $C\in Bor(\R)$ takie, że $G$ jest postaci $G=\{E_1\in C\}$. Policzymy $\expected{\prob{\{E_1+E_2>t\}}{\sigma(E_1)}}$ gdyż jak wyżej zauważyliśmy, $\prob{A}{\set{G}}$ jest zmienną losową. Mamy więc
  \begin{align*}
    \expected{\prob{E_1+E_2>t}{\sigma(E_1)}\mathds{1}_G}&\overset{\star}{=}\prob{\{E_1+E_2>t\}\cap G}=\\
                                                        &=\prob{\{E_1+E_2>t\}\cap\{E_1\in C\}}=\\
                                                        &=\iint\limits_{\substack{C\times\R_+\\x+y>t}}e^{-x}e^{-y}dxdy=\\
                                                        &=\int_Ce^{-x}\underbrace{\left[\int_{x+y>t}e^{-y}dy\right]}_{\star\star}dx=\\
                                                        &=\int_Ce^{-x}e^{-(t-x)_+}dx=\expected{e^{-(t-E_1)_+}\mathds{1}_{\{E_1\in C\}}}=\\
                                                        &=\expected{e^{-(t-E_1)_+}\mathds{1}_G}
  \end{align*}
  Równość $\star$ wynika z uwagi pod definicją prawdopodobieństwa warunkowego. Całka $\star\star$ jest równa $1$ gdy $x>t$ (gdyż wtedy dla każdego $y$ mamy $x+y>t$), natomiast dla $x\leq t$ wynosi ona $e^{-(t-x)}$.
\end{example}
