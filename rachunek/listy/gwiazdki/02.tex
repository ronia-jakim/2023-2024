\documentclass{article}

\usepackage{../../../template}

\title{Zadanie dodatkowe 2}
\author{Weronika Jakimowicz}
\date{15.12.2023}

\pagestyle{empty}

\newenvironment{zadanko}[1]{
  \bfseries{\large\color{orange}Zadanie #1}
}%
{}

\begin{document}
\maketitle
\thispagestyle{empty}

{\bfseries
Niech $\{\xi_k\}$ będzie ciągiem zmiennych iid. o symetrycznym rozkładzie ($\xi_k,-\xi_k$ mają ten sam rozkład). Niech $S_0=0$,
$$S_n=\sum_{k=1}^n\xi_k,\quad k\geq 1.$$
Rozważmy funkcję ogonową $F_k$ zmiennej $S_k$, czyli
$$F_k(x)=\prob{S_k\geq x},\quad x\in\R$$
}

\begin{zadanko}{1}
  Ustalmy $n\in\N$. Uzasadnij, że dla każdego $a\in\R$ ciąg zmiennych losowych 
  $$X_k=F_{n-k}(a-S_k),\quad k=0,2,...,n$$
  jest martyngałem wględem filtracji $\mathds{F}=\{\set{F}_k\}$ danej przez $\set{F}_0=\{\emptyset,\Omega\}$ i $\set{F}_k=\sigma(\xi_1,...,\xi_k)$ dla $k\geq 1$.
\end{zadanko}

Zaczęłam to pisać na ćwiczeniach z listy 4, gdzie studenci prezentujący przy tablicy są bici po rękach jeśli zapomną sprawdzić całkowalności tego co wsadzamy do wwo, w tym całkowalności składników martyngału. Stąd postaram się zrobić to tak dokładnie jak tylko potrafię.

\begin{enumerate}
  \item $X_k$ jest całkowalne, bo jest ograniczone
    $$X_k=F_{n-k}(a-S_k)=\prob{S_{n-k}\geq a-S_k}\in[0,1].$$
  \item $X_k$ jest $\set{F}_k$-mierzalne
    $$X_k=\prob{S_{n-k}\geq a-S_k}=\prob{\sum_{j=1}^{n-k}\xi_j\geq a-\sum_{j=1}^k\xi_j}$$
    ale ponieważ $\xi_i$ są symetryczne, to
    $$\prob{\xi_j\geq x}=\prob{-\xi_j\geq x}$$
    czy da się to uogólnić na dowolną sumę $\xi_j$?
    $$\prob{\xi_j+\xi_i\geq x}=\prob{\xi_j\geq x-\xi_i}=\prob{-\xi_j\geq x-\xi_i}=\prob{-\xi_j+\xi_i\geq x}$$
\end{enumerate}

\end{document}
