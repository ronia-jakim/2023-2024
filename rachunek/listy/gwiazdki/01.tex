\documentclass{article}

\usepackage{../../../template}

\title{Zadanie dodatkowe 1}
\author{Weronika Jakimowicz}
\date{30.10.2023}

\newcommand{\E}{\mathbb{E}}
\renewcommand{\P}{\mathbb{P}}

\pagestyle{empty}
\begin{document}
\maketitle
\thispagestyle{empty}

Niech $(\Omega,\set{F})$ będzie przestrzenią mierzalną i niech $\set{G}$ będzie pod$\sigma$-ciałem $\set{F}$. Niech $\P$ i $\mathbb{Q}$ będą równoważnymi miarami probabilistycznymi na $(\Omega,\set{F})$. Dokładniej $\P$ jest absolutnie ciągła względem $\mathbb{Q}$ (na $\set{F}$) i $\mathbb{Q}$ jest absolutnie ciągła względem $\P$ (na $\set{F}$). Oznaczmy przez $X_0$ pochodną Radona-Nikodyma $\mathbb{Q}$ względem $\P$ na $\set{F}$. Przez $\mathbb{E}_\P$ i $\mathbb{E}_{\mathbb{Q}}$ oznaczać będziemy wartość oczekiwaną wyznaczoną odpowiednią przez $\P$ i $\mathbb{Q}$.
\bigskip

{\bfseries{\large\color{orange}Zadanie 1.}

  Uzasadnij, że $\mathbb{E}_\mathbb{P}[X_0\;|\;\set{G}]>0$ $\P$-p.w.
}
\medskip

Zacznijmy od małej powtórki miary i całki, czyli powiedzenia, że jeśli $\P$ i $\Q$ są miarami równoważnymi, to $X_0=\frac{d\Q}{d\P}>0$ prawie wszędzie. Rozważmy zbiór 
$$A=\{\omega\;:\;X_0(\omega)\leq 0\}.$$

Wówczas
$$\Q(A)=\int_A X_0d\P\leq 0$$
gdyż całkujemy funkcję na zbiorze, na którym przyjmuje ona wyłącznie niedodatnie wartości. Z drugiej strony, $\Q(A)\geq 0$, czyli $\Q(A)=0$, a ponieważ $\P\ll \Q$, to również $\P(A)=0$.

Powiedzenie teraz, że skoro $X_0>0$ to $\E_{\P}[X_0\;|\;\set{G}]>0$ wynika teraz wprost z własności wwo.


%Rozważmy zbiór $A=\{\omega\;:\;\E_\P[X_0\;|\;\set{G}](\omega)\leq 0\}$. Wówczas
%\begin{align*}
%  0{\geq}\int_A\max_{\omega\in A}\E_\P[X_0\;|\;\set{G}](\omega)\;d\P=&\\
%  =\int_A\E_\P[X_0\;|\;\set{G}]\;d\P={\color{blue}\E_\P[\E_\P[X_0\;|\;\set{G}]\mathds{1}_A]}&{\color{blue}=\E_\P[X_0\mathds{1}_A]=}\\
%                                                                              &=\int_AX_0\;d\P=\int_A\frac{d\Q}{d\P}d\P=\\
%                                                                              &=\int_Ad\Q=\Q(A)\geq0
%\end{align*}
%
%Skoro więc $0\geq \E_\P[\E_\P[X_0\;|\;\set{G}]\mathds{1}_A]\geq 0$, to $\E_\P[\E_\P[X_0\;|\;\set{G}]\mathds{1}_A]=0$, a ponieważ jest to całka po zbiorze na którym funkcja $\E_\P[X_0\;|\;\set{G}]$ nie zmienia znaku, to zbiór $A$ musi być miary zero, co jest chcianym przez nas rezultatem.
\bigskip

{\bfseries{\large\color{orange}Zadanie 2.}

  Pokaż, że dla każdej ograniczonej, $\set{F}$-mierzalnej zmiennej $Y$,
  $$\E_\P[YX_0\;|\;\set{G}]=\E_{\Q} [Y\;|\;\set{G}]\E_{\P}[X_0\;|\;\set{G}]$$
}
\medskip

Zacznijmy od zauważenia, że $X_0'=\E_\P[X_0\;|\;\set{G}]$, gdzie $X_0'$ jest pochodna Radona-Nikodyma $\Q$ względem $\P$ na $\set{G}$. Z definicji $X_0'$ jest $\set{G}$-mierzalne, wystarczy więc sprawdzić (W2) w definicji wykładowej. Niech więc $G\in\set{G}$
\begin{align*}
  \E_{\P}[\E_{\P}[X_0\;|\;\set{G}]\mathds{1}_G]&=\E_{\P}[X_0\mathds{1}_G]=\int_GX_0d\P=\Q(G)
\end{align*}
ale ponieważ $X_0'$ jest pochodną na ciele $\set{G}$, a my jesteśmy w $\set{G}$, to z drugiej strony
$$\Q(G)=\int_GX_0'd\P.$$

Przechodząc już do sedna sprawy, przypomnijmy sobie, razem z Wikipedią, że jeśli $g$ jest $\Q$-całkowalną funkcją, to
$$\int_Ag\;d\Q=\int_Ag\frac{d\Q}{d\P}\;d\P\implies \E_{\Q}[g]=\E_{\P}[g\frac{d\Q}{d\P}]$$

%Doprowadźmy równanie docelowe do troszkę przyjemniejszej postaci, tzn. wwo względem jednej miary będę po tej stamej stronie:
%$$\E_{\Q}[Y\;|\;\set{G}]=\frac{\E_{\P}[YX_0\;|\;\set{G}]}{\E_{\P}[X_0\;|\;\set{G}]}$$
%co możemy zrobić dzięki zadaniu 1 ($\E_{\P}[X_0\;|\;\set{G}]\neq 0$ $\P$-p.w.).

Ustalmy $G\in\set{G}$

%\begin{align*}
%  \E_{\Q}\left[
%  \frac{\E_{\P}[YX_0\;|\;\set{G}]}{\E_{\P}[X_0\;|\;\set{G}]}\mathds{1}_G\right]&=
%  \int_G \frac{\E_{\P}[YX_0\;|\;\set{G}]}{X_0'}\;d\Q=\\
%                                                                   &=\int_G\frac{\E_{\P}[YX_0\;|\;\set{G}]}{X_0'}X_0'\;d\P=\\
%                                                                   &=\int_G\E_{\P}[YX_0\;|\;\set{G}]\;d\P=\\
%                                                                   &=\E_{\P}[\E_{\P}[YX_0\;|\;\set{G}]\mathds{1}_G]=\\
%                                                                   &=\E_\P[YX_0\mathds{1}_G]=\\
%                                                                   &=\int_GYX_0\;d\P
%\end{align*}

\begin{align*}
  \E_{\Q}[\E_{\P}[YX_0\;|\;\set{G}]\mathds{1}_G]&\overset{\star\star}{=}\E_{\P}[\E_{\P}[YX_0\;|\;\set{G}]X_0'\mathds{1}_G]\\
        &\overset{\star}{=}\E_{\P}[\E_{\P}[YX_0X_0'\;|\;\set{G}]\mathds{1}_G]=\\
        &=\E_{\P}[YX_0X_0'\mathds{1}_G]=\\
        &\overset{\star\star}{=}\E_{\Q}[YX_0'\mathds{1}_G]=\\
        &=\E_{\Q}[\E_{\Q}[YX_0'\;|\;\set{G}]\mathds{1}_G]=\\
        &\overset{\star}{=}\E_{\Q}[\E_{\Q}[Y\;|\;\set{G}]X_0'\mathds{1}_G]=\E_{\Q}[\E_{\Q}[Y\;|\;\set{G}]\E_{\P}[X_0\;|\;\set{G}]\mathds{1}_G]
\end{align*}

Wystarczy więc uzasadnić równości zaznaczone gwiazdką i dowód jest gotowy.

Przejścia $\star\star$ najpierw "wypluwają" $X_0'$, żeby zamienić całkę na $G\in\set{G}$ względem $\Q$ na całkę względem $\P$, a potem zjadają $X_0$ żeby zamienić całkę względem $\P$ na całkę względem $\Q$ (tutaj myślimy już w kontekścię całego $\set{F}$, bo i tak nie ma to różnicy gdy jesteśmy na $G\in\set{G}$).

Przejścia oznaczone $\star$ wymagają, żeby $YX_0X_0'$ było całkowalne względem $\P$ oraz żeby $YX_0'$ było całkowalne względem $\Q$. 
%Od razu przypomnijmy, ponownie z pomocą Wikipedii, że
%$$\frac{d|\Q|}{d\P}=\left|\frac{d\Q}{d\P}\right|$$
%a ponieważ $\Q(\omega)\in[0,1]$, to w naszym przypadku
%$$\frac{d\Q}{d\P}=\left|\frac{d\Q}{d\P}\right|$$

%Z uwagi wyżej wynika więc, że $|X_0|=X_0$ oraz $|X_0'|=X_0'$. Spróbujmy teraz pokazać, że $\E_{\P}[|YX_0X_0']]<\infty$, oznaczając $M=\sup_{\omega\in\Omega} |Y(\omega)|$
W pierwszym zadaniu powiedzieliśmy już, że $X_0>0$, a ponieważ $X_0'$ też jest pochodną Radona-Nikodyma, to $X_0'>0$ ze względu na ten sam argument.

Niech więc $M$ będzie taką stałą, że dla każdego $\omega$ $|Y(\omega)|\leq M$. Wówczas
\begin{align*}
  \E_{\P}[|YX_0X_0'|]&\leq M\E_{\P}[X_0X_0']=M\int X_0'\frac{d\Q}{d\P}d\P=M\int X_0'\;d\Q=\\
                     &=M\int\E_{\P}[X_0\;|\set{G}]d\Q=M\E_{\P}[X_0]=M\cdot 1<\infty
\end{align*}
gdyż $\E_{\P}[X_0]=\int_{\Omega}X_0\;d\P=\Q(\Omega)=1$. Korzystając z tych wyliczeń dostajemy analogiczny wynik dla $YX_0'$:
$$\E_{\Q}[YX_0']\leq M\E_{\Q}[X_0']=M\E_{\P}[X_0'X_0]<\infty$$
\bigskip

{\bfseries{\large\color{orange}Zadanie 3.}

  Załóżmy, że $X'=GX_0$ dla pewnej ograniczonej $\set{G}$-mierzalnej zmiennej losowej $G$. Pokaż, że spełnione są
  \begin{enumerate}[label=(\alph*)]
    \item $X'$ jest $\P$-całkowalna
    \item Dla każdej ograniczonej, $\set{F}$-mierzalnej zmiennej $Y$,
      $$\E_{\P}[YX'\;|\;\set{G}]=\E_{\Q}[Y\;|\;\set{G}]\E_{\P}[X'\;|\;\set{G}]$$
  \end{enumerate}
}

\textbf{\color{green}(a)}

Niech $M$ będzie takie, że $M\geq |G(\omega)|$ dla każdego $\omega\in\Omega$. Stosując to samo rozumowanie co wyżej dostajemy:
\begin{align*}
  \E_{\P}[|GX_0|]\leq M\E_{\P}[X_0]=M\int\frac{d\Q}{d\P}\;d\P=M\int d\Q=M\cdot 1=M<\infty
\end{align*}

\textbf{\color{green}(b)}

Po pierwsze zauważmy, że skoro $Y$ jak i $G$ są ograniczone, to również $YG$ są ograniczone. Wprost z poprzedniego zadania dostajemy
$$\E_{\P}[(YG)X_0\;|\;\set{G}]=\E_{\Q}[(YG)\;|\;\set{G}]\E_{\P}[X_0\;|\;\set{G}]$$
I wystarczy pokazać, że 
$$\E_{\Q}[YG\;|\;\set{G}]\E_{\P}[X_0\;|\;\set{G}]=\E_{\Q}[Y\;|\;\set{G}]\E_{\P}[GX_0\;|\;\set{G}]$$

Zauważmy, że
$$\E_{\Q}[|YG|]=\int|YG|\;d\Q\leq \int M\;d\Q=M<\infty$$
jeśli $M=\max|YG|$, czyli możemy użyć 6 własności wwo (na którą już się powoływałam), żeby dostać
$$\E_{\Q}[YG\;|\;\set{G}]=G\E_{\Q}[Y\;|\;\set{G}].$$
Wiemy też, że $\E_{\P}[|GX_0|]<\infty$, czyli można również napisać
$$G\E_{\P}[X_0\;|\;\set{G}]=\E_{\P}[GX_0\;|\;\set{G}].$$
Łącząc oba te kroki dostajemy
\begin{align*}
  \E_{\Q}[YG\;|\;\set{G}]\E_{\P}[X_0\;|\;\set{G}]&=G\E_{\Q}[Y\;|\;\set{G}]\E_{\P}[X_0\;|\;\set{G}]=\\
      &=\E_{\Q}[Y\;|\;\set{G}]\E_{\P}[GX_0\;|\;\set{G}]
\end{align*}
\bigskip

{\bfseries{\large\color{orange}Zadanie 4.}
  
  Załóżmy, że zmienna $X'$ spełnia warunki (a) i (b). Pokaż, że $X'=GX_0$ dla pewnej $\set{G}$-mierzalnej zmiennej $G$.
}
\medskip

%Niech $Y$ będzie $\set{G}$-mierzalne, wówczas $\E_{\Q}[Y\;|\;\set{G}]=Y$, czyli
%\begin{align*}
%  \E_{\P}[YX'\;|\;\set{G}]&=\E_{\Q}[Y\;|\;\set{G}]\E_{\P}[X'\;|\;\set{G}]=Y\E_{\P}[X'\;|\;\set{G}]
%\end{align*}
%
%Wtedy mamy dla $G\in\set{G}$
%\begin{align*}
%  \E_{\P}[\E_{\P}[X'\;|\;\set{G}]Y\mathds{1}_G]=\E_{\P}[\E_{\P}[YX'\;|\;\set{G}]\mathds{1}_G]&=\E_{\P}[YX'\mathds{1}_G]
%\end{align*}
%
%Wiem, że $\int_A X_0 d\P=\Q(A)$, czyli
%$$\int_AGX_0d\P=\int_AGd\Q=\E_{\Q}[G\mathds{1}_A]$$
%czyli 
%$$\mu(A)=\frac{\E_{\Q}[G\mathds{1}_A]}{\E_{\Q}[G]}$$
%gdy $\E_{\Q}[G]\neq 0$ jest miarą. I w dodatku $\mu<<\Q$ oraz $\mu <<\P$.



Z zadania drugiego łatwo otrzymać
%$$\E_{\P}[YX'\;|\;\set{G}]\E_{\P}[X_0\;|\;\set{G}]=\E_{\P}[YX_0\;|\;\set{G}]\E_{\P}[X'\;|\;\set{G}]$$
%czyli
$$\frac{\E_{\P}[YX'\;|\;\set{G}]}{\E_{\P}[X'\;|\;\set{G}]}=\frac{\E_{\P}[YX_0\;|\;\set{G}]}{\E_{\P}[X_0\;|\;\set{G}]}$$

W takim razie musi istnieć niezerowa funkcja $f$ taka, że
$$\begin{cases}{\E_{\P}[YX'\;|\;\set{G}}={f\cdot\E_{\P}[YX_0\;|\;\set{G}]}\\
{\E_{\P}[X'\;|\;\set{G}]}={f\cdot\E_{\P}[X_0\;|\;\set{G}]}\end{cases}$$
%co w szczególności oznacza, że
%$$\E_{\P}[YX'\;|\;\set{G}]=f\cdot\E_{\P}[YX_0\;|\;\set{G}]$$
i ponieważ zadanie pierwsze mówi, że $\E_{\P}[X_0\;|\;\set{G}]>0$ prawie wszędzie, to
$$\frac{\E_{\P}[X'\;|\;\set{G}]}{\E_{\P}[X_0\;|\;\set{G}]}=f$$
czyli $f$ jest funkcją $\set{G}$-mierzalną jako iloczyn dwóch $\set{G}$-mierzalnych funkcji z dzielnikiem $>0$ prawie zawsze.

Chcemy teraz sprawdzić, czy $\E_{\P}[|f\cdot YX_0|]<\infty$
\begin{align*}
  \E_{\P}[|fYX_0|]&\leq M\E_{\P}[|fX_0|]=M\E_{\P}[|f|X_0]=\\ 
                  &=M\E_{\Q}[|f|]=M\E_{\P}[|f|X_0']=\\ 
                  &=M\E_{\P}[|f\E_{\P}[X_0\;|\;\set{G}]|]=\\ 
                  &=M\E_{\P}[|\E_{\P}[X'\;|\;\set{G}]|]\leq \\ 
                  &\leq M\E_{\P}[\E_{\P}[|X'|\;|\;\set{G}]]=M\E_{\P}[|X'|]\infty
\end{align*}

%Zauważmy teraz, że jeśli napiszemy
%$$\E_{\P}[YX'\;|\;\set{G}]=f\cdot\E_{\P}[YX_0\;|\;\set{G}]$$
%i nałożymy na obie strony $\E_{\P}[|\cdot|]$, to dostaniemy
%\begin{align*}
%  \infty>M\E_{\P}[|X'|]\geq\E_{\P}[|YX'|]&=\\ 
%  =\E_{\P}[\E_{\P}[|YX'|\;|\;\set{G}]]&\geq\\ 
%  \geq \E_{\P}[|\E_{\P}[YX'\;|\;\set{G}]|]&=\E_{\P}[|f\cdot\E_{\P}[YX_0\;|\;\set{G}]|]=\\
%                                          &=\E_{\P}[|f\E_{\Q}[Y\;|\;\set{G}]||X_0'|]=\E_{\P}[|fY|X_0']=\\ 
%                                          &=\E_{\Q}[|f\E_{\Q}[Y\;|\;\set{G}]|]=\E_{\P}[|f\E_{\Q}[Y\;|\;\set{G}]|X_0]=\\ 
%                                                                                                &=\E_{\P}[|fYX_0|]
%\end{align*}
%dla $M$ jak zawsze do tej pory, tzn. $M\geq|Y(\omega)|$ dla każdego $\omega\in\Omega$.

%Zauważmy teraz, że
%$$\E_{\P}[|\E_{\P}[X'\;|\;\set{G}|]=\E_{\P}[|X'|]<\infty$$
%gdyż $X'$ spełnia warunek (a). Z drugiej strony,
%$$\E_{\P}[|\E_{\P}[X_0\;|\;\set{G}]f|]=\E_{\P}[|X_0'\cdot f|]=\E_{\Q}[|f|]=\E_{\P}|[X_0f|]$$
%a ponieważ obie jest to nałożenie $\E_{\P}[|\cdot|]$ na obie strony równości wyżej, to wiemy, że $\E_{\P}[|X_0f|]<\infty$.

Po dokonaniu tej formalności, możemy wciągnąć $f$ do środka wwo:
$$\E_{\P}[YX'\;|\;\set{G}]=\E_{\P}[fYX_0\;|\;\set{G}]$$
co daje, że
$$0=\E_{\P}[YX'\;|\;\set{G}]-\E_{\P}[fYX_0\;|\;\set{G}]=\E_{\P}[Y(X'-fX_0)\;|\;\set{G}]$$
dla dowolnej ograniczonej $\set{F}$-mierzalnej funkcji $Y$. Możemy teraz wybrać
$$Y=\begin{cases}1&X'-fX_0\geq 0\\ -1&X'-fX_0<0\end{cases}$$
co jest funkcją mierzalną jako kombinacja dwóch funkcji prostych.

Wystarczy teraz zauważyć, że $Y(X'-fX_0)=|X'-fX_0|$, czyli
$$0=\E_{\P}[\E_{\P}[|X'-fX_0|\;|\;\set{G}]]=\E_{\P}[|X'-fX_0|]$$
funkcja $|X'-fX_0|$ nie zmienia znaku, więc skoro całka z niej jest zerem, to $|X'-fX_0|=0$ prawie wszędzie, czyli $X'=fX_0$ prawie wszędzie.

%a więc całka z $X'$ i z $fX_0$ zgadza się na każdym zbiorze z $\set{G}$, czyli $X'=fX_0$ $\P$-p.w. {\scriptsize\itshape(całkiem szczerze, to mam wrażenie że gdzieś po drodze coś źle zakładam)}
\bigskip

{\bfseries{\large\color{orange}Zadanie 5.}

  Znajdź warunek konieczny i dostateczny (w terminach $X_0$) na to, aby
  $$\E_{\P}[Y\;|\;\set{G}]=\E_{\Q}[Y\;|\;\set{G}]$$
  dla każdej ograniczonej zmiennej $Y$.
}

Warunek konieczny to będzie $p$ takie, że $\E_{\P}[Y\;|\;\set{G}]=\E_{\Q}[Y\;|\;\set{G}]\implies p$. Warunek dostateczny to z kolei będzie $q$ taki, że $q\implies \E_{\P}[Y\;|\;\set{G}]=\E_{\Q}[Y\;|\;\set{G}]$.

Próbując znaleźć warunek dostateczny ciekawszy niż $X_0\equiv 1$, z przykrością stwierdziłam, że może on być praktycznie taki sam jak warunek konieczny. Z tego powodu, mogę napisać, że

\begin{center}\slshape
  $X_0$ jest $\set{G}$-mierzalny $\iff$ $\E_{\P}[Y\;|\;\set{G}]=\E_{\Q}[Y\;|\;\set{G}]$ dla każdego ograniczonej zmiennej $Y$.
\end{center}

$\implies$ jest dość prostym kierunkiem implikacji. Jeśli $X_0$ jest $\set{G}$-mierzalne, to $\E_{\P}[X_0\;|\;\set{G}]=X_0$. To, połączone z wcześniej już zauważonym faktem, że $\E_{\P}[|YX_0|]<\infty$ daje
$$X_0\E_{\P}[Y\;|\;\set{G}]=\E_{\P}[YX_0\;|\;\set{G}]=\E_{\Q}[Y\;|\;\set{G}]\E_{\P}[X_0\;|\;\set{G}]=\E_{\Q}[Y\;|\;\set{G}]X_0$$
Wystarczy teraz przypomnieć sobie, że $X_0>0$ jeśli miary $\Q$ i $\P$ są sobie równoważne (co jest prawdą w tym zadaniu), aby dostać pozwolenie na podzielenie obu stron równości przez $X_0$ i otrzymanie
$$\E_{\P}[Y\;|\;\set{G}]=\E_{\Q}[Y\;|\;\set{G}].$$

$\impliedby$ jest bardzo podobny do tego co się stało w zadaniu 4, tzn. zaczynamy od
$$\E_{\P}[Y\;|\;\set{G}]=\E_{\Q}[Y\;|\;\set{G}]=\frac{\E_{\P}[YX_0\;|\;\set{G}]}{\E_{\P}[X_0\;|\;\set{G}]}$$
biorąc dowolny $G\in\set{G}$ i całkując obie strony na $G$ względem miary $\Q$ dostajemy
\begin{align*}
  \E_{\P}[YX_0'\mathds{1}_G]=\E_{\Q}[Y\mathds{1}_G]=\E_{\Q}[\E_{\Q}[Y\;|\;\set{G}]\mathds{1}_G]&=\E_{\Q}\left[ \frac{\E_{\P}[YX_0\;|\;\set{G}]}{\E_{\P}[X_0\;|\;\set{G}]}\mathds{1}_G \right]=\\
                                                                    &=\E_{\P}\left[ \frac{\E_{\P}[YX_0\;|\;\set{G}]}{X_0'}X_0'\mathds{1}_G \right]=\\ 
                                                                    &=\E_{\P}[YX_0\mathds{1}_G]
\end{align*}
Ponieważ $G\in\set{G}$ był wybrany dowolnie, to
$$\E_{\P}[YX_0\;|\;\set{G}]=\E_{\P}[YX_0'\;|\;\set{G}]\implies \E_{\P}[YX_0-YX_0'\;|\;\set{G}]=0$$
a więc
$$0=\int YX_0-YX_0'd\P=\int Y(X_0-X_0')d\P$$
biorąc
$$Y=\begin{cases}1&X_0-X_0'\geq0\\-1&X_0-X_0'<0\end{cases}$$
mamy $Y(X_0-X_0')=|X_0-X_0'|$, a więc
$$0=\int |X_0-X_0'|d\P\implies X_0=X_0'\;\P-p.w.$$
Wiedząc, że $X_0'$ było $\set{G}$-mierzalne dowiadujemy się w taki sposób, że $X_0$ też jest $\set{G}$-mierzalne.


\end{document}
