\section{12.03.24 : }

Homografia, dla której określone $3$ punkty
$$(z_1,z_2, z_3)\mapsto (w_1, w_2, w_3)$$
jest określona wzorem 
$$\frac{w-w_1}{w-w_3}\frac{w_2-w_3}{w_2-w_1}=\frac{z-z_1}{z-z_3}\frac{z_2-z_3}{z_2-z_1}$$

Wyrażenie 
$$\frac{z-z_1}{z-z_3}\frac{z_2-z_3}{z_2-z_1}$$ 
nazywamy dwustosunkiem czterech liczb: $z$, $z_1$, $z_2$ i $z_3$. Homografia zawsze zachowuje dwustosunek.

\begin{example}
  \item Znaleźć homografię przekształcającą $z_1=0$, $z_2=i$ i $z_3=1$ na punkty $w_1=1$, $w_2=0$ i $w_3=-1$.
    \begin{align*}
  \frac{w-1}{w+1}\cdot \frac{1}{-1}=\frac{z}{z-1}\frac{i-1}{i}
    \end{align*}
    która w formie $w=(\alpha z+\beta)/(\gamma z+\delta)$ ma postać
    $$w=\frac{-iz-1}{(2+i)z-1}$$
  \item \acc{Jeden z punktów $z_i$, $w_i$ może przyjmować wartość $\infty$}. Sztuczka jest taka, żeby wartość $\infty$ upakować do mianownika. Na przykład dla $(z_1, z_2, z_3)=(1, 0, -1)$ i $(w_1, w_2, w_3)=(i, 1, \infty)$ przekształcamy oryginalny wzór na
    $$\frac{w-w_1}{\frac{w}{w_3}-1}\frac{\frac{w_2}{w_3}-1}{w_2-w_1}=\frac{z-z_1}{z-z_3}\frac{z_2-z_3}{z_2-z_1}$$
    i po podstawieniu dostajemy
    $$\frac{w-i}{-1}\frac{-1}{1-i}=\frac{z-1}{z+1}\frac{1}{-1}$$
    co po uporządkowaniu zapisuje się jako
    $$w=\frac{(-1+2i)z+1}{z+1}.$$

    W inny sposób do tego wzoru możemy dojść zaczynając od
    $$w=\frac{\star}{z+1},$$
    bo wiemy że $z=-1$ ma przejść na $\infty$, czyli musimy w tym punkcie mieć $0$ w mianowniku. Dalej, wiemy że gdy $z=0$, to wynik ma wyjść $1$. Mianownik już wynosi $1$, czyli licznik musi być postaci $\alpha z+1$:
    $$w=\frac{\alpha z+ 1}{z+1}.$$
    Podstawiając ostatnią wartość, dostajemy
    $$i=\frac{\alpha+1}{2}\implies 2i=\alpha+1$$
    co daje wzór zgadzający się z tym otrzymanym wyżej.
\end{example}

\begin{definition}
  Mówimy, że dwa punkty są odwrotne względem okręgu, gdy leżą na jednej półprostej ze środka okręgu (przedłużeniu promienia) oraz iloczyn ich odległości od środka okręgu wynosi $r^2$, gdzie $r$ to promień okręgu.
\end{definition}

Graficznie, wygląda to:
  \begin{center}\begin{tikzpicture}
    \draw (0,0) circle (2);
    \draw(0,0)--(20:6);
    \fill (20:1) circle (2pt);
    \fill (20:4) circle (2pt);
    \fill (0,0) circle (2pt);
    \node at (8:1) {$z''$};
    \node at (15:4) {$z'$};
    \draw[dashed] (0,0)--(5, 0);
    \draw (0,0)--(0, 2);
    \node at (-0.3, 1) {$R$};
  \end{tikzpicture}\end{center}
gdzie $|z'-z_0||z''-z_0|=R^2$.

\begin{theorem}
  Rozważamy homografię $\frac{\alpha z+\beta}{\gamma z+\delta}$. Jeśli punkty $z'$ i $z''$ są odwrotne względem okręgu, to ich obrazy $w'$ i $w''$ są odwrotne względem obrazu tego okręgu.
\end{theorem}

\begin{proof}
  Homografia jest złożeniem przekształceń afinicznych $w=Az+B$ oraz przekształcenia $w=\frac{1}{z}$. Własność z tezy zachowuje się przy $w=Az+B$ (to jest przesunięcie lub obrót, widać).

  Dla $w=\frac{1}{z}$ sprawdzimy własność z tezy tylko dla okręgów o środku w $0$. Niech $z'=\rho e^{i\theta}$ i $z''=\frac{R^2}{\rho}e^{i\theta}$ będą punktami odwrotnymi względem badanego okręgu. Ich obrazy to
  $$w'=\frac{1}{\rho}e^{-\theta}$$
  $$w''=\frac{\rho}{R^2}e^{-i\theta}.$$
  Ich odległości od środka okręgu (który jest środkiem układu współrzędnych) to odpowiednio $\frac{1}{\rho}$ i $\frac{rho}{R^2}$, czyli wydzielenie pierwszej przez drugą daje nadal $R^2$.

  Pozostaje pytanie, co zrobić dla okręgu o środku $z_0\neq 0$?
\end{proof}

Pojawia się teraz problem znalezienia wszystkich homografii przekształcających górną półpłaszczyznę (tzn. punkty $z$ o $\Im(z)\geq 0$) na koło jednostkowe o środku w $0$ (punkty $|w|leq 1$).

Prosta $\Im(z)=0$ musi przejść na okrąg $|w|=1$, bo wpp. gdyby $\Im(z)=0$ przeszło na inny okrąg, to górna półpłaszczyzna musiałaby wypełniać ten okrąg (nie może wychodzić poza). Można to pokazać bez argumentów topologicznych, patrząc na odwzorowanie odwrotne.

Zatem każdy punkt prostej $\Im(z)=0$ przechodzi na $|w|=1$.

Zbadamy trzy punktu: $z=1$, $z=0$ i $z=\infty$ i ich obrazy przez standardową homografię. Podstawiając $0$ dostajemy $|\beta|=|\gamma|$, podstawiając $\infty$ mamy dodatkowo $|\alpha|=|\gamma|$ i po podstawieniu $1$ ostatnia równość to  $|\alpha+\beta|=|\gamma+\delta|$.

Jeśli $\alpha=0$, to również $\gamma=0$ i wtedy mamy funkcję stałą - sprzeczność. W takim razie możemy dzielić przez te liczby. Mamy funkcję
$$w=\frac{\alpha}{\gamma}\frac{z+\frac{\beta}{\alpha}}{z+\frac{\delta}{\gamma}}$$
Przyjmując $\frac{\alpha}{\gamma}=e^{i\theta_0}$, $z_0=-\frac{\beta}{\alpha}$ i $z_1=-\frac{\delta}{\gamma}$ dostajemy
$$w=e^{i\theta_0}\frac{z-z_0}{z-z_1}$$
podstawiając $z=0$ otrzymujemy
$$|z_0|=|z_1|$$
Podstawiając $1$ dostaniemy 
$$|1-z_0|=|1-z_1|$$

{\large\color{red}COŚTAM PISANE BYŁO}

Można sprawdzić, że każda homografia przekształcająca koło jednostkowe w siebie ma postać $z=e^{i\theta_0}\frac{z-a}{1-a\overline{z}},$ gdzie $|a|<1$.

\subsection{Całkowanie}

Całkujemy funkcje zespolone względem krzywych na płaszczyźnie.

Jeśli $\gamma:[\alpha, \beta]\to\C$ jest funkcją ciągłą oraz $\gamma(t)=X(t)+iY(t)$, to całkę 
$$\int_\alpha^\beta \gamma(t)dt:=\int_\alpha^\beta X(t)dt+i\int_\alpha^\beta Y(t)dt$$

Kierunek przeciwny do ruchu wskazówek zegara przyjeło się nazywać dodatnią orientacją na krzywej (zamkniętej).

Rozważamy podział $P$ krzywej $C$ o początku $a$ i końcu $b$ (potencjalnie $a=b$), $P=\{z_0=a, z_1,..., z_n=b\}$. Na każdym fragmencie wybieramy punkty pośrednie $z_k^*$ leżące odpowiednio na każdym fragmencie od $z_{k-1}$ do $z_k$. Definiujemy sumy całkowe
$$S_P=\sum_{k=1}^nf(z_k^*)\Delta z_k$$
gdzie $S_P$ zależy nie tylko od wyboru podziału $P$, ale też wyboru punktów pośrednich.

Jeśli liczby $S_P$ leżą "blisko" jakiejś liczby $L$, to przyjmujemy, że $\int_C f(z)dz=L$. "Bliskość" oznacza, że dla dowolnej liczby $\varepsilon>0$ istnieje podział $P_\varepsilon$ taki, że dla podziałów drobniejszych $P\supseteq P_\varepsilon$ mamy $|S_P-L|<\varepsilon$

\begin{fact}[własności całki]
  liniowość: $\int_Ccf(z)dz=c\int_Cf(z)dz$ i $\int_C(f(z)+g(z))dz=\int_Cf(z)dz+\int_Cg(z)dz$

  jest też liniowość ze względu na krzywą (o ile kierunek się zgadza)
\end{fact}
