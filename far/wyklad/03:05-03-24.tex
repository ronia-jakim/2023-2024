\section{05.03.24 : }

W logarytmie głównym decydujemy się na kąty $(-\pi, \pi)$

\begin{center}
  \begin{tikzpicture}
    \draw (0, -3)--(0, 3);
    \draw (-3, 0)--(3, 0);
    \draw (0,0)--(30:2.5);
    \draw[->] (0:1) arc (0:30:1);

    \draw (0,0)--(-40:2.5);
    \draw[->](0:1) arc(0:-40:1);
  \end{tikzpicture}
\end{center}

$$\Log(z)=\log\sqrt{|z|}+i\arg(z)$$

Sprawdzamy różniczkowalność logarytmu funkcji $\Log z$ przy pomocy warunków Cauchy'ego-Riemmana:
$$\begin{matrix}
  \frac{\partial u}{\partial x}=\frac{x}{x^2+y^2} & \frac{\partial v}{\partial x}=-\frac{y}{x^2+y^2}\\ 
  \frac{\partial u}{\partial y}=\frac{y}{x^2+y^2} & \frac{\partial v}{\partial y}=\frac{x}{x^2+y^2}
\end{matrix}$$
z czego widać, że poza $0$ funkcja spełnia warunki C-R. Powinniśmy jeszcze sprawdzić ciągłość pochodnych cząstkowych, ale tutaj nie ma problemu.

Chcemy teraz sprawdzić, czy $(\Log z)'=\frac{1}{z}$, analogicznie jak było w $\R$:
\begin{align*}
  (\Log z)'&=\frac{\partial u}{\partial x}+i\frac{\partial v}{\partial x}=\frac{x}{x^2+y^2}-i\frac{y}{x^2+y^2}=\\ 
           &=\frac{x-iy}{(x+iy)(x-iy)}=\frac{1}{x+iy}=\frac{1}{z}
\end{align*}

Wypadałoby jeszcze sprawdzić, czy $e^{\Log z}=z$

\begin{definition}[funkcja potęgowa]
  Dla dowolnego $c\in\C$, $c\neq 0$, definiujemy 
  $$z^c=e^{c\Log z}$$
  i nazywamy częścią główną.
\end{definition}

\begin{example}
  \item Obliczyć część główną $(-i)^i$
    \begin{align*}
      (-i)^i&=e^{i\Log(-i)}=e^{i(-i\frac{\pi}{2})}=e^{\frac{\pi}{2}}
    \end{align*}
\end{example}

\subsection{Geometria funkcji elementarnych}

Będziemy teraz badać odwzorowania $w=f(z)$.

\begin{example}
\item $w=z+c$, czyli translacja o wektor $c$
\item $w=bz$, dla $b\neq 0$, jeśli zapiszemy $b=r_0(\cos \theta_0+i\sin \theta_0)$ to zobaczymy, że
  $$w=bz=rr_0(\cos(\theta+\theta_0)+i\sin(\theta+\theta_0))$$
  jest złożeniem obrotu o $\theta_0$ i jednokładności względem zera w skali $r_0$.
\item Dla funkcji $w=(1+i)z+(2-i)$ znaleźć obraz prostokąta 
  \begin{center}
    \begin{tikzpicture}
      \draw (0, -1.5)--(0, 3.5);
      \draw (-0.5, 0)--(3, 0);

    \fill (1, 0) circle (2pt);
  \fill(0, 0) circle (2pt);
\fill (1, 2) circle (2pt);
\fill(0, 2) circle (2pt);

\draw (0,0)--(1,0)--(1,2)--(0,2)--(0,0);

\fill [green] (0,1) circle (2pt);
\fill [green] (1, 2) circle (2pt);
\fill[green] (3, 0) circle (2pt);
\fill[green] (2, -1) circle (2pt);
\draw[dashed, green] (0,1)--(1, 2)--(3, 0)--(2, -1)--(0,1);
    \end{tikzpicture}
  \end{center}
\item Dla funkcji $w=z^n$ obrazem pierwszej ćwiartki jest cała górna półpłaszczyzna
\item Funkcja $w=\frac{1}{z}$ to inwersja względem okręgu jednostkowego złożona z odbiciem względem osi $OX$:
  \begin{center}
    \begin{tikzpicture}
      \draw (-2, 0)--(2, 0);
      \draw (0, -2)--(0, 2);
      \draw (0,0) circle (1);

      \draw (0,0)--(50:1.8);
      \draw (0,0)--(-50:{10/18});
      
    \fill (50:1.8) circle (1pt);
    \fill (-50:{10/18}) circle (1pt);

    \node at (55:1.8) {$z$};
    \node at (-45:{14/18}) {$\frac{1}{z}$};
    \end{tikzpicture}
  \end{center}

  Możemy zapisać inaczej:
  $$w=\frac{1}{z}=\frac{1}{x+iy}=\frac{x}{x^2+y^2}+i\frac{-y}{x^2+y^2}$$
  czyli
  $$u=\frac{x}{x^2+y^2}\quad\quad v=\frac{-y}{x^2+y^2}$$
  $$x=\frac{u}{u^2+v^2}\quad\quad y=\frac{-v}{u^2+y^2}$$

  Pokażemy, że obrazem dowolnego okręgu przez tę funkcję jest nadal okrąg. Załóżmy, że $x$ i $y$ spełniają równanie
  $$a(x^2+y^2)+bx+cy+d=0.$$
  Podstawiając $w=\frac{1}{z}$ dostajemy
  $$a\left[\frac{u^2}{(u^2+y^2)^2}+\frac{v^2}{(u^2+v^2)^2}\right]+b\frac{u}{u^2+v^2}-c\frac{v}{v^2+y^2}+d=0$$
  $$d(u^2+v^2)+bu-cv+a=0$$
  czyli nadal jest to równanie okręgu. Jednak środek tego okręgu nie musi przechodzić na środek okręgu będącego obrazem.

  Dodatkowo, okrąg przechodzący przez $0$ stanie się prostą, natomiast okrąg który nie przechodzi przez $0$ będzie nadal okręgiem niedotykającym $0$.
\end{example}

\begin{definition}[homografia]
  Odwzorowania postaci
  $$w=\frac{\alpha z+\beta}{\gamma z+\delta}$$
  dla $\alpha,\beta,\gamma,\delta\in\C$ i $\alpha \delta-\beta\gamma \neq 0$ (liniowo niezależne) nazywamy \buff{homografiami}.
\end{definition}

Z warunku wyżej możemy wyliczyć $z$ w  zależności od $w$ i pozostałych stałych:
$$z=\frac{-\delta w+\beta}{\gamma w-\alpha}$$
i jest to nadal homotopia.

Dla $\gamma=0$ przekształcenia nazywamy afinicznymi. Jeśli jednak $\gamma\neq 0$, to mianownik zeruje się w punkcie $z=-\delta/\gamma$.

Dla $\gamma=0$ proste i okręgi przechodzą na proste i okręgi.

Jeśli $\gamma\neq 0$, to
$$w=\frac{\alpha z+\beta}{\gamma z+\delta}=\frac{\alpha}{\gamma}+\frac{\beta-\frac{\alpha\delta}{\gamma}}{\gamma z+\delta}$$
patrząc co się dzieje po kolei ze zmienną $z$:
\begin{align*}
  z\overset{aff.}{\mapsto} \gamma z+\delta \overset{1/z}{\mapsto} \frac{1}{\gamma z+\delta}\overset{aff.}{\mapsto} \frac{\alpha}{\delta}+\frac{\beta-\frac{\alpha\delta}{\gamma}}{\gamma z+\delta}
\end{align*}

Wniosek: homografia jest złożeniem przekształceń afinicznych i przekształcenia $z\mapsto \frac{1}{z}$, czyli zawsze przekształca proste i okręgi na proste i okręgi.

Innym sposobem określenia homografii jest przez równanie
$$Azw+Bz+Cw+D=0.$$
Wystarczy przemnożyć oryginalną definicję i przerzucić wszystko na jedną stronę.

W równaniu homografii występują tak naprawdę tylko $3$ istotne stałe: ilorazy $A$, $B$, $C$, $D$ przez jedną z nich (tę jedną niezerową).

\begin{example}
\item Wiemy, że równanie jest spełnione przez $3$ punkty $(z_1,w_1)$, $(z_2, w_2)$ i $(z_3, w_3)$ i chcemy znaleźć odpowiadającą temu homografię. Możemy podstawić do formy $Azw+Bz+Cw+D=0$ i pieczołowicie je wyszukać .
\end{example}
