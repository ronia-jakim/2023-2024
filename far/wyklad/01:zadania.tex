\subsection{Zadania}

\begin{problem}
  Zapisać w postaci $x+iy$ podane liczby.
  \begin{multicols*}{2}
    $${\color{orange}(3+2i)-(4-i)}=-1+i$$
    $${\color{orange}(1-i)^4}=1-4i-6+4i+1=-4$$
    $${\color{orange}(2-3i)(-2+i)}=-1+8i$$
    $${\color{orange}(3+i)(3-i)(\frac{2+i}{10})}=2+i$$
    $${\color{orange}i(2-7i)}=7+2i$$
    $${\color{orange}\frac{1+2i}{3-4i}+\frac{2-i}{5i}}=\frac{(1+2i)(3+4i)}{25}-\frac{5(1+2i)}{25}=\frac{-1}{5}$$
    $${\color{orange}\frac{1+i}{2-i}}=\frac{(1+i)(2+i)}{5}=\frac{1}{5}+i\frac{3}{5}$$
    $${\color{orange}\frac{2i}{(i-1)(i-2)(i-3)}}=\frac{2i(i+1)(i+2)(i+3)}{2\cdot 5\cdot 10}=-\frac{1}{5}$$
  \end{multicols*}
\end{problem}

\begin{problem}
  Rozwiązać równania
  $$z^2-2z+2=0;\quad 3z^2+2z+1=0$$
\end{problem}

\begin{solution}
  Uzupełniając pierwsze równanie do kwadratu
  \begin{align*}
    z^2-2z+1&=-1\\ 
    (z-1)^2&=-1
  \end{align*}
  jeśli $w=z-1$, to $w^2+1=0$, czyli $w=i$ lub $w=-i$. W takim razie
  $$z=1+i\quad\lor\quad z=1-i$$

  Znowu uzupełniając do kwadratu
  \begin{align*}
    3z^2+2z+1&=0\\ 
    3\left(z+\frac{1}{3}\right)^2+\frac{2}{3}&=0
  \end{align*}
  czyli dla $w=z+\frac{1}{3}$ mamy $w^2=-\frac{2}{9}$, a więc $w=\pm i\frac{\sqrt{2}}{3}$, co daje
  $$z=-\frac{1}{3}+i\frac{\sqrt2}{3}\quad\lor\quad z=-\frac{1}{3}-i\frac{\sqrt2}{3}$$
\end{solution}

\begin{problem}
  Uzasadnić, że jeśli $z_1z_2z_3=0$, to jeden z czynników jest równy zeru.
\end{problem}

\begin{solution}
  %Niech $z_j=x_j+iy_j$ dla $j=1,2,3$. Wtedy 
  %\begin{align*}
  %  z_1z_2z_3&=(x_1+iy_1)(x_2+iy_2)(x_3+iy_3)=(x_1x_2-y_1y_2+i(x_1y_2+y_1x_2))(x_3+iy_3)=\\ 
  %           &=(x_1x_2x_3-y_1y_2x_3-x_1y_2y_3-y_1x_2y_3+i(x_1y_2x_3+y_1x_2x_3+x_1x_2y_3-y_1y_2y_3))
  %\end{align*}
%
  Niech $z_j=r_j(\cos\theta_j+i\sin\theta_j)$ dla $j=1,2,3$. Wtedy
  $$0=z_1z_2z_3=r_1r_2r_3(\cos(\theta_1+\theta_2+\theta_3)+i\sin(\theta_1+\theta_2+\theta_3))$$
  ponieważ nawias jest zawsze niezerowy, to co najmniej jedno z $r_j$ musi wynosić $0$.
\end{solution}

\begin{problem}
  Uzasadnić zasady przemienności i łączności mnożenia:
  $$z_1z_2=z_2z_1\quad z_1(z_2z_3)=(z_1z_2)z_3$$
\end{problem}

\begin{solution}
  Niech $z_j=x_j+iy_j$ (chociaż korzystając z trygonometrycznej wersji byłoby zapewne szybciej). 

  \begin{align*}
    z_1z_2&=(x_1+iy_1)(x_2+iy_2)=x_1x_2-y_1y_2+i(x_1y_2+y_1x_2)=\\ 
          &=x_2x_1-y_2y_1+i(y_2x_1+x_2y_1)=(x_2+iy_2)(x_1+iy_1)=\\ 
          &=z_2z_1
  \end{align*}

\end{solution}

\begin{problem}
  Zaznaczyć na płaszczyźnie liczby $z_1$, $z_2$, $z_1+z_2$ oraz $z_1-z_2$.
  \begin{enumerate}[label=\alph*)]
    \item $z_1=-3+i$, $z_2=1+4i$
    \item $z_1=3$, $z_2=-3+5i$.
  \end{enumerate}
\end{problem}

\begin{solution}$ $

  \begin{center}
    \begin{tikzpicture}
      \draw[->] (-3, 0)--(3, 0);
      \draw[->] (0, -3)--(0, 3);
      \fill (-1.5, 0.5) circle (2pt);
      \fill (0.5, 2) circle (2pt);
      \node at (-1.5, 0.8) {$z_1$};
      \node at (0.5, 2.3) {$z_2$};
      \fill (-1, 2.5) circle (2pt);
      \node at (-1, 2.8) {$z_1+z_2$}; 
      \draw (-1.5, 0.5)--(0,0)--(0.5, 2);
      \draw (0,0)--(-1, 2.5);
      \draw[dashed] (-1.5, 0.5)--(-1, 2.5);
      \draw[dashed] (0.5, 2)--(-1, 2.5);
    \end{tikzpicture}
  \end{center}


  \begin{center}
    \begin{tikzpicture}
      \draw[->] (-3, 0)--(3, 0);
      \draw[->] (0, -3)--(0, 3);
      \fill (1.5, 0) circle (2pt);
      \fill (-1.5, 2.5) circle (2pt);
      \node at (1.5, 0.3) {$z_1$};
      \node at (-1.5, 2.8) {$z_2$};
      \fill (0, 2.5) circle (2pt);
      \node at (0.5, 2.7) {$z_1+z_2$}; 
      \draw (0,0)--(-1.5, 2.5);
      \draw[dashed] (-1.5, 2.5)--(0, 2.5)--(1.5, 0);
    \end{tikzpicture}
  \end{center}
\end{solution}

\begin{problem}
  Pokazać, że wektor przedstawiający sumę $z_1+z_2+z_3$ jest zamykającym bokiem czworokąta, o pozostałych bokach $z_1$, $z_2$ i $z_3$. Jaki jest kierunek tego wektora?
\end{problem}

\begin{solution}
  Zaczynamy od rysunku

  \begin{center}
    \begin{tikzpicture}
      \draw[->] (-3, 0)--(3, 0);
      \draw[->] (0, -3)--(0, 3);
        
      \coordinate (z1) at ({random(-2.5, 2.5)}, {random(-2.5, 2.5)});
      \coordinate (z2) at ({random(-2.5, 2.5)}, {random(-2.5, 2.5)});
      \coordinate (z3) at ({random(-2.5, 2.5)}, {random(-2.5, 2.5)});
      
      \fill[orange] (z1) circle (2pt);
      \fill[green] (z2) circle (2pt);
      \fill[blue] (z3) circle (2pt);

      \fill ($(z1) + (z2) + (z3)$) circle (2pt);
    \end{tikzpicture}
  \end{center}
\end{solution}
